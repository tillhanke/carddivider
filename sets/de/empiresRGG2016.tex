% Basic settings for this card set
\renewcommand{\cardcolor}{empires}
\renewcommand{\cardextension}{Erweiterung IX}
\renewcommand{\cardextensiontitle}{Empires}
\renewcommand{\seticon}{empires.png}

% \clearpage
% \newpage
\section{\cardextension \ - \cardextensiontitle \ (Rio Grande Games 2016)}

\begin{tikzpicture}
	\card
	\cardstrip
	\cardbanner{banner/white.png}
	\cardicon{icons/coin.png}
	\cardprice{2}
	\cardtitle{Feldlager}
	\cardcontent{Du darfst ein \emph{GOLD} oder ein \emph{DIEBESGUT} aus der Hand aufdecken. Wenn du das nicht kannst oder möchtest, legst du diese Karte zur Seite und legst sie zu Beginn deiner Aufräumphase zurück in den Vorrat. Sollte dort zu diesem Zeitpunkt bereits ein \emph{DIEBESGUT} offen liegen, muss nun erst wieder das zurückgelegte \emph{FELDLAGER} genommen werden, bevor das \emph{DIEBESGUT} genommen werden darf.}
\end{tikzpicture}
\hspace{-0.6cm}
\begin{tikzpicture}
	\card
	\cardstrip
	\cardbanner{banner/white.png}
	\cardicon{icons/coin.png}
	\cardprice{2}
	\cardtitle{Patrizier}
	\cardcontent{Du musst die oberste Karte deines Nachziehstapels aufdecken. Wenn der Stapel aufgebraucht ist, mischst du deinen Ablagestapel und legst ihn als Nachziehstapel bereit. Wenn auch dort keine Karten liegen, erhältst du nichts.

	\medskip

	Nur eine Karte, die mehr als \coin[5] kostet, darfst du auf die Hand nehmen. Ob die Karte außerdem noch Kosten in Form von Schulden \hex aufweist, ist dabei unerheblich (z.B. \emph{REICHTUM} darf auf die Hand genommen werden, \emph{STADTVIERTEL} nicht).}
\end{tikzpicture}
\hspace{-0.6cm}
\begin{tikzpicture}
	\card
	\cardstrip
	\cardbanner{banner/white.png}
	\cardicon{icons/coin.png}
	\cardprice{2}
	\cardtitle{Siedler}
	\cardcontent{Auch wenn du weißt, dass sich kein \emph{KUPFER} in deinem Ablagestapel befindet, darfst du ihn ansehen. Du musst kein \emph{KUPFER} auf die Hand nehmen, wenn du das nicht möchtest.}
\end{tikzpicture}
\hspace{-0.6cm}
\begin{tikzpicture}
	\card
	\cardstrip
	\cardbanner{banner/white.png}
	\cardicon{icons/coin.png}
	\cardprice{3}
	\cardtitle{\footnotesize{Bauernmarkt}}
	\cardcontent{Diese Karte beinhaltet den neuen Typ SAMMLUNG, d.h. hier kommen die Siegpunktmarker zum Einsatz.

	\medskip

	Wenn diese Karte das erste Mal ausgespielt wird, legt der Spieler einen \victorypointtoken-Marker auf den BAUERNMARKT-Vorratsstapel und erhält dann +\coin[1] für den gerade gelegten Marker. Wird die Karte zum zweiten, dritten und vierten Mal ausgespielt, legt der Spieler jeweils einen weiten Marker auf den Stapel und erhält +\coin[2], +\coin[3] bzw. +\coin[4], egal welcher Spieler die vorherigen Marker auf den Stapel gelegt hat. Wird die Karte danach erneut ausgespielt, nimmt der Spieler die 4 \victorypointtoken-Marker (dafür aber kein \coin) und muss den ausgespielten \emph{BAUERNMARKT} entsorgen.

	\medskip

	Danach beginnt der Vorgang wieder von vorn und wird fortgesetzt, falls der Vorratsstapel leer ist.}
\end{tikzpicture}
\hspace{-0.6cm}
\begin{tikzpicture}
	\card
	\cardstrip
	\cardbanner{banner/white.png}
	\cardicon{icons/coin.png}
	\cardprice{3}
	\cardtitle{Gladiator}
	\cardcontent{Wenn du mindestens 1 Handkarte hast, musst du diese aufdecken. Wenn dein linker Mitspieler keine Karte mit gleichem Namen aufdecken kann oder will (z.B. auch, wenn du keine Handkarte aufdecken konntest, weil du keine hast), erhältst du zusätzlich +\coin[1] . Sind noch Karten auf dem \emph{GLADIATOR}-Vorratsstapel vorhanden, musst du eine entsorgen. Deckt der Mitspieler eine Karte mit gleichem Namen auf, erhältst du nur +\coin[2] und darfst keinen \emph{GLADIATOR} entsorgen.}
\end{tikzpicture}
\hspace{-0.6cm}
\begin{tikzpicture}
	\card
	\cardstrip
	\cardbanner{banner/white.png}
	\cardicon{icons/coin.png}
	\cardprice{3}
	\cardtitle{Katapult}
	\cardcontent{Wenn du mindestens 1 Handkarte hast, musst du auch eine entsorgen. Kostet die entsorgte Karte \coin[3] oder mehr, nimmt sich jeder Mitspieler (beginnend bei deinem linken Nachbarn) einen \emph{FLUCH}. Karten mit Schulden kosten nur dann \coin[3] oder mehr, wenn sie zusätzlich zu etwaigen Schulden-Kosten mindestens \coin[3] kosten. Ist die entsorgte Karte eine Geldkarte muss jeder Mitspieler - unabhängig von den Kosten der Karte - seine Handkarten auf 3 reduzieren.}
\end{tikzpicture}
\hspace{-0.6cm}
\begin{tikzpicture}
	\card
	\cardstrip
	\cardbanner{banner/white.png}
	\cardicon{icons/coin.png}
	\cardprice{3}
	\cardtitle{\footnotesize{Wagenrennen}}
	\cardcontent{Nimm deine aufgedeckte Karte nach dem Vergleich der Kosten mit der aufgedeckten Karte deines linken Mitspielers auf die Hand. Der Mitspieler legt seine aufgedeckte Karte zurück auf den Nachziehstapel.

	Kosten beide Karten gleich viel oder kostet die Karte des Mitspielers mehr, erhältst du nichts. Kostet deine Karte mehr erhältst du +1 \coin und +1 \victorypointtoken-Marker. Hast entweder du oder dein linker Mitspieler (auch nach dem eventuellen Mischen des Ablagestapels) keine Karte zum Aufdecken, erhältst du nichts.}
\end{tikzpicture}
\hspace{-0.6cm}
\begin{tikzpicture}
	\card
	\cardstrip
	\cardbanner{banner/orange.png}
	\cardicon{icons/coin.png}
	\cardprice{3}
	\cardtitle{Zauberin}
	\cardcontent{Spieler, die mit einer Reaktionskarte wie dem \emph{BURGGRABEN} (aus dem Basisspiel) reagieren möchten, müssen dies tun, sobald die \emph{ZAUBERIN} ausgespielt wurde, auch wenn der Angriff sie erst in ihrem nächsten Zug betrifft.

	\medskip

	Jeder Mitspieler erhält in seinem nächsten Zug für die erste gespielte Aktionskarte + 1 Karte sowie + 1 Aktion, darf aber den eigentlichen Effekt der Karte beim Ausspielen nicht durchführen. Anweisungen, die sich auf einen anderen Zeitpunkt im Spiel beziehen (z.B. die beim Kauf der Karte zum Tragen kommen), werden nicht beeinflusst.

	Um anzuzeigen, dass die erste ausgespielte Aktionskarte von der \emph{ZAUBERIN} beeinflusst wird, empfehlen wir, diese beim Ausspielen quer auszulegen. Karten, die bereits ausgespielt wurden (z.B. Dauerkarten wie das \emph{ARCHIV}), werden zu Beginn des Zuges normal abgehandelt und nicht von der \emph{ZAUBERIN} beeinflusst. Spielt ein Spieler in seiner Aktionsphase keine Aktionskarte aus, dafür aber in seiner Kaufphase eine \emph{KRONE} (kombinierte Aktions- und Geldkarte), kommt der Effekt der \emph{ZAUBERIN} zum Tragen, da es sich um eine Aktionskarte handelt, auch wenn diese in der Kaufphase ausgespielt wurde. Normalerweise kann der Spieler die + 1 Aktion zu diesem Zeitpunkt nicht nutzen, es sei denn, er kauft zum Beispiel eine \emph{VILLA}.}
\end{tikzpicture}
\hspace{-0.6cm}
\begin{tikzpicture}
	\card
	\cardstrip
	\cardbanner{banner/gold.png}
	\cardicon{icons/coin.png}
	\cardprice{4}
	\cardtitle{Felsen}
	\cardcontent{Wenn du diese Karte in deiner Kaufphase nimmst oder entsorgst, nimm ein \emph{SILBER} und lege es auf deinen Nachziehstapel. Wenn du diese Karte zu einem anderen Zeitpunkt (auch während des Zuges eines anderen Spielers) nimmst oder entsorgst, nimm ein \emph{SILBER} auf die Hand.}
\end{tikzpicture}
\hspace{-0.6cm}
\begin{tikzpicture}
	\card
	\cardstrip
	\cardbanner{banner/white.png}
	\cardicon{icons/coin.png}
	\cardprice{4}
	\cardtitle{Opfer}
	\cardcontent{Wenn die entsorgte Karte eine kombinierte Karte ist, erhältst du die Boni aller entsprechenden Typen dieser Karte. Entsorgst du eine Karte, die keinem der angegebenen Typen entspricht (z.B. einen \emph{FLUCH}), erhältst du nichts.}
\end{tikzpicture}
\hspace{-0.6cm}
\begin{tikzpicture}
	\card
	\cardstrip
	\cardbanner{banner/white.png}
	\cardicon{icons/coin.png}
	\cardprice{4}
	\cardtitle{Tempel}
	\cardcontent{Es dürfen nur Karten mit unterschiedlichem Namen entsorgt werden, z.B. ein \emph{KUPFER} und ein \emph{ANWESEN}.

	Auch wenn der \emph{TEMPEL}-Vorratsstapel leer ist, legst du einen \victorypointtoken-Marker auf den leeren Platz. Das kann relevant werden, wenn durch Anweisungen auf anderen Karten ein \emph{TEMPEL} in den Vorrat zurückgelegt wird (z.B. durch den \emph{BOTSCHAFTER} aus \emph{Seaside}) nehmen darf.

	\medskip

	Wenn du einen \emph{TEMPEL} nimmst, nimmst du auch alle \victorypointtoken-Marker, die zu diesem Zeitpunkt auf dem Vorratsstapel liegen.}
\end{tikzpicture}
\hspace{-0.6cm}
\begin{tikzpicture}
	\card
	\cardstrip
	\cardbanner{banner/white.png}
	\cardicon{icons/coin.png}
	\cardprice{4}
	\cardtitle{Villa}
	\cardcontent{Wenn du diese Karte in deiner Aktionsphase nimmst (z.B. durch die \emph{INGENIEURIN}), nimm sie sofort auf die Hand und erhalte + 1 Aktion. Dadurch kannst du z.B. die gerade genommene \emph{VILLA} sofort ausspielen. Wenn du diese Karte in deiner Kaufphase nimmst (z.B. indem du sie kaufst), nimm sie auf die Hand und kehre sofort in die Aktionsphase zurück, wo du + 1 Aktion hast. Hast du die Aktionsphase erneut komplett abgeschlossen, kehrst du wieder zur Kaufphase zurück. Hier kannst du weitere Geldkarten ausspielen (und z.B. die \emph{ARENA} kommt wieder zum Tragen). Wenn du diese Karte während des Zuges eines Mitspielers nimmst, nimmst du die Karte auf die Hand und erhältst zwar + 1 Aktion, kannst diese aber nicht nutzen, da es nicht dein Zug ist. Es ist möglich, mehrmals pro Zug (z.B. durch das Nehmen mehrerer \emph{VILLEN}) in die Aktionsphase zurückzukehren. Dies bedeutet aber nicht, dass du an den \enquote{Beginn deines Zuges} zurückkehrst. Anweisungen, die sich darauf beziehen, haben keine Auswirkung.}
\end{tikzpicture}
\hspace{-0.6cm}
\begin{tikzpicture}
	\card
	\cardstrip
	\cardbanner{banner/orange.png}
	\cardicon{icons/coin.png}
	\cardprice{5}
	\cardtitle{Archiv}
	\cardcontent{Lege die obersten drei Karten deines Nachziehstapels zur Seite und schau sie dir an. Nimm eine der Karten sofort auf die Hand und lege die anderen Karten unter dieses \emph{ARCHIV}. Spielst du zwei \emph{ARCHIVE}, lege die Karten für die nächsten Züge unter das jeweils ausgespielte \emph{ARCHIV}. Hast du nicht genügend Karten, um drei Karten zur Seite zu legen, legst du nur so viele wie möglich zur Seite. Das \emph{ARCHIV} wird in dem Spielzug abgelegt, in dem die letzte zur Seite gelegte Karte des jeweiligen \emph{ARCHIVS} auf die Hand genommen wurde.}
\end{tikzpicture}
\hspace{-0.6cm}
\begin{tikzpicture}
	\card
	\cardstrip
	\cardbanner{banner/gold.png}
	\cardicon{icons/coin.png}
	\cardprice{5}
	\cardtitle{Diebesgut}
	\cardcontent{Nimm dir jedes Mal, wenn du diese Karte spielst, einen \victorypointtoken-Marker und lege ihn bei dir ab.}
\end{tikzpicture}
\hspace{-0.6cm}
\begin{tikzpicture}
	\card
	\cardstrip
	\cardbanner{banner/white.png}
	\cardicon{icons/coin.png}
	\cardprice{5}
	\cardtitle{Emsiges Dorf}
	\cardcontent{Du darfst deinen Ablagestapel auch dann durchsehen, wenn du weißt, dass du keine \emph{SIEDLER} darin hast. Du darfst die Reihenfolge der Karten in deinem Ablagestapel nicht verändern.}
\end{tikzpicture}
\hspace{-0.6cm}
\begin{tikzpicture}
	\card
	\cardstrip
	\cardbanner{banner/white.png}
	\cardicon{icons/coin.png}
	\cardprice{5}
	\cardtitle{Forum}
	\cardcontent{Wenn du diese Karte kaufst, erhältst du + 1 Kauf. Du kannst beispielsweise mit \coin[\hspace{-0.3em}13] und nur einem freien Kauf, zuerst diese Karte kaufen und dann mit dem zusätzlichen Kauf noch eine \emph{PROVINZ}.}
\end{tikzpicture}
\hspace{-0.6cm}
\begin{tikzpicture}
	\card
	\cardstrip
	\cardbanner{banner/white.png}
	\cardicon{icons/coin.png}
	\cardprice{5}
	\cardtitle{Gärtnerin}
	\cardcontent{Ist diese Karte im Spiel und du nimmst eine Punktekarte – egal in welcher Spielphase – nimmst du dir einen \victorypointtoken-Marker und legst ihn bei dir ab. Wenn du mehrere Karten nimmst, nimmst du dir für jede genommene Karten einen \victorypointtoken-Marker. Hast du mehrere \emph{GÄRTNERINNEN} im Spiel, nimmst du dir für jede \emph{GÄRTNERIN} pro genommener Karte einen \victorypointtoken-Marker.

	Wenn du z.B. eine \emph{GÄRTNERIN} auf eine \emph{KRONE} spielst, befindet sich die \emph{GÄRTNERIN} trotzdem nur einmal im Spiel und du darfst dir pro genommener Karte nur einen \victorypointtoken-Marker nehmen.}
\end{tikzpicture}
\hspace{-0.6cm}
\begin{tikzpicture}
	\card
	\cardstrip
	\cardbanner{banner/white.png}
	\cardicon{icons/coin.png}
	\cardprice{5}
	\cardtitle{\footnotesize{Handelsplatz}}
	\cardcontent{Zu den Aktionskarten, die du zu diesem Zeitpunkt im Spiel hast zählen alle Aktionskarten, die du ausgespielt hast, Dauerkarten, die sich aus vergangenen Zügen im Spiel befinden und Reservekarten (aus \emph{Abenteuer}), die du in diesem Zug bereits aufgerufen hast. Wenn du diese Karte außerhalb deines Zuges nimmst, hast du keine Aktionskarten im Spiel und du darfst dir keine \victorypointtoken-Marker nehmen.}
\end{tikzpicture}
\hspace{-0.6cm}
\begin{tikzpicture}
	\card
	\cardstrip
	\cardbanner{banner/whitegold.png}
	\cardicon{icons/coin.png}
	\cardprice{5}
	\cardtitle{Krone}
	\cardcontent{Diese Karte ist eine kombinierte Aktions- und Geldkarte. Wenn du sie in deiner Aktionsphase ausspielst, darfst du eine Aktionskarte von deiner Hand wählen und ausspielen. Du nimmst die gewählte Karte nicht wieder auf die Hand, sondern spielst die Aktion ein zweites Mal. Dafür benötigst du keine weiteren Aktionen. Wählst du eine \emph{KRONE}, musst du diese auch als Aktionskarte ausspielen (und dann darfst du bis zu zwei weitere Aktionskarten jeweils zweimal spielen).

	Spielst du diese Karte in deiner Aktionsphase als Geldkarte aus (z.B. durch den \emph{GESCHICHTENERZÄHLER} aus \emph{Abenteuer}), darfst du trotzdem eine Aktionskarte zweimal ausspielen.

	\medskip

	Spielst du diese Karte in deiner Kaufphase, darfst du eine beliebige Geldkarte von deiner Hand wählen, sie ausspielen und zweimal ausführen. Wählst du eine \emph{KRONE}, spielst du diese aus und dann eine weitere Geldkarte von der Hand zweimal und dann noch eine Geldkarte zweimal.}
\end{tikzpicture}
\hspace{-0.6cm}
\begin{tikzpicture}
	\card
	\cardstrip
	\cardbanner{banner/white.png}
	\cardicon{icons/coin.png}
	\cardprice{5}
	\cardtitle{Legionär}
	\cardcontent{Mitspieler, die auf das Ausspielen dieser Karte mit einer Reaktionskarte reagieren möchten, müssen dies tun, bevor du dich entscheidest, ob du ein \emph{GOLD} aufdeckst oder nicht.

	\medskip

	Mitspieler, die bereits zwei oder weniger Karten auf der Hand haben, müssen keine Karte ablegen, müssen gleichwohl aber eine Karte ziehen.}
\end{tikzpicture}
\hspace{-0.6cm}
\begin{tikzpicture}
	\card
	\cardstrip
	\cardbanner{banner/gold.png}
	\cardicon{icons/coin.png}
	\cardprice{5}
	\cardtitle{Vermögen}
	\cardcontent{Diese Karte ist eine Geldkarte mit zusätzlichen Anweisungen. Sie hat den Wert \coin[6]. Außerdem erhältst du + 1 Kauf.

	\medskip

	Wenn du diese Karte ablegst (in der Regel in deiner Aufräumphase), nimm \hex[6] vom Vorrat. Dann kannst du sofort beliebig viele \hex (auch mehr als die \hex[6], die du durch das Ablegen dieser Karte erhalten hast) zurückzahlen.

	\medskip

	Wenn du diese Karte nicht ablegst (z.B. wenn du sie stattdessen entsorgst), erhältst du keine \hex. Wenn du diese Karte zweimal ausgespielt hast (z.B. durch eine \emph{KRONE}), erhältst du trotzdem nur \hex[6], da du nur eine Karte ablegst.}
\end{tikzpicture}
\hspace{-0.6cm}
\begin{tikzpicture}
	\card
	\cardstrip
	\cardbanner{banner/white.png}
	\cardicon{icons/coin.png}
	\cardprice{5}
	\cardtitle{Wilde Jagd}
	\cardcontent{Wählst du die erste Option, lege einen \victorypointtoken-Marker vom Vorrat auf den \emph{WILDE-JAGD}-Vorratsstapel.

	\medskip

	Wählst du die zweite Option und der \emph{ANWESEN}-Vorratsstapel ist leer (d.h. du kannst dir kein \emph{ANWESEN} nehmen), darfst du dir die \victorypointtoken-Marker vom \emph{WILDE-JAGD}-Vorratsstapel nicht nehmen. Du darfst aber diese Option trotzdem wählen.

	\medskip

	Ist der \emph{WILDE-JAGD}-Vorratsstapel leer, funktioniert das Ausspielen dieser Karte trotzdem in der beschriebenen Weise weiter. Nutzt die Platzhalterkarte, um den Vorratsstapel zu markieren.}
\end{tikzpicture}
\hspace{-0.6cm}
\begin{tikzpicture}
	\card
	\cardstrip
	\cardbanner{banner/gold.png}
	\cardicon{icons/coin.png}
	\cardprice{5}
	\cardtitle{Zauber}
	\cardcontent{Wenn du diese Karte ausspielst und dich für die zweite Option entscheidest, darfst du (musst aber nicht) sofort, wenn du die \emph{nächste} Karte in deinem Zug kaufst, eine Karte mit anderem Namen nehmen, die \emph{exakt so viel} kostet, wie die gekaufte Karte. Dann erst nimmst du die gekaufte Karte. Das kann wichtig bei Karten sein, die Anweisungen beim Nehmen einer Karte beinhalten.

	Spielst du mehrere \emph{ZAUBER} in einem Zug, darfst du dir für die nächste gekaufte Karte mehrere Karten mit anderem Namen als die gekaufte aber gleichen Kosten nehmen. Die Karten, die du nimmst müssen zwar einen anderen Namen als die Gekaufte haben, dürfen aber untereinander alle den gleichen Namen haben.}
\end{tikzpicture}
\hspace{-0.6cm}
\begin{tikzpicture}
	\card
	\cardstrip
	\cardbanner{banner/white.png}
	\cardicon{icons/hex.png}
	\cardprice{\textcolor{white}{4}}
	\cardtitle{Ingenieurin}
	\cardcontent{Du darfst dir keine Karte nehmen, die mehr als \coin[4] kostet oder mit \hex in den Kosten hat. Nimm die gewählte Karte.

	Dann darfst du diese \emph{INGENIEURIN} entsorgen. Wenn du das tust, nimm eine weitere Karte, die bis zu \coin[4] kostet. Dies kann die gleiche Karte wie die erste sein oder eine andere.}
\end{tikzpicture}
\hspace{-0.6cm}
\begin{tikzpicture}
	\card
	\cardstrip
	\cardbanner{banner/white.png}
	\cardicon{icons/hex.png}
	\cardprice{\textcolor{white}{8}}
	\cardtitle{\miniscule{KöniglicherSchmied}}
	\cardcontent{Du musst, nachdem du 5 Karten nachgezogen hast, alle deine Handkarten vorzeigen und jedes \emph{KUPFER}, das du zu diesem Zeitpunkt auf der Hand hast, ablegen.}
\end{tikzpicture}
\hspace{-0.6cm}
\begin{tikzpicture}
	\card
	\cardstrip
	\cardbanner{banner/white.png}
	\cardicon{icons/hex.png}
	\cardprice{\textcolor{white}{8}}
	\cardtitle{Lehnsherr}
	\cardcontent{Wähle eine Karte vom Vorrat, die zu diesem Zeitpunkt bis zu \coin[5] kostet, d.h. du darfst keine Karte eines leeren Stapels, eine nicht sichtbare Karte eines gemischten Stapels oder eine Karte eines Nicht-Vorratsstapels wählen.

	\medskip

	Behandle nun den ausgespielten \emph{LEHNSHERR}, wie die gewählte Karte (und nicht mehr als \emph{LEHNSHERR}) – bis sie nicht mehr im Spiel ist. Das heißt du befolgst alle Anweisungen der anderen Karte. Auch nimmt der \emph{LEHNSHERR} den Namen, die Kosten und den Typ der gewählten Karte an, bis er nicht mehr im Spiel ist. Als Dauerkarte bleibt dieser \emph{LEHNSHERR} ebenso im Spiel, wie er als Reservekarte (aus \emph{Abenteuer}) zur Seite gelegt wird. Spielst du diesen \emph{LEHNSHERR} auf einen \emph{THRONSAAL} (aus dem \emph{Basisspiel}), wählst du beim ersten Ausspielen die Karte, die dieser \emph{LEHNSHERR} ab sofort ist – beim zweiten Ausspielen ist er damit wieder genau diese Karte – du darfst keine andere Karte wählen. Erst mit dem Ausspielen des \emph{LEHNSHERRN} nimmt er Typ und Namen der gewählten Karte an – d.h. du darfst ihn nicht als \emph{KRONE} in deiner Kaufphase spielen, da er selbst keine Geldkarte ist und nicht in der Kaufphase ausgespielt werden darf.}
\end{tikzpicture}
\hspace{-0.6cm}
\begin{tikzpicture}
	\card
	\cardstrip
	\cardbanner{banner/white.png}
	\cardicon{icons/hex.png}
	\cardprice{\textcolor{white}{8}}
	\cardtitle{Stadtviertel}
	\cardcontent{Du musst deine Handkarten aufdecken. Für jede Aktionskarte (auch ggf. kombinierte), die du aufdeckst, ziehst du eine Karte nach.}
\end{tikzpicture}
\hspace{-0.6cm}
\begin{tikzpicture}
	\card
	\cardstrip
	\cardbanner{banner/white.png}
	\cardicon{icons/coin.png}
	\cardprice{8}
	\cardiconaddition{icons/hex.png}
	\cardpriceaddition{\textcolor{white}{8}}
	\cardtitle{\quad Reichtum}
	\cardcontent{Es werden nur alle \coin verdoppelt, die du vor dem Ausspielen dieser Karte ausgespielt hast und nur, wenn du in diesem Zug noch keinen \emph{REICHTUM} ausgespielt hast. Für jedes weitere Ausspielen eines \emph{REICHTUMS} erhältst du nur + 1 Kauf.}
\end{tikzpicture}
\hspace{-0.6cm}
\begin{tikzpicture}
	\card
	\cardstrip
	\cardbanner{banner/green.png}
	\cardtitle{Schlösser (1/2)\quad}
	\cardcontent{\emph{Schloss-Karten:} Der Schloss-Stapel ist ein gemischter Vorratsstapel. Alle Schlösser werden nach Kosten sortiert auf dem Vorratsstapel bereitgelegt (die teuerste zuunterst).

	\bigskip

	\emph{Bescheidenes Schloss:} Spielst du sie in deiner Kaufphase aus, ist sie \coin[1] wert. Bei Spielende erhältst du pro Karte, die den Typ \emph{SCHLOSS} beinhaltet, einen \victorypointtoken-Marker.

	\medskip

	\emph{Verfallendes Schloss:} Diese Karte ist zu Spielende 1 \victorypoint wert – wie ein \emph{ANWESEN}. Wenn du diese Karte während des Spiels nimmst, nimm dir einen \victorypointtoken-Marker sowie ein \emph{SILBER} vom Vorrat. Wenn du diese Karte während des Spiels entsorgst, nimm dir einen weiteren \victorypointtoken-Marker sowie ein \emph{SILBER} vom Vorrat.

	\medskip

	\emph{Kleines Schloss:} Spielst du sie in deiner Aktionsphase aus, entsorge dieses \emph{KLEINE SCHLOSS} oder eine andere \emph{SCHLOSS}-Karte aus deiner Hand. Wenn du das tust, nimm dir die \emph{SCHLOSS}-Karte vom Vorratsstapel, die zu diesem Zeitpunkt oben liegt. Dies kann eine teurere sein, als die, die du entsorgst. Du musst die Kosten nicht bezahlen. Bei Spielende ist diese Karte 2 \victorypoint wert.}
\end{tikzpicture}
\hspace{-0.6cm}
\begin{tikzpicture}
	\card
	\cardstrip
	\cardbanner{banner/green.png}
	\cardtitle{Schlösser (2/2)\quad}
	\cardcontent{\emph{Spukschloss:} Diese Karte ist zu Spielende 2 \victorypoint wert. Wenn du diese Karte während deines Zuges nimmst (kaufst oder auf andere Art und Weise nimmst), nimm dir ein \emph{GOLD} vom Vorrat. Ist kein \emph{GOLD} mehr im Vorrat, erhältst du nichts. Außerdem (egal ob du ein \emph{GOLD} nehmen kannst oder nicht) müssen alle Mitspieler mit 5 oder mehr Handkarten 2 Handkarten auf ihren Nachziehstapel zurücklegen. Da diese Karte keine Angriffskarte ist, dürfen die Mitspieler keine Reaktionskarte spielen.

	\smallskip

	\emph{Reiches Schloss:} Spielst du sie in deiner Aktionsphase aus, lege beliebig viele Punktekarten (auch ggf. kombinierte) aus deiner Hand ab. Pro abgelegter Karte erhältst du +\coin[2]. Bei Spielende ist diese Karte 3 \victorypoint wert.

	\smallskip

	\emph{Ausgedehntes Schloss:} Wenn du diese Karte kaufst oder auf andere Art und Weise nimmst, nimm ein \emph{HERZOGTUM} oder drei \emph{ANWESEN}. Bei Spielende ist diese Karte 4 \victorypoint wert.

	\smallskip

	\emph{Prunkschloss:} Wenn du diese Karte kaufst oder auf andere Art und Weise nimmst, zeige deine Handkarten vor. Nimm einen \victorypointtoken-Marker vom Vorrat für jede Punktekarte (auch ggf. kombinierte), die du zu diesem Zeitpunkt auf der Hand oder im Spiel hast.

	\smallskip
 
	\emph{Königsschloss:} Bei Spielende erhältst du pro Karte, die den Typ \emph{SCHLOSS} beinhaltet (inklusive dieser Karte) 2 \victorypoint.}
\end{tikzpicture}
\hspace{-0.6cm}
\begin{tikzpicture}
	\card
	\cardstrip
	\cardbanner{banner/white.png}
	\cardtitle{Katapult/Felsen\qquad}
	\cardcontent{Spielvorbereitung: Legt auf diese Karte 5 Felsen und oben darauf 5 Katapulte.

	\bigskip

	Es darf immer nur die oberste Karte des Stapels genommen oder gekauft werden.}
\end{tikzpicture}
\hspace{-0.6cm}
\begin{tikzpicture}
	\card
	\cardstrip
	\cardbanner{banner/white.png}
	\cardtitle{\scriptsize{Gladiator/Reichtum}\qquad}
	\cardcontent{Spielvorbereitung: Legt auf diese Karte 5 Reichtum und oben darauf 5 Gladiatoren. 

	\bigskip

	Es darf immer nur die oberste Karte des Stapels genommen oder gekauft werden.}
\end{tikzpicture}
\hspace{-0.6cm}
\begin{tikzpicture}
	\card
	\cardstrip
	\cardbanner{banner/white.png}
	\cardtitle{\scriptsize{Siedler/Emsiges Dorf}\qquad}
	\cardcontent{Spielvorbereitung: Legt auf diese Karte 5 Emsige Dörfer und oben darauf 5 Siedler. 

	\bigskip

	Es darf immer nur die oberste Karte des Stapels genommen oder gekauft werden.}
\end{tikzpicture}
\hspace{-0.6cm}
\begin{tikzpicture}
	\card
	\cardstrip
	\cardbanner{banner/white.png}
	\cardtitle{\scriptsize{Patrizier/Handelsplatz}\qquad}
	\cardcontent{Spielvorbereitung: Legt auf diese Karte 5 Handelsplätze und oben darauf 5 Patrizier. 

	\bigskip

	Es darf immer nur die oberste Karte des Stapels genommen oder gekauft werden.}
\end{tikzpicture}
\hspace{-0.6cm}
\begin{tikzpicture}
	\card
	\cardstrip
	\cardbanner{banner/white.png}
	\cardtitle{\scriptsize{Feldlager/Diebesgut}\qquad}
	\cardcontent{Spielvorbereitung: Legt auf diese Karte 5 Diebesgut und oben darauf 5 Feldlager. 

	\bigskip

	Es darf immer nur die oberste Karte des Stapels genommen oder gekauft werden.}
\end{tikzpicture}
\hspace{-0.6cm}
\begin{tikzpicture}
	\card
	\cardstrip
	\cardbanner{banner/white.png}
	\cardtitle{Ereignisse (1/4)\quad}
	\cardcontent{Ereignisse können nur in der Kaufphase erworben werden. Dies benötigt 1 Kauf sowie genügend (vorher ausgespielte) Geldwerte. Erwirbst du ein Ereignis mit Schulden \hex, nimmst du die entsprechende Anzahl \hex-Marker an dich. Die Kosten (\hex \coin) sind auf jedem Ereignis oben links zu finden. Sobald du ein Ereignis erwirbst, führst du die darauf beschriebene Anweisung aus. Du nimmst das Ereignis aber \emph{nicht} an dich.

	\bigskip
 
	\emph{Aufstieg:} Wenn du keine Aktionskarte entsorgst passiert nichts weiter.

	\medskip
 
	\emph{Erforschen:} Jeder Erwerb eines \emph{ERFORSCHEN} gibt dir den Kauf zurück, den du für den Erwerb benötigt hast. Mit \coin[7] und 1 Kauf kannst du zum Beispiel 2 \emph{ERFORSCHEN} erwerben und dann eine Karte kaufen oder ein Ereignis für \coin[3] erwerben.}
\end{tikzpicture}
\hspace{-0.6cm}
\begin{tikzpicture}
	\card
	\cardstrip
	\cardbanner{banner/white.png}
	\cardtitle{Ereignisse (2/4)\quad}
	\cardcontent{\emph{Steuer:} Auf jeden Vorratsstapel (d.h. alle Königreichkarten, Fluchkarten und Basiskarten, nicht Ereignisse und Landmarken) wird in der Spielvorbereitung 1 \hex-Marker gelegt. Spieler, die eine Karte von einem Stapel kaufen, auf dem \hex-Marker liegen, müssen alle Marker des Stapels nehmen. Nimmt ein Spieler eine Karte auf andere Art und Weise (d.h. er kauft sie nicht), werden eventuelle \hex-Marker auf die nächste Karte des Vorratsstapels gelegt. Wenn du dieses Ereignis erwirbst, legst du 2 \hex-Marker auf einen beliebigen Vorratsstapel – egal ob dort zu diesem Zeitpunkt bereits \hex-Marker liegen oder nicht.

	\medskip
 
	\emph{Bankett:} Du kannst dieses Ereignis auch kaufen, wenn der \emph{KUPFER}-Vorratsstapel aufgebraucht ist.

	\medskip
 
	\emph{Versalztes Land:} Wenn die entsorgte Karte eine Anweisung beinhaltet, die eintritt, wenn diese Karte entsorgt wird, musst du diese Anweisung ausführen.

	\medskip
 
	\emph{Ritual:} Wenn du keinen \emph{FLUCH} nehmen kannst (z.B. weil der Vorratsstapel leer ist), passiert nichts. Es werden nur die \coin-Kosten gezählt – für \hex-Kosten oder \potion-Kosten (aus \emph{Alchemisten}) erhältst du nichts.}
\end{tikzpicture}
\hspace{-0.6cm}
\begin{tikzpicture}
	\card
	\cardstrip
	\cardbanner{banner/white.png}
	\cardtitle{Ereignisse (3/4)\quad}
	\cardcontent{\emph{Glücksfall:} Wenn weniger als 3 \emph{GOLD} im Vorrat sind, nimm dir die restlichen \emph{GOLD}.

	\medskip
 
	\emph{Eroberung:} Pro \emph{SILBER}, das du in diesem Zug genommen hast (inklusive der 2 \emph{SILBER} durch diese Karte), nimm dir einen \victorypointtoken-Marker vom Vorrat. Dies ist kumulativ. Erwirbst du z.B. eine \emph{EROBERUNG} und erhältst dafür 2 \victorypointtoken-Marker (für die beiden SILBER durch diese Karte) und dann noch eine \emph{EROBERUNG}, für die du 2 \emph{SILBER} nehmen kannst, erhältst du für die zweite \emph{EROBERUNG} schon 4 \victorypointtoken-Marker. Sind nicht genügend \emph{SILBER} im Vorrat, nimmst du dir so viele wie möglich. Dann erhältst du aber auch entsprechend weniger \victorypointtoken-Marker.

	\medskip
 
	\emph{Beherrschen:} Ist der \emph{PROVINZ}-Vorratsstapel leer oder du kannst aus einem anderen Grund keine \emph{PROVINZ} nehmen, hat dieses Ereignis keine Auswirkung.

	\medskip
 
	\emph{Hochzeit:} Den \victorypointtoken-Marker nimmst du in jedem Fall – auch wenn der \emph{GOLD}-Vorratsstapel leer ist.}
\end{tikzpicture}
\hspace{-0.6cm}
\begin{tikzpicture}
	\card
	\cardstrip
	\cardbanner{banner/white.png}
	\cardtitle{Ereignisse (4/4)\quad}
	\cardcontent{\emph{Siegeszug:} Wenn du ein \emph{ANWESEN} nimmst, nimmst du für jede Karte, die du in diesem Zug bereits genommen hast (inklusive dem \emph{ANWESEN} jedoch nicht für Ereignisse), einen \victorypointtoken-Marker. Wenn du kein \emph{ANWESEN} nehmen kannst (z.B. weil der Vorratsstapel leer ist), passiert nichts.

	\medskip
 
	\emph{Schlacht:} Du kannst dieses Ereignis auch erwerben wenn der \emph{HERZOGTUM}-Vorratsstapel leer ist. Die bis zu ausgewählten 5 Karten verbleiben in deinem Ablagestapel. Die restlichen Karten mischst du in deinen Nachziehstapel.

	\medskip
 
	\emph{Spende:} Befinden sich unter den entsorgten Karten welche, die Anweisungen beinhalten, die beim Entsorgen ausgeführt werden, musst du diese ausführen, bevor du die restlichen Karten mischst. Die \emph{SPENDE} wird erst nach dem Zug, in dem sie erworben wird, ausgeführt (d.h. zwischen zwei Zügen). Damit hat zum Beispiel die \emph{BESESSENHEIT} (aus \emph{Alchemisten}) auf diese Anweisung keine Auswirkung.}
\end{tikzpicture}
\hspace{-0.6cm}
\begin{tikzpicture}
	\card
	\cardstrip
	\cardbanner{banner/green.png}
	\cardtitle{\footnotesize{Landmarken (1/8)}\quad}
	\cardcontent{Einige Landmarken enthalten Anweisungen für die Spielvorbereitung (unterhalb der Trennlinie). Spielt ihr mit einer dieser Karten, beachtet dies in der Spielvorbereitung.

	\medskip
 
	Darfst du dir auf Grund einer Anweisung \victorypointtoken-Marker von einer Landmarkenkarte oder einem Vorratsstapel nehmen und dort sind zu diesem Zeitpunkt keine \victorypointtoken-Marker vorhanden, erhältst du nichts. Sind die zu Spielbeginn platzierten \victorypointtoken-Marker aufgebraucht, werden keine neuen \victorypointtoken-Marker platziert.

	\bigskip
 
	\emph{Aquädukt:} Wenn du eine Geldkarte von einem Vorratsstapel nimmst, auf dem ein oder mehrere \victorypointtoken-Marker liegen (auch ggf. kombinierte Karten oder \emph{KUPFER}, wenn dort durch Anweisungen auf Karten oder Ereignissen \victorypointtoken-Marker platziert wurden), nimm einen \victorypointtoken-Marker und lege ihn hierher auf das \emph{AQUÄDUKT}.

	\smallskip
 
	Wenn du eine Punktekarte (auch ggf. kombinierte) nimmst, nimm dir alle \victorypointtoken{\ }-Marken, die zu diesem Zeitpunkt hier auf dem \emph{AQUÄDUKT} liegen.

	\smallskip
 
	Wenn du eine kombinierte Geld- und Punktekarte nimmst, kannst du dich entscheiden, in welcher Reihenfolge du die Anweisungen ausführst.}
\end{tikzpicture}
\hspace{-0.6cm}
\begin{tikzpicture}
	\card
	\cardstrip
	\cardbanner{banner/green.png}
	\cardtitle{\footnotesize{Landmarken (2/8)}\quad}
	\cardcontent{\emph{Arena:} Beginnst du (z.B. durch die \emph{VILLA}) in deinem Zug mehrfach mit deiner Kaufphase, kannst du die \emph{ARENA} mehrfach nutzen.

	\medskip
 
	\emph{Badehaus:} Egal ob du eine Karte kaufst oder auf andere Art und Weise nimmst (bzw. nehmen musst) – erhältst du in diesem Fall keine \victorypointtoken{\ }-Marker vom \emph{BADEHAUS}. Wer ein Ereignis erwirbt, nimmt damit keine Karte und kann – insofern keine andere Karte genommen wurde - 2 \victorypointtoken-Marker von hier nehmen.

	\medskip
 
	\emph{Basilika:} Für jede Karte die du kaufst, nimmst du 2 \victorypointtoken-Marker von der \emph{BASILIKA}, falls du zu diesem Zeitpunkt mindestens \coin[2] ausgespielt aber noch nicht verbraucht hast. Hast du beispielsweise \coin[4] und 3 Käufe, kannst du ein \emph{KUPFER} kaufen (\coin[4] übrig), dir 2 \victorypointtoken-Marker nehmen, ein \emph{ANWESEN} kaufen (\coin[2] übrig), dir 2 \victorypointtoken-Marker nehmen und ein weiteres \emph{ANWESEN} kaufen (\emph{0} übrig) – für den letzten Kauf erhältst du keine \victorypointtoken-Marker.}
\end{tikzpicture}
\hspace{-0.6cm}
\begin{tikzpicture}
	\card
	\cardstrip
	\cardbanner{banner/green.png}
	\cardtitle{\footnotesize{Landmarken (3/8)}\quad}
	\cardcontent{\emph{Bollwerk:} Hier werden alle Geldkarten (auch ggf. kombinierte) ausgewertet, die im Spiel benutzt wurden (auch ggf. Geldkarten, die im \emph{SCHWARZMARKT} (aus Basisspiel \emph{Special Edition} bzw. \emph{Promokarte}) enthalten waren). Haben zwei oder mehrere Spieler die gleiche höchste Anzahl einer Geldkarte, erhalten alle diese Spieler 5 \victorypointtoken-Marker.

	\medskip
 
	\emph{Brunnen:} Du erhältst entweder 15 \victorypoint oder 0 \victorypoint. Es gibt keinen Extra-Bonus, wenn du mehr als 10 \emph{KUPFER} besitzt.

	\medskip
 
	\emph{Entweihter Schrein:} Immer wenn du eine beliebige Aktionskarte nimmst und auf dem entsprechenden Vorratsstapel ein oder mehrere \victorypointtoken-Marker liegen (egal ob sie dort auf Grund der Anweisung auf dieser Landmarken-Karte oder einer anderen Karte, Ereignis oder Landmarken-Karte liegen), nimm einen \victorypointtoken-Marker von dort und lege ihn hierher auf den \emph{ENTWEIHTEN SCHREIN}.

	\smallskip
 
	Nur wenn du einen \emph{FLUCH} kaufst (nicht, wenn du ihn auf andere Art und Weise nimmst), nimmst du alle \victorypointtoken-Marker, die zu diesem Zeitpunkt hier liegen.

	In der Spielvorbereitung legt ihr auf jeden Vorratsstapel, der den Typ AKTION, nicht aber den Typ SAMMLUNG (also nicht auf die Karten \emph{BAUERNMARKT}, \emph{TEMPEL} und \emph{WILDE JAGD}) beinhaltet, 2 \victorypointtoken-Marker.}
\end{tikzpicture}
\hspace{-0.6cm}
\begin{tikzpicture}
	\card
	\cardstrip
	\cardbanner{banner/green.png}
	\cardtitle{\footnotesize{Landmarken (4/8)}\quad}
	\cardcontent{\emph{Gebirgspass:} Diese Landmarken-Karte wird genau einmal pro Spiel ausgeführt – nämlich nach Beendigung des Zuges, in dem ein Spieler die erste \emph{PROVINZ} aus dem Vorrat nimmt. Entsorgt vorher ein Spieler bereits eine \emph{PROVINZ} (z.B. durch das Ereignis \emph{VERSALZTES LAND}), hat jener Spieler diese \emph{PROVINZ} aber nicht vorher genommen und erfüllt deshalb diese Bedingung auch noch nicht. In einem Spiel, indem keine \emph{PROVINZ} genommen wird, findet diese Landmarken-Karte keine Anwendung.

	\smallskip
 
	Der \emph{GEBIRGSPASS} wird zwischen zwei Zügen ausgeführt und kann damit z.B. von der \emph{BESESSENHEIT} (aus \emph{Alchemisten}) nicht beeinflusst werden. Der Mitspieler links von dem Spieler, der die erste \emph{PROVINZ} genommen hat, beginnt mit einem Gebot oder passt. Ein Gebot besteht aus einer Anzahl \hex zwischen \hex[1] und \hex[\hspace{-0.25em}40]. Der nächste Spieler muss mindestens \hex[1] mehr bieten als der vorherige oder passen. Ein Gebot von \hex[\hspace{-0.25em}40] kann nicht überboten werden. Haben alle Spieler ein Gebot abgegeben oder gepasst, bzw. wurde bereits das Höchstgebot von \hex[\hspace{-0.25em}40] erreicht, erhält der Spieler mit dem höchsten Gebot die entsprechende Anzahl \hex-Marker sowie 8 \victorypointtoken-Marker. Passen alle Spieler, erhält keiner etwas.}
\end{tikzpicture}
\hspace{-0.6cm}
\begin{tikzpicture}
	\card
	\cardstrip
	\cardbanner{banner/green.png}
	\cardtitle{\footnotesize{Landmarken (5/8)}\quad}
	\cardcontent{\emph{Grabmal:} Dies funktioniert auch außerhalb deines Zuges (z.B. mit dem \emph{TRICKSER} aus \emph{Intrige}) oder wenn du eine Karte entsorgst, die nicht deine eigene ist (z.B. durch das Ereignis \emph{VERSALZTES LAND}).

	\medskip
 
	\emph{Kolonnaden:} Wenn du eine Aktionskarte kaufst (nicht, wenn du sie auf andere Art und Weise nimmst), musst du eine Karte mit dem gleichen Namen bereits im Spiel haben, um 2 \victorypointtoken-Marker von hier zu erhalten. Karten eines Stapels haben nicht unbedingt alle den gleichen Namen (z.B. bei gemischten Stapeln).

	\medskip
 
	\emph{Labyrinth:} Dies kann nur einmal pro Zug eines Spielers eintreten, nämlich genau in dem Moment, in dem er die zweite Karte in seinem Zug nimmt. Nimmt er außerhalb seines Zuges zwei Karten, erhält er nichts.

	\medskip
 
	\emph{Mauer:} Hast du mehr als 15 Karten in deinem Kartensatz, bekommst für jede Karte darüber hinaus 1 \victorypoint. Spieler, die zum Beispiel 27 Karten im Kartensatz haben, erhalten -12 \victorypoint, Spieler mit 14 Karten im Kartensatz erhalten keinen \victorypoint Abzug. Die Gesamtpunktzahl kann damit auch negativ sein.}
\end{tikzpicture}
\hspace{-0.6cm}
\begin{tikzpicture}
	\card
	\cardstrip
	\cardbanner{banner/green.png}
	\cardtitle{\footnotesize{Landmarken (6/8)}\quad}
	\cardcontent{\emph{Museum:} Auch Karten, die vom gleichen Stapel stammen, aber unterschiedliche Namen haben (z.B. gemischte Stapel), werden mit jeweils 2 \victorypoint abgerechnet.

	\medskip
 
	\emph{Obelisk:} Es zählen alle Karten des gewählten Stapels, auch wenn sie unterschiedliche Namen haben (z.B. bei gemischten Stapeln).

	\smallskip
 
	Zu Spielbeginn ermittelt ihr einen zufälligen Stapel, der den Typ AKTION beinhaltet (auch ggf. kombinierte Karten) und zum Vorrat gehört. \emph{RUINEN} (aus \emph{Dark Ages}) können bestimmt werden, ebenfalls der Stapel, der auch als Bannstapel für die \emph{JUNGE HEXE} (aus \emph{Reiche Ernte}) genutzt wird. Dazu zählen jedoch nicht die Eintausch- und Preiskarten (aus \emph{Reiche Ernte}), da diese nicht zum Vorrat gehören.

	\medskip
 
	\emph{Obstgarten:} Du erhältst keinen zusätzlichen Bonus, wenn du zum Beispiel von einer Aktionskarte 6 Exemplare besitzt, d.h. du erhältst für eine Aktionskarte, von der du 3 Exemplare besitzt genauso 4 \victorypoint wie für eine, von der du 7 Exemplare besitzt.}
\end{tikzpicture}
\hspace{-0.6cm}
\begin{tikzpicture}
	\card
	\cardstrip
	\cardbanner{banner/green.png}
	\cardtitle{\footnotesize{Landmarken (7/8)}\quad}
	\cardcontent{\emph{Palast:} Wenn du bei Spielende beispielsweise 7 \emph{KUPFER}, 5 \emph{SILBER} und 2 \emph{GOLD} in deinem Kartensatz hast, erhältst du 6 \victorypoint, da du zwei komplette Sätze aus je 1 \emph{KUPFER}, \emph{SILBER} und \emph{GOLD} besitzt. Hättest du noch ein drittes \emph{GOLD}, würdest du 9\victorypoint  erhalten.

	\medskip
 
	\emph{Räuberfestung:} Hast du bei Spielende zum Beispiel 3 \emph{SILBER} und 1 \emph{GOLD} in deinem Kartensatz, werden dir 8 \victorypoint abgezogen. Die Gesamtpunktzahl kann damit auch negativ sein.

	\medskip
 
	\emph{Schlachtfeld:} Du erhältst 2 \victorypointtoken-Marker von hier, egal ob du die Punktekarte (auch ggf. kombinierte) kaufst oder auf andere Art und Weise nimmst. Dies funktioniert auch außerhalb deines Zuges. Falls mehrere Spieler eine Punktekarte nehmen, wird dies in Spielerreihenfolge (beginnend bei dem Spieler links des aktuellen Spielers) getan.

	\medskip
 
	\emph{Triumphbogen:} Wenn du bei Spielende beispielsweise 7 \emph{VILLEN} und 4 \emph{WILDE JAGDEN} (und keine andere (auch ggf. kombinierte) Aktionskarte häufiger) in deinem Kartensatz hast, erhältst du 12 \victorypoint (d.h. 3 \victorypoint für jede der 4 \emph{WILDE JAGDEN}). Hast du neben 7 \emph{VILLEN} auch 7 \emph{WILDE JAGDEN}, erhältst du für beide zusammen 21 \victorypoint.}
\end{tikzpicture}
\hspace{-0.6cm}
\begin{tikzpicture}
	\card
	\cardstrip
	\cardbanner{banner/green.png}
	\cardtitle{\footnotesize{Landmarken (8/8)}\quad}
	\cardcontent{\emph{Turm:} Der Vorratsstapel muss leer sein. Ein gemischter Stapel, bei dem nur eine Sorte Karten fehlt, zählt nicht. Die Vorratsstapel mit Punktekarten zählen ebenfalls nicht, ein leerer Fluch-Stapel aber schon.

	\medskip
 
	\emph{Wolfsbau:} Du bekommst keine Minuspunkte durch den \emph{WOLFSBAU}, wenn du von einer Karte gar keine bzw. zwei oder mehr Stück in deinem kompletten Kartensatz besitzt. Hast du zum Beispiel einen \emph{FLUCH} in deinem Nachziehstapel und einen in deinem Ablagestapel, hast du insgesamt zwei \emph{FLÜCHE} und erhältst keine Minuspunkte durch den \emph{WOLFSBAU}. Die Gesamtpunktzahl kann negativ sein.}
\end{tikzpicture}
\hspace{-0.6cm}
\begin{tikzpicture}
	\card
	\cardstrip
	\cardbanner{banner/white.png}
	\cardtitle{\scriptsize{Empfohlene 10er Sätze\qquad}}
	\cardcontent{\emph{Basis Einführung} (\underline{Ereignisse und Landmarken-Karten}):\\
	\underline{Turm}, \underline{Hochzeit}, Schlösser (alle 8 bzw. 12 Schlosskarten), Wagenrennen, Stadtviertel, Ingenieurin, Bauernmarkt, Forum, Legionär, Patrizier/Handelsplatz, Opfer, Villa

	\smallskip

	\emph{Fortgeschrittene Einführung} (\underline{Ereignisse und Landmarken-Karten}):\\
	\underline{Arena}, \underline{Triumphbogen}, \underline{Hochzeit}, \underline{Spende}, Archiv, Vermögen, Katapult/Felsen, Krone, Zauberin, Gladiator/Reichtum, Gärtnerin, Königlicher Schmied, Siedler/Emsiges Dorf, Tempel

	\smallskip

	\emph{Alles in Maßen} (Empires + \underline{Ereignisse und Landmarken-Karten} + \textit{Basisspiel}):\\
	\underline{Obstgarten}, \underline{Glücksfall}, Zauberin, Forum, Legionär, Lehnsherr, Tempel, \textit{Keller}, \textit{Bibliothek}, \textit{Umbau}, \textit{Dorf}, \textit{Werkstatt}

	\smallskip

	\emph{Silberne Kugeln} (Empires + \underline{Ereignisse und Landmarken-Karten} + \textit{Basisspiel}):\\
	\underline{Aquädukt}, \underline{Eroberung}, Katapult/Felsen, Zauber, Bauernmarkt, Gärtnerin, Patrizier/Handelsplatz, \textit{Bürokrat}, \textit{Gärtner}, \textit{Laboratorium}, \textit{Markt}, \textit{Geldverleiher}}
\end{tikzpicture}
\hspace{-0.6cm}
\begin{tikzpicture}
	\card
	\cardstrip
	\cardbanner{banner/white.png}
	\cardtitle{\scriptsize{Empfohlene 10er Sätze\qquad}}
	\cardcontent{\emph{Köstliche Folter} (Empires + \underline{Ereignisse und Landmarken-Karten} + \textit{Intrige}):\\
	\underline{Arena}, \underline{Bankett}, Schlösser (alle 8 bzw. 12 Schlosskarten), Krone, Gärtnerin, Opfer, Siedler/Emsiges Dorf, \textit{Baron}, \textit{Brücke}, \textit{Harem}, \textit{Eisenhütte}, \textit{Kerkermeister}

	\smallskip

	\emph{Buddy-Prinzip} (Empires + \underline{Ereignisse und Landmarken-Karten} + \textit{Intrige}):\\
	\underline{Aussaat}, \underline{Wolfsbau}, Archiv, Vermögen, Katapult/Felsen, Ingenieurin, Forum, \textit{Maskerade}, \textit{Bergwerk}, \textit{Adlige}, \textit{Handlanger}, \textit{Handelsposten}

	\smallskip

	\emph{Kontrollbereich} (Empires + \underline{Ereignisse und Landmarken-Karten} + \textit{Abenteuer}):\\
	\underline{Bankett}, \underline{Bollwerk}, Vermögen, Katapult/Felsen, Zauber, Krone, Bauernmarkt, \textit{Königliche Münzen}, \textit{Page}, \textit{Relikt}, \textit{Schatz}, \textit{Weinhändler}

	\smallskip

	\emph{Kein Geld, keine Probleme} (Empires + \underline{Ereignisse und Landmarken-Karten} + \textit{Abenteuer}):\\
	\underline{Räuberfestung}, Archiv, Feldlager/Diebesgut, Königlicher Schmied, Tempel, Villa, \textit{Mission}, \textit{Verlies}, \textit{Duplikat}, \textit{Gefolgsmann}, \textit{Kleinbauer}, \textit{Transformation}}
\end{tikzpicture}
\hspace{-0.6cm}
\begin{tikzpicture}
	\card
	\cardstrip
	\cardbanner{banner/white.png}
	\cardtitle{Platzhalter\quad}
\end{tikzpicture}
\hspace{0.6cm}
