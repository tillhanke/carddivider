% Basic settings for this card set
\renewcommand{\cardcolor}{prosperity}
\renewcommand{\cardextension}{Update Pack}
\renewcommand{\cardextensiontitle}{Blütezeit}
\renewcommand{\seticon}{prosperity.png}

\clearpage
\newpage
\section{\cardextension \ - \cardextensiontitle \ (Rio Grande Games 2022)}

\begin{tikzpicture}
	\card
	\cardstrip
	\cardbanner{banner/gold.png}
	\cardicon{icons/coin.png}
	\cardprice{3}
	\cardtitle{Amboss}
	\cardcontent{Eine Geldkarte abzulegen, ist optional. Wenn du eine ablegst, nimmst du eine Karte aus dem Vorrat, die bis zu \coin[4] kostet, auf deinen Ablagestapel.}
\end{tikzpicture}
\hspace{-0.6cm}
\begin{tikzpicture}
	\card
	\cardstrip
	\cardbanner{banner/gold.png}
	\cardicon{icons/coin.png}
	\cardprice{4}
	\cardtitle{Brautkrone}
	\cardcontent{Wenn du später im Zug mehrere Karten nimmst, nachdem du die \emph{BRAUTKRONE} gespielt hast, darfst du eine beliebige Anzahl davon auf deinen Nachziehstapel legen. Dies betrifft Karten, die du kaufst oder auf eine andere Art und Weise nimmst, wie z.B. durch die \emph{WAFFENKISTE}.

	\medskip 

	Wenn du eine \emph{BRAUTKRONE} mit einer \emph{BRAUTKRONE} spielst, darfst du 2 weitere Geldkarten aus deiner Hand je zweimal spielen, nicht etwa eine Geldkarte viermal.}
\end{tikzpicture}
\hspace{-0.6cm}
\begin{tikzpicture}
	\card
	\cardstrip
	\cardbanner{banner/blue.png}
	\cardicon{icons/coin.png}
	\cardprice{4}
	\cardtitle{Buchhalterin}
	\cardcontent{Für Spieler mit einem leeren Nachziehstapel wird die abgelegte Karte zur einzigen Karte ihres Nachziehstapels.

	\medskip

	Zu Beginn deines Zuges darfst du beliebig viele \emph{BUCHHALTERINNEN} aus deiner Hand einzeln nacheinander spielen, ohne dafür Aktionen zu verwenden.}
\end{tikzpicture}
\hspace{-0.6cm}
\begin{tikzpicture}
	\card
	\cardstrip
	\cardbanner{banner/gold.png}
	\cardicon{icons/coin.png}
	\cardprice{4}
	\cardtitle{Geldanalge}
	\cardcontent{Du musst eine Karte entsorgen. Nur falls die \emph{GELDANLAGE} deine letzte Handkarte war, entsorgst du keine Karte. Dann entscheidest du dich entweder für +\coin[1] oder dafür, die \emph{GELDANLAGE} zu entsorgen. Wenn du sie entsorgst, deckst du deine Handkarten auf und erhältst +\emph{1} \victorypoint pro Geldkarte mit unterschiedlichem Namen auf deiner Hand. Wenn du z.B. 2 \emph{KUPFER} und ein \emph{SILBER} aufdeckst, erhältst du +\emph{2} \victorypoint. Du darfst die aufgedeckten Geldkarten weiterhin spielen, nachdem du die \emph{GELDANLAGE} ausgeführt hast.}
\end{tikzpicture}
\hspace{-0.6cm}
\begin{tikzpicture}
	\card
	\cardstrip
	\cardbanner{banner/gold.png}
	\cardicon{icons/coin.png}
	\cardprice{5}
	\cardtitle{Kristallkugel}
	\cardcontent{Wenn du dich gegen alle diese Alternativen entscheidest, musst du die Karte auf deinen Nachziehstapel zurücklegen.

	\medskip

	Wenn du durch die \emph{KRISTALLKUGEL} eine Aktionskarte während deiner Kaufphase spielst, durch die du +Aktionen erhältst, darfst du trotzdem keine weiteren Aktionskarten in deiner Kaufphase spielen; wenn du durch die Aktionskarte Geldkarten erhältst, darfst du diese noch spielen.}
\end{tikzpicture}
\hspace{-0.6cm}
\begin{tikzpicture}
	\card
	\cardstrip
	\cardbanner{banner/white.png}
	\cardicon{icons/coin.png}
	\cardprice{5}
	\cardtitle{Magnatin}
	\cardcontent{Wenn du z.B. zwei \emph{KUPFER} und ein \emph{SILBER} auf der Hand hast, ziehst du 3 Karten.}
\end{tikzpicture}
\hspace{-0.6cm}
\begin{tikzpicture}
	\card
	\cardstrip
	\cardbanner{banner/gold.png}
	\cardicon{icons/coin.png}
	\cardprice{5}
	\cardtitle{\footnotesize{Sammelsurium}}
	\cardcontent{Du erhältst +1 \victorypoint für jede Aktionskarte, die du kaufst oder auf eine andere Art und Weise nimmst. Mehrere \emph{SAMMELSURIEN} wirken kumulativ. Wenn du z.B. 2 \emph{SAMMELSURIEN} im Spiel hast und ein \emph{DORF} kaufst, erhältst du +2 \victorypoint.}
\end{tikzpicture}
\hspace{-0.6cm}
\begin{tikzpicture}
	\card
	\cardstrip
	\cardbanner{banner/gold.png}
	\cardicon{icons/coin.png}
	\cardprice{5}
	\cardtitle{Waffenkiste}
	\cardcontent{Mit der ersten \emph{WAFFENKISTE}, die du in einem Zug spielst, kannst du nicht die vom Mitspieler genannte Karte nehmen. Mit der zweiten \emph{WAFFENKISTE} darfst du die dann genannte Karte nicht nehmen und die Karte, die der Mitspieler bei der ersten \emph{WAFFENKISTE} genannt hat, usw. Die Karte nimmst du aus dem Vorrat, und legst sie auf deinen Ablagestapel. Du darfst die genannten Karten trotzdem auf andere Arten und Weisen nehmen, aber nicht mit einer \emph{WAFFENKISTE}.

	\medskip

	Dein Mitspieler muss keine Karte aus dem Vorrat nennen; aber mit der \emph{WAFFENKISTE} nimmst du eine Karte vom Vorrat auf deinen Ablagestapel.}
\end{tikzpicture}
\hspace{-0.6cm}
\begin{tikzpicture}
	\card
	\cardstrip
	\cardbanner{banner/white.png}
	\cardicon{icons/coin.png}
	\cardprice{5}
	\cardtitle{\scriptsize{Wunderheilerin}}
	\cardcontent{Diese Karte macht aus \emph{FLÜCHEN} während des gesamten Spiels und in allen Situationen Geldkarten, als ob im unteren Bereich der Karte \enquote{\emph{FLUCH - GELD}} stünde. Sie können in der Kaufphase für +\coin[1] gespielt werden. Sie werden entsorgt, wenn du einen \emph{MÜNZER} nimmst und sie werden in deinen Handkarten bei einer \emph{MAGNATIN} mitgezählt. Mit den \emph{HÖFLINGEN} (aus \emph{Intrige 2. Edition}) erhältst du zwei Wahlmöglichkeiten, wenn du einen \emph{FLUCH} aufdeckst, usw. 
	
	\medskip

	\emph{FLÜCHE} sind aber auch weiterhin \emph{FLÜCHE} und zählen -1 \victorypoint bei Spielende.}
\end{tikzpicture}
\hspace{-0.6cm}
\begin{tikzpicture}
	\card
	\cardstrip
	\cardbanner{banner/white.png}
	\cardtitle{\scriptsize{Empfohlene 10er Sätze\qquad}}
	\cardcontent{\emph{Anfänger:}\\
	Arbeiterdorf, Ausbau, Bank, Brautkrone, Buchhalterin, Denkmal, Gesindel, Kristallkugel, Magnatin, Wachturm

	\smallskip

	\emph{Freundliche Interaktion:}\\
	Arbeiterdorf, Bischof, Brautkrone, Gewölbe, Hausiererin, Hort, Kunstschmiede, Sammelsurium, Stadt, Waffenkiste

	\smallskip

	\emph{Das große Geld} (+ \textit{Basisspiel (2. Edition)}):\\
	Bank, Brautkrone, Großer Markt, Kristallkugel, Münzer, \textit{Geldverleiher}, \textit{Laboratorium}, \textit{Mine}, \textit{Töpferei}, \textit{Vorbotin}

	\smallskip

	\emph{Die Armee des Königs} (+ \textit{Basisspiel (2. Edition)}):\\
	Ausbau, Gesindel, Gewölbe, Königshof, Sammelsurium, \textit{Bürokrat}, \textit{Burggraben}, \textit{Dorf}, \textit{Händlerin}, \textit{Ratsversammlung}

	\smallskip

	\emph{Wege zum Sieg} (+ \textit{Die Intrige (2. Edition)}):\\
	Bischof, Denkmal, Hausiererin, Magnatin, Sammelsurium, \textit{Anbau}, \textit{Armenviertel}, \textit{Baron}, \textit{Handlanger}, \textit{Harem}

	\smallskip

	\emph{Die glücklichen Sieben} (+ \textit{Die Intrige (2. Edition)}):\\
	Ausbau, Bank, Brautkrone, Königshof, Kunstschmiede, \textit{Anbau}, \textit{Baron}, \textit{Bergwerk}, \textit{Patrouille}, \textit{Wunschbrunnen}}
\end{tikzpicture}
\hspace{-0.6cm}
\begin{tikzpicture}
	\card
	\cardstrip
	\cardbanner{banner/white.png}
	\cardtitle{\scriptsize{Empfohlene 10er Sätze\qquad}}
	\cardcontent{\emph{Explodierendes Königreich} (+ \textit{Seaside (2. Edition)}):\\
	Bischof, Großer Markt, Königshof, Stadt, Steinbruch, \textit{Ausguck}, \textit{Außenposten}, \textit{Fischerdorf}, \textit{Taktiker}, \textit{Werft}

	\smallskip

	\emph{Piratenbucht} (+ \textit{Seaside (2. Edition)}):\\
	Geldanalge, Hort, Magnatin, Münzer, Wunderheilerin, \textit{Affe}, \textit{Astrolabium}, \textit{Eingeborenendorf}, \textit{Korsarenschiff}, \textit{Schatzkammer}

	\smallskip

	\emph{Berufsausbildung} (+ \textit{Die Alchemisten}):\\
	Amboss, Arbeiterdorf, Bischof, Hausiererin, Münzer, Wunderheilerin, \textit{Leherling}, \textit{Universität}, \textit{Vertrauter}, \textit{Weinberg}

	\smallskip

	\emph{Umleitungen} (+ \textit{Reiche Ernte / Die Gilden}):\\
	Buchhalterin, Hort, Kristallkugel, Kunstschmiede, Magnatin, \textit{Bauerndorf}, \textit{Füllhorn}, \textit{Harlekin}, \textit{Nachbau}, \textit{Turnier}

	\smallskip

	\emph{Steinhauer} (+ \textit{Reiche Ernte / Die Gilden}):\\
	Ausbau, Großer Markt, Stadt, Steinbruch, Wunderheilerin, \textit{Bäcker}, \textit{Hellseherin}, \textit{Kaufmannsgilde}, \textit{Leuchtenmacher}, \textit{Metzger}}
\end{tikzpicture}
\hspace{-0.6cm}
\begin{tikzpicture}
	\card
	\cardstrip
	\cardbanner{banner/white.png}
	\cardtitle{\scriptsize{Empfohlene 10er Sätze\qquad}}
	\cardcontent{\emph{Müllabfuhr} (+ \textit{Dark Ages (mit Unterschlüfpen)}):\\
	Amboss, Kristallkugel, Magnatin, Stadt, Waffenkiste, \textit{Falschgeld}, \textit{Knappe}, \textit{Marktplatz}, \textit{Mundraub}, \textit{Raubzug}

	\smallskip

	\emph{Ganovenehre} (+ \textit{Dark Ages (mit Unterschlüpfen)}):\\
	Hort, Kunstschmiede, Sammelsurium, Steinbruch, Wachturm, \textit{Banditenlager}, \textit{Knappe}, \textit{Marodeur}, \textit{Prozession}, \textit{Schurke}

	\smallskip

	\emph{Der Letzte Wille} (+ \textit{Abenteuer}):\\
	Bischof, Denkmal, Gewölbe, Magnatin, Sammelsurium, \textit{Hafenstadt}, \textit{Königliche Münzen}, \textit{Kurier}, \textit{Relikt}, \textit{Verlies}, \textit{\underline{Erbschaft}}

	\smallskip

	\emph{Think Big} (+ \textit{Abenteuer}):\\
	Ausbau, Hausiererin, Hort, Königshof, Waffenkiste, \textit{Ferne Lande}, \textit{Gefolgsmann}, \textit{Geizhals}, \textit{Geschichtenerzähler}, \textit{Riese}, \textit{\underline{Ball}}, \textit{\underline{Überfahrt}}

	\smallskip

	\emph{Big Times} (+ \textit{Empires}):\\
	Bank, Brautkrone, Geldanalge, Großer Markt, Kunstschmiede, \textit{Gladiator/Reichtum}, \textit{Königlicher Schmied}, \textit{Patrizier/Handelsplatz}, \textit{Vermögen}, \textit{Villa}, \textit{\underline{Beherrschen}}, \textit{\underline{Obelisk}}

	\smallskip

	\emph{Geschmiedete Tore} (+ \textit{Empires}):\\
	Amboss, Hausiererin, Münzer, Sammelsurium, Waffenkiste, \textit{Feldlager/Diebesgut}, \textit{Gärtnerin}, \textit{Stadtviertel}, \textit{Wagenrennen}, \textit{Wilde Jagd}, \textit{\underline{Basilika}}, \textit{\underline{Palast}}}
\end{tikzpicture}
\hspace{-0.6cm}
\begin{tikzpicture}
	\card
	\cardstrip
	\cardbanner{banner/white.png}
	\cardtitle{\scriptsize{Empfohlene 10er Sätze\qquad}}
	\cardcontent{\emph{Schätze der Nacht} (+ \textit{Nocturne}):\\
	Brautkrone, Geldanalge, Kristallkugel, Waffenkiste, Wunderheilerin, \textit{Krypta}, \textit{Nachtwache}, \textit{Plünderer}, \textit{Vampirin}, \textit{Wächterin}

	\smallskip

	\emph{Ein Tag beim Pferderennen} (+ \textit{Nocturne}):\\
	Amboss, Bischof, Buchhalterin, Hausiererin, Wachturm, \textit{Druidin (Geschenk des Sumpfes, Geschenk des Flusses, Geschenk des Waldes)}, \textit{Folterknecht}, \textit{Friedhof}, \textit{Seliges Dorf}, \textit{Tragischer Held}

	\smallskip

	\emph{Traumtänzer} (+ \textit{Renaissance}):\\
	Arbeiterdorf, Denkmal, Gewölbe, Wachturm, Wunderheilerin, \textit{Alte Hexe}, \textit{Frachtschiff}, \textit{Gelehrter}, \textit{Priester}, \textit{Zepter}, \textit{\underline{Akademie}}

	\smallskip

	\emph{Geld regiert die Welt} (+ \textit{Renaissance}):\\
	Bank, Geldanalge, Gesindel, Großer Markt, Stadt, \textit{Forscherin}, \textit{Patron}, \textit{Schatzmeisterin}, \textit{Schwarzer Meister}, \textit{Versteck}, \textit{\underline{Kapitalismus}}, \textit{\underline{Zitadelle}}}
\end{tikzpicture}
\hspace{-0.6cm}
\begin{tikzpicture}
	\card
	\cardstrip
	\cardbanner{banner/white.png}
	\cardtitle{\scriptsize{Empfohlene 10er Sätze\qquad}}
	\cardcontent{\emph{Zeitlich limitiertes Angebot} (+ \textit{Menagerie}):\\
	Amboss, Arbeiterdorf, Großer Markt, Hausiererin, Münzer, \textit{Fischer}, \textit{Nachschub}, \textit{Schlachtross}, \textit{Vertreibung}, \textit{Wanderin}, \textit{\underline{Verzweiflung}}, \textit{\underline{Weg des Frosches}}

	\smallskip

	\emph{Otter-Chaos} (+ \textit{Menagerie}):\\
	Buchhalterin, Denkmal, Stadt, Steinbruch, Waffenkiste, \textit{Drahtzieher}, \textit{Jagdhütte}, \textit{Kamelzug}, \textit{Koppel}, \textit{Viehmarkt}, \textit{\underline{Reiche Ernte}}, \textit{\underline{Weg des Otters}}
	
	\smallskip

	\emph{Erfinderboom} (+ \textit{Verbündete}):\\
	Amboss, Ausbau, Gesindel, Königshof, Steinbruch, \textit{Augurinnen}, \textit{Hauptstadt}, \textit{Importeurin}, \textit{Schreinerin}, \textit{Tand}, \textit{\underline{Erfinderfamilie}}

	\smallskip

	\emph{Schmeicheln macht reich} (+ \textit{Verbündete}):\\
	Bank, Buchhalterin, Geldanalge, Gewölbe, Stadt, \textit{Irrfahrten}, \textit{Marquis}, \textit{Mittelsmann}, \textit{Ortschaft}, \textit{Schmeichler}, \textit{\underline{Bankiersverbund}}}
\end{tikzpicture}
\hspace{-0.6cm}
\begin{tikzpicture}
	\card
	\cardstrip
	\cardbanner{banner/white.png}
	\cardtitle{Platzhalter\quad}
\end{tikzpicture}
\hspace{0.6cm}
