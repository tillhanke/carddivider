% Basic settings for this card set
\renewcommand{\cardcolor}{hinterlands}
\renewcommand{\cardextension}{Erweiterung V}
\renewcommand{\cardextensiontitle}{Hinterland}
\renewcommand{\seticon}{hinterlands.png}

\clearpage
\newpage
\section{\cardextension \ - \cardextensiontitle \ 2. Edition (Rio Grande Games 2022)}

\begin{tikzpicture}
	\card
	\cardstrip
	\cardbanner{banner/goldblue.png}
	\cardicon{icons/coin.png}
	\cardprice{2}
	\cardtitle{Katzengold}
	\cardcontent{Diese Karte ist sowohl eine Geldkarte ais auch eine Reaktionskarte. Wenn du sie spielst, ist \emph{KATZENGOLD} beim ersten Mal in einem Zug \coin[1] wert, bei jedem weiteren Mal in diesem Zug /coin[4]. Spielst du zum Beispiel mit einem \emph{FALSCHGELD} ein \emph{KATZENGOLD} zweimal, bekommst du \coin[1] und \coin[4], also insgesamt \coin[5], obwohl es beide Male das erste ausgespielte Katzengold ist.

	\smallskip

	Wenn du ein \emph{KATZENGOLD} auf deiner Hand hast und ein:e Mitspieler:in eine \emph{PROVINZ} nimmt, darfst du mit dem \emph{KATZENGOLD} reagieren: Du entsorgst das \emph{KATZENGOLD} und nimmst aus dem Vorrat ein \emph{GOLD}, das du verdeckt auf deinen Nachziehstapel legst.}
\end{tikzpicture}
\hspace{-0.6cm}
\begin{tikzpicture}
	\card
	\cardstrip
	\cardbanner{banner/white.png}
	\cardicon{icons/coin.png}
	\cardprice{2}
	\cardtitle{\footnotesize{Wegkreuzung}}
	\cardcontent{Decke deine Handkarten auf. Für jede Punktekarte (auch ggf. kombinierte) erhältst du \emph{+1 Karte}. Wenn du die \emph{WEGKREUZUNG} zum ersten Mal in diesem Zug spielst, erhältst du \emph{+3 Aktionen}. Für jede weitere ausgespielte \emph{WEGKREUZUNG} in diesem Zug erhältst du keine weiteren Aktionen.}
\end{tikzpicture}
\hspace{-0.6cm}
\begin{tikzpicture}
	\card
	\cardstrip
	\cardbanner{banner/white.png}
	\cardicon{icons/coin.png}
	\cardprice{3}
	\cardtitle{Aufbau}
	\cardcontent{Entsorge eine beliebige deiner Handkarten. Wenn du eine Karte entsorgt hast, nimm zwei Karten vom Vorrat: 1 Karte, die genau \coin[1] mehr kostet als die entsorgte Karte, und 1 Karte, die genau \coin[1] weniger kostet als die entsorgte Karte. Kannst du eine bzw. beide Karten nicht nehmen, weil keine entsprechende Karte im Vorrat ist, erhältst du nur die verfügbare Karte bzw. nichts. Lege die genommenen Karten in beliebiger Reihenfolge auf deinen Nachziehstapel.}
\end{tikzpicture}
\hspace{-0.6cm}
\begin{tikzpicture}
	\card
	\cardstrip
	\cardbanner{banner/white.png}
	\cardicon{icons/coin.png}
	\cardprice{3}
	\cardtitle{Komplott}
	\cardcontent{Wenn du diese Karte spielst, ziehst du eine Karte nach, erhältst \emph{+1 Aktion} und bereitest einen Effekt vor, der später in diesem Zug passiert: einmal, wenn du eine Aktionskarte aus dem Spiel ablegst, darfst du sie auf deinen Nachziehstapel legen. Die Aktionskarte kann das \emph{KOMPLOTT} selbst sein oder jede andere Aktionskarte, die du im Spiel hast, egal ob du sie vor oder nach dem \emph{KOMPLOTT} ausgespielt hast. Das \emph{KOMPLOTT} ist kumulativ; wenn du zwei \emph{KOMPLOTTE} spielst oder ein \emph{KOMPLOTT} mit einem \emph{THRONSAAL}, darfst du bis zu zwei abgelegte Aktionskarten auf deinen Nachziehstapel legen usw.}
\end{tikzpicture}
\hspace{-0.6cm}
\begin{tikzpicture}
	\card
	\cardstrip
	\cardbanner{banner/white.png}
	\cardicon{icons/coin.png}
	\cardprice{3}
	\cardtitle{Oase}
	\cardcontent{Du legst eine Handkarte ab - das darf auch die gerade gezogene Karte sein. Auch wenn du keine Karte nachziehen kannst, legst du trotzdem eine Handkarte ab.}
\end{tikzpicture}
\hspace{-0.6cm}
\begin{tikzpicture}
	\card
	\cardstrip
	\cardbanner{banner/greenblue.png}
	\cardicon{icons/coin.png}
	\cardprice{3}
	\cardtitle{Tunnel}
	\cardcontent{Diese Karte ist sowohl eine Punktekarte als auch eine Reaktionskarte. Wenn du durch eine Anweisung auf einer Karte (egal ob in deinem Zug oder dem Zug eines Mitspielers/einer Mitspielerin) gezwungen wirst, eine Karte außerhalb der normalen Aufräumphase abzulegen und du dich für einen \emph{TUNNEL} entscheidest, darfst du diesen aufdecken, bevor du ihn ablegst. Wenn du das tust, nimm ein \emph{GOLD} vom Vorrat und lege dieses auf deinen Ablagestapel. Siehe den Abschnitt: \emph{Wenn du ... ablegst} der neuen Anweisungen.}
\end{tikzpicture}
\hspace{-0.6cm}
\begin{tikzpicture}
	\card
	\cardstrip
	\cardbanner{banner/blue.png}
	\cardicon{icons/coin.png}
	\cardprice{3}
	\cardtitle{Wachhund}
	\cardcontent{Wenn du diese Karte spielst, ziehst du 2 Karten und zählst dann die Karten auf deiner Hand. Sind das 5 oder weniger, ziehst du zwei weitere Karten. Wenn ein:e Mitspieler:in eine Angriffskarte spielt, darfst du diese Karte spielen, bevor der Angriff ausgeführt wird. Dann wird der Angriff ausgeführt (außer du verhinderst dies auf andere Weise). Wenn du diese Karte als Reaktion nutzt, spielst du sie, das heißt sie kommt ins Spiel und du folgst ihren Anweisungen. Wenn du sie während des Zuges eines Mitspielers/einer Mitspielerin spielst, legst du sie während jener Aufräumphase ab.}
\end{tikzpicture}
\hspace{-0.6cm}
\begin{tikzpicture}
	\card
	\cardstrip
	\cardbanner{banner/blue.png}
	\cardicon{icons/coin.png}
	\cardprice{4}
	\cardtitle{\tiny{Fahrende Händlerin}}
	\cardcontent{\tiny{Wenn du diese Karte spielst, musst du eine deiner Handkarten entsorgen. Pro \coin[1], das die entsorgte Karte kostet, nimmst du ein \emph{SILBER} vom Vorrat. So nimmst du für eine Karte, die (in diesem Moment) \coin[3] kostet, drei \emph{SILBER} vom Vorrat. Kannst du keine Karte entsorgen, erhältst du auch kein \emph{SILBER}. Kostet eine Karte z.B. \coin[2] und \potion (aus \emph{Alchemisten}), nimmst du 2 \emph{SILBER}. \potion oder \hex (aus anderen \emph{DOMINION}-Erweiterungen) haben keinen Einfluss.

	\smallskip

	Wenn du die \emph{FAHRENDE HÄNDLERIN} als Reaktion einsetzt, weil du (egal ob in deinem Zug oder dem Zug eines Mitspielers) eine Karte genommen hast, deckst du die \emph{FAHRENDE HÄNDLERIN} aus deiner Hand auf. Wenn du das tust, legst du die genommene Karte auf ihren Stapel (auch dann, wenn du sie von woanders her genommen hast) und erhältst ein \emph{SILBER} vom Vorrat auf deinen Ablagestapel. Im seltenen Fall, dass die genommene Karte nicht auf ihren Stapel gelegt werden kann (weil sie keinen Stapel hat oder sie inzwischen wegbewegt wurde) oder der Silberstapel im Vorrat leer ist, wird keine der beiden Teile (zurücklegen, \emph{SILBER} erhalten) durchgeführt. Wenn du eine Karte kaufst, du stattdessen aber ein \emph{SILBER} nimmst, musst du die Kosten der eigentlich gekauften Karte bezahlen. Für Effekte, die sich auf das Nehmen einer Karte beziehen, gilt die Karte, die du genommen hast, immer noch als \enquote{genommen} (und nicht das \emph{SILBER}). Zum Beispiel könntest du ein \emph{GRENZDORF} nehmen, es mit einer \emph{FAHRENDEN HÄNDLERIN} in ein \emph{SILBER} eintauschen, dann mit der Fähigkeit des \emph{GRENZDORFES} ein \emph{HERZOGTUM} nehmen, welches du mit der \emph{FAHRENDEN HÄNDLERIN} ebenfalls in ein \emph{SILBER} eintauschst.}}
\end{tikzpicture}
\hspace{-0.6cm}
\begin{tikzpicture}
	\card
	\cardstrip
	\cardbanner{banner/white.png}
	\cardicon{icons/coin.png}
	\cardprice{4}
	\cardtitle{\scriptsize{Gewürzhändler}}
	\cardcontent{Wenn du diese Karte spielst, darfst du eine beliebige \emph{Geldkarte} aus deiner Hand entsorgen. Wenn du eine Karte entsorgst, darfst du entscheiden zwischen: \emph{+2 Karten} und \emph{+1 Aktion} oder +\coin[2] und \emph{+1 Kauf}. Wenn du keine Karte entsorgen kannst oder willst, führst du keine der angegebenen Anweisungen aus.}
\end{tikzpicture}
\hspace{-0.6cm}
\begin{tikzpicture}
	\card
	\cardstrip
	\cardbanner{banner/white.png}
	\cardicon{icons/coin.png}
	\cardprice{4}
	\cardtitle{\scriptsize{Lebenskünstler}}
	\cardcontent{Wenn du diese Karte spielst, führst du die 4 Anweisungen der Reihe nach aus:
	\begin{enumerate}
		\item Nimm ein \emph{SILBER} vom Vorrat und lege es auf deinen Ablagestapel.
		\item Sieh dir die oberste Karte deines Nachziehstapels an. Du entscheidest, ob du diese Karte ablegst oder auf den Nachziehstapel zurücklegst.
		\item Ziehe solange Karten nach, bis du 5 Karten auf der Hand hast. Hast du bereits 5 oder mehr Karten auf der Hand, ziehst du keine Karten nach.
		\item Du \emph{darfst} eine Karte, die kiene Geldkarte (auch keine kombinierte) ist, aus deiner Hand entsorgen.
	\end{enumerate}}
\end{tikzpicture}
\hspace{-0.6cm}
\begin{tikzpicture}
	\card
	\cardstrip
	\cardbanner{banner/white.png}
	\cardicon{icons/coin.png}
	\cardprice{4}
	\cardtitle{Nomaden}
	\cardcontent{Wenn du diese Karte während des Zuges eines Mitspielers/einer Mitspielerin nimmst oder entsorgst, erhältst du trotzdem die +\coin[2], du kannst dies aber normalerweise nicht nutzen.}
\end{tikzpicture}
\hspace{-0.6cm}
\begin{tikzpicture}
	\card
	\cardstrip
	\cardbanner{banner/blue.png}
	\cardicon{icons/coin.png}
	\cardprice{4}
	\cardtitle{Pfad}
	\cardcontent{Wenn du diese Karte spielst, erhältst du einfach nur \emph{+1 Karte} und \emph{+1 Aktion}.	Wenn du diese Karte außerhalb der Aufräumphase nimmst, entsorgst oder ablegst (vgl. den Abschnitt: \emph{Wenn du ... ablegst} der neuen Anweisungen), darfst du sie spielen. Die Karte zu spielen bedeutet sie ins Spiel zu bringen und ihre Anweisungen auszuführen. Wenn du den \emph{PFAD} im Zug eines Mitspielers / einer Mitspielerin spielst, kannst du die \emph{+1 Aktion} nicht nutzen und legst den \emph{PFAD} in jener Aufräumphase ab. Wenn du einen \emph{PFAD} entsorgst, kommt er wieder ins Spiel und wird in dieser Aufräumphase abgelegt. Dies zählt trotzdem als \enquote{entsorgen}, aber nicht als \enquote{nehmen}. Wenn du einen \emph{PFAD} mithilfe des \emph{UMBAUS} (aus dem \emph{Basisspiel}) aus der Hand entsorgst und dir dafür ein \emph{GOLD} nimmst, darfst du den entsorgten \emph{PFAD} trotzdem spielen und so weiter.}
\end{tikzpicture}
\hspace{-0.6cm}
\begin{tikzpicture}
	\card
	\cardstrip
	\cardbanner{banner/blue.png}
	\cardicon{icons/coin.png}
	\cardprice{4}
	\cardtitle{Weberin}
	\cardcontent{Entweder nimmst du 2 \emph{SILBER} oder eine Karte, die bis zu \coin[4] kostet (das kann auch ein \emph{SILBER} sein). Wenn du diese Karte außerhalb der Aufräumphase ablegst (vgl. den Abschnitt: \emph{Wenn du ... ablegst} der neuen Anweisungen), darfst du sie spielen. Die Karte zu spielen bedeutet sie ins Spiel zu bringen and ihre Anweisungen zu befolgen. Wenn du die \emph{WEBERIN} im Zug eines Mitspielers/einer Mitspielerin spielst, legst du sie in jener Aufräumphase ab.}
\end{tikzpicture}
\hspace{-0.6cm}
\begin{tikzpicture}
	\card
	\cardstrip
	\cardbanner{banner/white.png}
	\cardicon{icons/coin.png}
	\cardprice{5}
	\cardtitle{Berserker}
	\cardcontent{Wenn du einen \emph{BERSERKER} nimmst, während du eine Aktionskarte im Spiel hast, spielst du den genommenen \emph{BERSERKER}, das heißt die genommene Karte kommt ins Spiel. Damit (oder wenn du einen \emph{BERSERKER} aus der Hand spielst) wird die Anweisung über der Trennlinie aktiviert, d. h. du nimmst eine billigere Karte und dann legen alle Mitspielerinnen ihre Handkarten bis auf 3 ab. Karten wie der \emph{BURGGRABEN} können von den Mitspielerinnen genutzt werden. Nimmst du einen \emph{BERSERKER} und hast keine Aktionskarte im Spiel, spielst du den \emph{BERSERKER} nicht.}
\end{tikzpicture}
\hspace{-0.6cm}
\begin{tikzpicture}
	\card
	\cardstrip
	\cardbanner{banner/white.png}
	\cardicon{icons/coin.png}
	\cardprice{5}
	\cardtitle{Feilscher}
	\cardcontent{Für den Rest des Zuges musst du, immer wenn du eine von dir gekaufte Karte nimmst, eine Karte nehmen, die weniger kostet als die genommene Karte und die keine Punktekarte ist. Zum Beispiel kannst du eine \emph{PROVINZ} kaufen, die \emph{PROVINZ} nehmen und dafür dann mit dem \emph{FEILSCHER} ein \emph{GOLD} nehmen. Das Nehmen ist Pflicht. Die genommene Karte kommt aus den Vorrat auf deinen Ablagestapel. Der \emph{FEILSCHER} gibt dir nur eine weitere Karte, wenn du eine von dir gekaufte Karte nimmst, nicht wenn du auf eine andere Weise eine Karte nimmst (etwa mit dem \emph{FEILSCHER} selbst). Gibe es keine billigere Karte im Vorrat (zum Beispiel wenn du \emph{KUPFER} kaufst), dann nimmst du keine Karte. Die Fähigkeit ist kumulativ; wenn du einen \emph{FEILSCHER} spielst und dann mit einem \emph{THRONSAAL} einen weiteren \emph{FEILSCHER}, dann nimmst du für jede gekaufte Karte drei Karten.}
\end{tikzpicture}
\hspace{-0.6cm}
\begin{tikzpicture}
	\card
	\cardstrip
	\cardbanner{banner/white.png}
	\cardicon{icons/coin.png}
	\cardprice{5}
	\cardtitle{Fernstrasse}
	\cardcontent{Wenn diese Karte gespielt wurde, werden die Kosten aller Karten (d.h.
	Handkarten, Karten aus den Nachzieh- und Ablagestapeln aller Spieler, Karten im Vorrat, evtl. beiseite gelegte Karten usw.) um \coin[1] reduziert, niemals jedoch auf weniger als \coin[0].
	Der Effekt ist kumulativ, d.h. sind z.B. 2 \emph{FERNSTRASSEN} im Spiel, reduzieren sich die Kosten der Karten um \coin[2]. Wenn du mit einem \emph{THRONSAAL} (aus dem \emph{Basisspiel}) die gleiche \emph{FERNSTRASSE} zweimal spielst, reduzieren sich die Kosten aller Karten auch um insgesamt \coin[2].}
\end{tikzpicture}
\hspace{-0.6cm}
\begin{tikzpicture}
	\card
	\cardstrip
	\cardbanner{banner/white.png}
	\cardicon{icons/coin.png}
	\cardprice{5}
	\cardtitle{Gasthaus}
	\cardcontent{Wenn du diese Karte kaufst oder auf andere Art und Weise nimmst, darfst du deinen Ablagestapel anschauen. Decke dann beliebig viele Aktionskarten aus deinem Ablagestapel auf (auch diese) und mische alle aufgedeckten Karten in deinen Nachziehstapel. Sollte dein Nachziehstapel gerade leer sein, wenn du den Ablagestapel durchsiehst, werden die ausgewählten Aktionskarten zu den einzigen Karten deines Nachziehstapels.}
\end{tikzpicture}
\hspace{-0.6cm}
\begin{tikzpicture}
	\card
	\cardstrip
	\cardbanner{banner/white.png}
	\cardicon{icons/coin.png}
	\cardprice{5}
	\cardtitle{Hexenhütte}
	\cardcontent{Du musst beide abgelegte Karten aufdecken, auch wenn sie nicht beide Aktionskarten sind. Wenn sie beide Aktionskarten sind - auch wenn sie zusätzlich noch andere Typen enthalten - nehmen alle Mitspieler:innen einen \emph{FLUCH}.}
\end{tikzpicture}
\hspace{-0.6cm}
\begin{tikzpicture}
	\card
	\cardstrip
	\cardbanner{banner/gold.png}
	\cardicon{icons/coin.png}
	\cardprice{5}
	\cardtitle{Hexenkessel}
	\cardcontent{Der Effekt, dass jeder Mitspieler einen \emph{FLUCH} nehmen muss, liegt in der Zukunft, muss aber nicht eintreten. Wenn du in einem Zug, \emph{bevor} du den \emph{HEXENKESSEL} spielst, schon drei Aktionskarten genommen hast, nehmen die Mitspieler:innen keinen \emph{FLUCH}. Es ist egal, wie viele Nicht-Aktionskarten du nimmst, oder wie viele Aktionskarten du nach dem Spielen des \emph{HEXENKESSELS} noch nimmst, einzig relevant ist die Gesamtanzahl deiner genommenen Aktionskarten im aktuellen Zug. Geschieht das Nehmen deiner 3. Aktionskarte dieses Zuges, nachdem du einen \emph{HEXENKESSEL} gespielt hast, nehmen alle Mitspieler:innen einen \emph{FLUCH}. Dies ist kumulativ, d. h. wenn du mehrere \emph{HEXENKESSEL} spielst, nehmen alle Mitspielerinnen pro \emph{HEXENKESSEL} einen \emph{FLUCH}. Der \emph{HEXENKESSEL} ist eine Geldkarte, d.h. du spielst ihn regulär in deiner Kaufphase. Da er aber auch eine Angriffskarte ist, dürfen Reaktionskarten wie der \emph{WACHHUND} und der \emph{BURGGRABEN} genutzt werden.}
\end{tikzpicture}
\hspace{-0.6cm}
\begin{tikzpicture}
	\card
	\cardstrip
	\cardbanner{banner/white.png}
	\cardicon{icons/coin.png}
	\cardprice{5}
	\cardtitle{Kartograph}
	\cardcontent{Schau dir die obersten 4 Karten deines Nachziehstapels an. Lege beliebig viele davon ab. Die restlichen legst du in beliebiger Reihenfolge zurück auf deinen Nachziehstapel.}
\end{tikzpicture}
\hspace{-0.6cm}
\begin{tikzpicture}
	\card
	\cardstrip
	\cardbanner{banner/white.png}
	\cardicon{icons/coin.png}
	\cardprice{5}
	\cardtitle{Markgraf}
	\cardcontent{Jeder Mitspieler:in, beginnend mit deinem linken Nachbarn/deiner linken Nachbarin, ziehen eine Karte von ihrem Nachziehstapel und legen dann so lange Karten ab, bis jeder/jede maximal 3 Karten auf seiner/ihrer Hand hat. Hast du, auch nachdem du 1 Karte gezogen hast, 3 Karten oder weniger auf deiner Hand, musst du keine Karte ablegen.}
\end{tikzpicture}
\hspace{-0.6cm}
\begin{tikzpicture}
	\card
	\cardstrip
	\cardbanner{banner/white.png}
	\cardicon{icons/coin.png}
	\cardprice{5}
	\cardtitle{Radmacherin}
	\cardcontent{Du darfst jeden beliebigen Kartentyp ablegen, aber die genommene Karte muss eine Aktionskarte sein. Wenn du eine Aktionskarte ablegst, kannst du eine gleiche Aktionskarte nehmen.}
\end{tikzpicture}
\hspace{-0.6cm}
\begin{tikzpicture}
	\card
	\cardstrip
	\cardbanner{banner/white.png}
	\cardicon{icons/coin.png}
	\cardprice{5}
	\cardtitle{Souk}
	\cardcontent{Wenn du zum Beispiel einen \emph{SOUK} spielst, während du noch 3 andere Karten auf deiner Hand hast, erhältst du \emph{+1 Kauf} und \coin[7] und verlierst dann \coin[3], für insgesamt \emph{+1 Kauf} und +\coin[4]. Du kannst nicht weniger als \coin[0] haben, aber wenn du vorher schon Geldwerte bekommen hast (z.B. durch andere Aktionskarten), kannst du diese wieder verlieren. Wenn du einen \emph{SOUK} nimmst, kannst du bis zu 2 deiner Handkarten entsorgen, du musst aber nicht.}
\end{tikzpicture}
\hspace{-0.6cm}
\begin{tikzpicture}
	\card
	\cardstrip
	\cardbanner{banner/white.png}
	\cardicon{icons/coin.png}
	\cardprice{5}
	\cardtitle{Stallungen}
	\cardcontent{Du darfst eine beliebige Geldkarte aus deiner Hand ablegen. Wenn du das tust, ziehst du 3 Karten und erhältst \emph{+1 Aktion}. Wenn du das nicht tust, darfst du keine Karten ziehen und erhältst keine zusätzliche Aktion.}
\end{tikzpicture}
\hspace{-0.6cm}
\begin{tikzpicture}
	\card
	\cardstrip
	\cardbanner{banner/green.png}
	\cardicon{icons/coin.png}
	\cardprice{6}
	\cardtitle{\tiny{Fruchtbares Land}}
	\cardcontent{Bei Spielende erhältst du 2\victorypoint für jedes \emph{FRUCHTBARE LAND} in deinem Kartensatz. Wenn du diese Karte nimmst, \emph{musst} du eine Karte aus deiner Hand entsorgen. Nimm eine Karte vom Vorrat, die genau \coin[2] mehr kostet als die entsorgte Karte. Kannst du keine Karte entsorgen oder befindet sich keine Karte im Vorrat, die genau \coin[2] mehr kostet, nimmst du keine zusätzliche Karte. Du kannst dir damit kein \emph{FRUCHTBARES LAND} nehmen.}
\end{tikzpicture}
\hspace{-0.6cm}
\begin{tikzpicture}
	\card
	\cardstrip
	\cardbanner{banner/white.png}
	\cardicon{icons/coin.png}
	\cardprice{6}
	\cardtitle{Grenzdorf}
	\cardcontent{Wenn du diese Karte spielst, ziehst du 1 Karte und erhältst \emph{+2 Aktionen}.
	Wenn du diese Karte kaufst oder auf andere Art und Weise nimmst, nimmst du eine weitere Karte vom Vorrat, die weniger kostet als das \emph{GRENZDORF}.}
\end{tikzpicture}
\hspace{-0.6cm}
\begin{tikzpicture}
	\card
	\cardstrip
	\cardbanner{banner/white.png}
	\cardtitle{\scriptsize{Neue Regeln (1/2)}\qquad}
	\cardcontent{
		\emph{Es gelten die Basisspielregeln mit folgenden Ergänzungen:}
		
		\medskip
		
		\emph{Geldkarten mit zusätzlichen Anweisungen:} Die Geldkarten \emph{HEXENKESSEL} und \emph{KATZENGOLD} werden, wie alle Geldkarten, in der entsprechenden Spielphase eingesetzt. Darüber hinaus beinhalten diese Karten zusätzliche Anweisungen, die beim Ausspielen beachtet werden müssen (siehe hierzu die entsprechenden Kartenbeschreibungen).
		
		\medskip

		\emph{Kombinierte Königreichkarten:} Die Karten \emph{FAHRENDE HÄNDLERIN}, \emph{KATZENGOLD}, \emph{PFAD}, \emph{TUNNEL}, \emph{WACHHUND} und \emph{WEBERIN} sind kombinierte Karten und haben die Funktion beider angegebener Kartentypen, d.h. sie sind jeweils sowohl eine Reaktionskarte, die während des Spiels als Reaktion genutzt werden kann, als auch eine klassische Punkte-, Geld- bzw. Aktionskarte. Wird die Karte als Reaktionskarte eingesetzt, gilt die Anweisung unterhalb der Trennlinie. Anweisungen auf anderen Aktionskarten, die sich auf Reaktions-, Geld- der Punktekarte beziehen, betreffen auch die jeweiligen kombinierten Karten.
	}
\end{tikzpicture}
\hspace{-0.6cm}
\begin{tikzpicture}
	\card
	\cardstrip
	\cardbanner{banner/white.png}
	\cardtitle{\scriptsize{Neue Regeln (2/2)}\qquad}
	\cardcontent{
		\emph{In der Kaufphase:} Alle Geldkarten, die in einem Zug eingesetzt werden sollen, müssen einzeln nacheinander und \emph{vor dem ersten} Kauf gespielt werden. Der Spieler darf in der Regel \emph{keine weiteren} Geldkarten spielen, nachdem ein Kauf getätigt wurde. Wenn er eine Geldkarte mit zusätzlichen Anweisungen spielt, muss er die Anweisungen zunächst (soweit möglich) ausführen, bevor er eine weiter Geldkarte spielt.
		
		\medskip
		
		\emph{Mehrere Dinge passieren gleichzeitig:} Sollten mehrere Anweisungen gleichzeitig ausgeführt werden können, entscheidet der Spieler/die Spielerin selbst, in welcher Reihenfolge sie/er sie ausführt. Sind mehrere Spieler:innen von einer Anweisung betroffen, wird diese in Spielreihenfolge, beginnend bei der Spielerin/dem Spieler, die/der gerade am Zug ist, ausgeführt.
	}
\end{tikzpicture}
\hspace{-0.6cm}
\begin{tikzpicture}
	\card
	\cardstrip
	\cardbanner{banner/white.png}
	\cardtitle{\footnotesize{Anweisungen (1/3)}\qquad}
	\cardcontent{
		\emph{Wenn du ... nimmst:} Etliche Anweisungen in \emph{Hinterland} lösen etwas aus, \enquote{wenn du diese/eine Karte nimmst}. Du nimmst eine Karte entweder direkt wie z.B. durch die \emph{WERKSTATT}, oder indem du sie kaufst. \enquote{Wenn du diese Karte nimmst}-Fähigkeiten führst du sofort aus, nachdem du die Karte genommen hast. Die Karte liegt zu dem Zeitpunkt schon auf deinem Ablagestapel (oder wohin du sie nehmen solltest), wenn du diese Fähigkeit ausführst. Wenn du eine Karte spielst, wirkt ihre
		\enquote{Wenn du diese Karte nimmst}-Fähigkeit nicht mehr Beispiele: 
		\begin{itemize}
			\item Du kaufst ein \emph{GRENZDORF} und nimmst es. Dann führst du seine Fähigkeit aus, du nimmst also eine billigere Karte und entscheidest dich für einen \emph{SOUK}. Du führst die \enquote{Wenn du diese Karte nimmst}-Fähigkeit vom \emph{SOUK} aus und du darfst bis zu 2 deiner Handkarten entsorgen.
			\item Du spielst einen \emph{FEILSCHER}, kaufst ein \emph{GRENZDORF} und nimmst es. Du führst die Fähigkeit des \emph{GRENZDORFS} aus und du nimmst ein \emph{HERZOGTUM}. Du führst ebenfalls die Fähigkeit des \emph{FEILSCHERS} aus und nimmst noch einen \emph{FEILSCHER} für das Nehmen des \emph{GRENZDORFS}. Der \emph{FEILSCHER} gibt dir keine weitere Karte für das Nehmen des \emph{HERZOGTUMS} oder des \emph{FEILSCHERS}, da er nur für Karten wirkt, die du gekauft hast.
			\item Du spielst ein \emph{GRENZDORF} und erhältst \emph{+1 Karte} und \emph{+2 Aktionen}. Du nimmst keine Karte, denn das geschicht nur, wenn du ein \emph{GRENZDORF} nimmst, nicht wenn du es spielst.
		\end{itemize}
	}
\end{tikzpicture}
\hspace{-0.6cm}
\begin{tikzpicture}
	\card
	\cardstrip
	\cardbanner{banner/white.png}
	\cardtitle{\footnotesize{Anweisungen (2/3)}\qquad}
	\cardcontent{
		\emph{Wenn du ... ablegst:} In Hinterland gibt es drei Karten, die eine Fähigkeit haben, wenn du sie ablegst: \emph{PFAD}, \emph{TUNNEL} und \emph{WEBERIN}.

		Du darfst dich nicht einfach nur entscheiden, diese Karten abzulegen. Du musst von etwas veranlasst oder gezwungen worden sein, sie abzulegen, um die Fähigkeiten auszuführen. Das Schlüsselwort, auf das du bei Anweisungen anderer Karten hierfür achten solltest, ist tatsächlich \enquote{ablegen}.

		Diese Fähigkeiten können in deinem eigenen Zug ausgeführt werden (wie z. B. durch die \emph{OASE}) oder im Zug eines Mitspielers/einer Mitspielerin (wie z. B. durch den \emph{MARKGRAFEN}). Sie werden nicht in der Aufräumphase ausgeführt, in der du normalerweise alle gespielten und nicht gespielten Karten ablegst.

		Diese Fähigkeiten werden unabhängig davon ausgeführt, von wo die Karte abgelegt wird, z.B. aus deiner Hand (wie durch die \emph{OASE}) oder von deinem Nachziehstapel (wie durch den \emph{KARTOGRAPHEN}) oder aus dem Exil (einem Tableau aus \emph{DOMINION Menagerie}).

		Wenn die Karte nicht unbedingt aufgedeckt werden würde (z.B. wenn du mit dem \emph{KARTOGRAPHEN} mehrere Karten ablegst), musst du sie trotzdem aufdecken, um die Fähigkeit auszuführen.

		Diese Fähigkeiten sind optional, auch wenn die Karte schon aus einem anderen Grund aufgedeckt wurde.
	}
\end{tikzpicture}

\hspace{-0.6cm}
\begin{tikzpicture}
	\card
	\cardstrip
	\cardbanner{banner/white.png}
	\cardtitle{\footnotesize{Anweisungen (3/3)}\qquad}
	\cardcontent{
		Diese Fähigkeiten werden nicht ausgeführt, wenn Karten auf deinen Ablagestapel gelegt werden, ohne abgelegt zu werden, wie z.B. wenn du eine Karte nimmst oder wenn dein Nachziehstapel auf deinen Adlagestapel gelegt wird (wie z.B. mit dem
		\emph{KURIER} aus \emph{DOMINION Abenteuer}).

		Diese Fähigkeiten werden nur einmal pro Ablegen der Karte ausgeführt. Du kannst nicht einen \emph{TUNNEL} ablegen und ihn zweimal aufdecken, um zwei \emph{GOLD} zu erhalten.

		Wenn du mehrere Karten auf einmal ablegst, (wie z.B. beim \emph{MARKGRAFEN}), werden sie alle gleichzeitig abgelegt, und dann werden ihre Fähigkeiten einzeln nacheinander ausgeführt. Das heißt z. B., wenn du durch einem \emph{MARKGRAFEN} zwei \emph{PFADE} ablegst und dich der erste \emph{PFAD} mischen lässt, wirst du den zweiten \emph{PFAD} nicht mehr spielen können (da er dann irgendwo in deinem Nachziehstapel verloren ist).

		\medskip

		\emph{Im Spiel:} Geld- und Aktionskarten, die ein Spieler offen in seinem Spielbereich vor sich liegen hat, befinden sich \enquote{im Spiel}, bis sie woanders hingelegt (i.d.R. in der Aufräumphase abgelegt) werden. Nicht \enquote{im Spiel} befinden sich beiseitegelegte und entsorgte Karten, sowie alle Handkarten, Karten im Vorrat und in den Nachzieh- und Ablagestapeln. Auch als Reaktion aufgedeckte Reaktionskarten befinden sich nicht \enquote{im Spiel}.
	}
\end{tikzpicture}
\hspace{-0.6cm}
\begin{tikzpicture}
	\card
	\cardstrip
	\cardbanner{banner/white.png}
	\cardtitle{\scriptsize{Empfohlene 10er Sätze\qquad}}
	\cardcontent{\emph{Einführung:} \\ 
	Aufbau, Gewürzhändler, Kartograph, Lebenskünstler, Markgraf, Nomaden, Oase, Stallungen, Weberin, Wegkreuzung 

	\smallskip 
	
	\emph{Gelegenheiten:} \\ 
	Fahrender Händlerin, Feilscher, Fernstraße, Grenzdorf, Hexenkessel, Katzengold, Komplott, Pfad, Radmacherin, Souk 

	\smallskip

	\emph{Glückliche Pfade} (+ \textit{Basisspiel (2. Edition)}):\\ 
	Berserker, Fernstraße, Nomaden, Oase, Pfad, \textit{Bibliothek}, \textit{Geldverleiher}, \textit{Keller}, \textit{Thronsaal}, \textit{Werkstatt}

	\smallskip 
	
	\emph{Abenteuerfahrt}  (+ \textit{Basisspiel (2. Edition)}):\\ 
	Hexenhütte, Katzengold, Souk, Wachhund, Wegkreuzung, \textit{Jahrmarkt}, \textit{Laboratorium}, \textit{Torwächterin}, \textit{Umbau}, \textit{Vasall}

	\smallskip

	\emph{Geld für nichts} (+ \textit{Intrige (2. Edition)}):\\ 
	Kartograph, Lebenskünstler, Radmacherin, Tunnel, Weberin, \textit{Armenviertel}, \textit{Austausch}, \textit{Handlanger}, \textit{Kerkermeister}, \textit{Patrouille}

	\smallskip 
	
	\emph{Der Ball des Herzogs}  (+ \textit{Intrige (2. Edition)}):\\ 
	Gasthaus, Komplott, Pfad, Radmacherin, Wachhund, \textit{Anbau}, \textit{Harem}, \textit{Herzog}, \textit{Maskerade}, \textit{Verschwörer}}
\end{tikzpicture}
\hspace{-0.6cm}
\begin{tikzpicture}
	\card
	\cardstrip
	\cardbanner{banner/white.png}
	\cardtitle{\scriptsize{Empfohlene 10er Sätze\qquad}}
	\cardcontent{\emph{Reisende}  (+ \textit{Seaside (2. Edition)}):\\ 
	Fruchtbares Land, Kartograph, Souk, Stallungen, Wegkreuzung, \textit{Ausguck}, \textit{Beutelschneider}, \textit{Handelsschiff}, \textit{Insel}, \textit{Lagerhaus}

	\smallskip 
	
	\emph{Läufer}  (+ \textit{Seaside (2. Edition)}):\\ 
	Berserker, Hexenkessel, Nomaden, Radmacherin, Wachhund, \textit{Bazar}, \textit{Blockade}, \textit{Karawane}, \textit{Schmuggler}, \textit{Seefahrerin}
	
	\smallskip
	
	\emph{Händler und Diebe} (+ \textit{Abenteuer}):\\ 
	Berserker, Fahrende Händlerin, Feilscher, Gewürzhändler, Wachhund, \textit{Geisterwald}, \textit{Hafenstadt}, \textit{Pagin}, \textit{Verlorene Stadt}, \textit{Weinhändler}, \textit{\underline{Überfall}}

	\smallskip 
	
	\emph{Reisen}  (+ \textit{Abenteuer}):\\ 
	Fernstraße Gasthaus, Kartograph, Pfad, Wegkreuzung, \textit{Brückentroll}, \textit{Ferne Lande}, \textit{Kundschafterin}, \textit{Riese}, \textit{Wildhüterin}, \textit{\underline{Erbschaft}}, \textit{\underline{Expedition}}
	
	\smallskip
	
	\emph{Auf zur Party} (+ \textit{Nocturne}):\\ 
	Gasthaus, Kartograph, Komplott, Oase, Souk, \textit{Druidin (Geschenk des Berges, Gesch. d. Himmels, Gesch. d. Sonne)}, \textit{Getreuer Hund}, \textit{Konklave}, \textit{Schuster}, \textit{Werwolf}

	\smallskip 
	
	\emph{Schafe zählen}  (+ \textit{Nocturne}):\\ 
	Fruchtbares Land, Oase, Tunnel, Weberin, Wegkreuzung, \textit{Geheime Höhle}, \textit{Kobold}, \textit{Krypta}, \textit{Puka}, \textit{Schäferin}}
\end{tikzpicture}
\hspace{-0.6cm}
\begin{tikzpicture}
	\card
	\cardstrip
	\cardbanner{banner/white.png}
	\cardtitle{\scriptsize{Empfohlene 10er Sätze\qquad}}
	\cardcontent{\emph{Einfache Pläne}  (+ \textit{Empires}):\\ 
	Feilscher, Grenzdorf, Hexenkessel, Radmacherin, Stallungen, \textit{Forum}, \textit{Katapult/Felsen}, \textit{Patrizier/Handelsplatz}, \textit{Villa}, \textit{Zauberin}, \textit{\underline{Spende}}, \textit{\underline{Labyrinth}}

	\smallskip 
	
	\emph{Expansion}  (+ \textit{Empires}):\\ 
	Fruchtbares Land, Gewürzhändler, Oase, Stallungen, Tunnel, \textit{Feldlager/Diebesgut}, \textit{Ingenieurin}, \textit{Legionär}, \textit{Schlösser}, \textit{Zauber}, \textit{\underline{Brunnen}}, \textit{\underline{Schlachtfeld}}
	
	\smallskip
	
	\emph{Versüßte Deals} (+ \textit{Renaissance}):\\ 
	Aufbau, Feilscher, Gewürzhändler, Hexenhütte, Wachhund, \textit{Bergdorf}, \textit{Diener}, \textit{Fahnenträger}, \textit{Gewürze}, \textit{Seidenhändlerin}, \textit{\underline{Speicher}}

	\smallskip 
	
	\emph{Ein gesparter Penny}  (+ \textit{Renaissance}):\\ 
	Berserker, Fahrende Händlerin, Grenzdorf, Oase, Souk, \textit{Freibeuterin}, \textit{Goldmünze}, \textit{Patron}, \textit{Seher}, \textit{Zepter}, \textit{\underline{Gildenhaus}}, \textit{\underline{Kaserne}}}
\end{tikzpicture}
\hspace{-0.6cm}
\begin{tikzpicture}
	\card
	\cardstrip
	\cardbanner{banner/white.png}
	\cardtitle{\scriptsize{Empfohlene 10er Sätze\qquad}}
	\cardcontent{\emph{Big Blue} (+ \textit{Menagerie}):\\ 
	Fahrende Händlerin, Hexenhütte, Pfad, Tunnel, Weberin, \textit{Dorfanger}, \textit{Falknerin}, \textit{Hirtenhund}, \textit{Schlitten}, \textit{Schwarze Katze}, \textit{\underline{Verbannung}}, \textit{\underline{Weg der Schildkröte}}

	\smallskip 
	
	\emph{Kreuzung}  (+ \textit{Menagerie}):\\ 
	Aufbau, Fruchtbares Land, Nomaden, Radmacherin, Stallungen, \textit{Drahtzieher}, \textit{Herberge}, \textit{Kardinal}, \textit{Nachschub}, \textit{Pferdestall}, \textit{\underline{Weg der Maus (mit Wegkreuzung)}}, \textit{\underline{Wagnis}}
	
	\smallskip

	\emph{Längster Tunnel}  (+ \textit{Verbündete}):\\ 
	Feilscher, Lebenskünstler, Markgraf, Pfad, Tunnel, \textit{Hauptstadt}, \textit{Schreinerin}, \textit{Tand}, \textit{Vertrag}, \textit{Wirtin}, \textit{\underline{Gefolgschaft der Schreiber}}

	\smallskip 
	
	\emph{Expertise}  (+ \textit{Verbündete}):\\ 
	Fernstraße, Gasthaus, Gewürzhändler, Grenzdorf, Wegkreuzung, \textit{Barbar}, \textit{Bürger}, \textit{Spezialistin}, \textit{Untergebener}, \textit{Wegelagerer}, \textit{\underline{Freimauerloge}}}
\end{tikzpicture}
\hspace{-0.6cm}
\begin{tikzpicture}
	\card
	\cardstrip
	\cardbanner{banner/white.png}
	\cardtitle{Platzhalter\quad}
\end{tikzpicture}
