% Basic settings for this card set
\renewcommand{\cardcolor}{cornucopia}
\renewcommand{\cardextension}{Erweiterung IV}
\renewcommand{\cardextensiontitle}{Reiche Ernte}
\renewcommand{\seticon}{cornucopia.png}

\clearpage
\newpage
\section{\cardextension \ - \cardextensiontitle \ (Rio Grande Games 2014)}

\begin{tikzpicture}
	\card
	\cardstrip
	\cardbanner{banner/white.png}
	\cardicon{icons/coin.png}
	\cardprice{2}
	\cardtitle{Weiler}
	\cardcontent{Du \emph{darfst} 1 oder 2 Karten ablegen. Wenn du 1 Karte ablegst, erhältst du  + 1 Aktion. Wenn du 2 Karten ablegst, erhältst du + 1 Aktion und + 1 Kauf. Wenn du keine Karte ablegst, erhältst du nichts.}
\end{tikzpicture}
\hspace{-0.6cm}
\begin{tikzpicture}
	\card
	\cardstrip
	\cardbanner{banner/white.png}
	\cardicon{icons/coin.png}
	\cardprice{3}
	\cardtitle{Menagerie}
	\cardcontent{Zeige deine Handkarten vor. Hast du nur Karten mit unterschiedlichen  Namen (z. B. eine \emph{MENAGERIE}, ein \emph{SILBER} und ein \emph{KUPFER}), ziehst du 3 Karten.  Hast du mindestens eine Karte doppelt auf der Hand (z. B. 2 \emph{KUPFER}), ziehst du eine Karte.}
\end{tikzpicture}
\hspace{-0.6cm}
\begin{tikzpicture}
	\card
	\cardstrip
	\cardbanner{banner/white.png}
	\cardicon{icons/coin.png}
	\cardprice{3}
	\cardtitle{Wahrsagerin}
	\cardcontent{Jeder Mitspieler, beginnend bei deinem linken Nachbarn, deckt solange Karten von seinem Nachziehstapel auf, bis er entweder eine Punkte- oder eine Fluchkarte aufgedeckt hat. Diese Karte muss er oben auf seinen Nachziehstapel legen. Ist der Nachziehstapel aufgebraucht, ohne dass eine entsprechende Karte aufgedeckt wurde, muss der Ablagestapel gemischt und weitere Karten aufgedeckt werden. Wird trotzdem keine Punkte- oder Fluchkarte aufgedeckt, legt der Spieler keine Karte auf den Nachziehstapel. Alle anderen aufgedeckten Karten werden abgelegt.}
\end{tikzpicture}
\hspace{-0.6cm}
\begin{tikzpicture}
	\card
	\cardstrip
	\cardbanner{banner/white.png}
	\cardicon{icons/coin.png}
	\cardprice{4}
	\cardtitle{Bauerndorf}
	\cardcontent{Decke solange Karten von deinem Nachziehstapel auf, bis du entweder eine Geldkarte oder eine Aktionskarte aufgedeckt hast. Nimm diese Karte auf die Hand und lege die anderen aufgedeckten Karten ab. Hast du in deinem Nachziehstapel (auch nach dem Mischen des Ablagestapels) keine Geld- oder Aktionskarte (oder entsprechende kombinierte Karte), nimmst du keine Karte auf die Hand.}
\end{tikzpicture}
\hspace{-0.6cm}
\begin{tikzpicture}
	\card
	\cardstrip
	\cardbanner{banner/white.png}
	\cardicon{icons/coin.png}
	\cardprice{4}
	\cardtitle{Junge Hexe}
	\cardcontent{Wenn die \emph{JUNGE HEXE} als Königreichkarte für das Spiel ausgewählt wurde, benötigt ihr als 11. Stapel im Vorrat einen Bannstapel (Spielvorbereitung, S. 3).

	\medskip

	Wenn du die \emph{JUNGE HEXE} ausspielst, ziehst du zuerst 2 Karten und legst dann 2 Handkarten ab. Jeder Mitspieler, der keine Bannkarte von seiner Hand aufdeckt, muss sich, beginnend bei deinem linken Nachbarn, einen \emph{FLUCH} vom Vorrat nehmen. Wird der Vorrat an \emph{FLÜCHEN} dabei aufgebraucht, müssen die Spieler, für die kein \emph{FLUCH} mehr vorhanden ist, keinen \emph{FLUCH} nehmen. 

	\medskip

	Die Spieler dürfen auch Reaktionskarten ausspielen, bevor sie eine Bannkarte ausspielen. Sind die Bannkarten gleichzeitig Reaktionskarten, dürfen sie zuerst als Reaktionskarten und dann als Bannkarten aufgedeckt werden.}
\end{tikzpicture}
\hspace{-0.6cm}
\begin{tikzpicture}
	\card
	\cardstrip
	\cardbanner{banner/white.png}
	\cardicon{icons/coin.png}
	\cardprice{4}
	\cardtitle{Nachbau}
	\cardcontent{Entsorge eine Handkarte. Nimm dafür eine Karte vom Vorrat, die genau \coin[1] mehr kostet als die entsorgte Karte. Lege die neue Karte ab. Entsorge dann eine weitere Handkarte und nimm eine Karte, die genau \coin[1] mehr kostet. Lege auch diese Karte ab. Ist im Vorrat keine Karte, die genau \coin[1] mehr kostet, musst du die Handkarte trotzdem entsorgen, erhältst dafür aber nichts. Du kannst keine Karte vom Vorrat nehmen und diese gleich wieder entsorgen, da du sie ablegen musst.}
\end{tikzpicture}
\hspace{-0.6cm}
\begin{tikzpicture}
	\card
	\cardstrip
	\cardbanner{banner/blue.png}
	\cardicon{icons/coin.png}
	\cardprice{4}
	\cardtitle{\footnotesize{Pferdehändler}}
	\cardcontent{Wenn du diese Karte ausspielst, erhältst du + 1 Kauf und +\coin[3]. Dann legst du 2 Handkarten ab. Wenn du weniger als 2 Handkarten hast, legst du so viele Karten ab wie möglich.

	\medskip

	Wenn ein Mitspieler eine Angriffskarte ausspielt, darfst du diese Karte aus deiner Hand aufdecken. Dann legst du den \emph{PFERDEHÄNDLER} zur Seite und der Angriff wird ausgeführt. Zu Beginn deines nächsten Zuges ziehst du eine Karte nach und nimmst den (oder die) zur Seite gelegten \emph{PFERDEHÄNDLER} wieder auf die Hand.}
\end{tikzpicture}
\hspace{-0.6cm}
\begin{tikzpicture}
	\card
	\cardstrip
	\cardbanner{banner/white.png}
	\cardicon{icons/coin.png}
	\cardprice{4}
	\cardtitle{Turnier}
	\cardcontent{Wenn du eine \emph{PROVINZ} aus deiner Hand ablegst, nimmst du dir entweder ein \emph{HERZOGTUM} vom Vorrat oder eine beliebige Karte vom Preisstapel. Ist der Stapel, für den du dich entscheidest, leer, nimmst du keine Karte. Lege die Karte, die du nimmst, oben auf deinen Nachziehstapel.

	\medskip

	Dann dürfen alle Mitspieler eine \emph{PROVINZ} aus ihrer Hand aufdecken. Wenn keiner eine \emph{PROVINZ} aufdeckt, erhältst du + 1 Karte und +\coin[1].}
\end{tikzpicture}
\hspace{-0.6cm}
\begin{tikzpicture}
	\card
	\cardstrip
	\cardbanner{banner/white.png}
	\cardicon{icons/coin.png}
	\cardprice{5}
	\cardtitle{Ernte}
	\cardcontent{Decke die obersten 4 Karten von deinem Nachziehstapel auf. Für jede aufgedeckte Karte mit unterschiedlichem Namen, erhältst du +\coin[1]. Kannst du (auch nach dem  Mischen des Ablagestapels) weniger als 4 Karten aufdecken, deckst du nur so viele Karten auf, wie möglich.}
\end{tikzpicture}
\hspace{-0.6cm}
\begin{tikzpicture}
	\card
	\cardstrip
	\cardbanner{banner/gold.png}
	\cardicon{icons/coin.png}
	\cardprice{5}
	\cardtitle{Füllhorn}
	\cardcontent{Diese Karte ist eine Geldkarte mit dem Basiswert 0. Wenn du das \emph{FÜLLHORN} ausspielst, zählst du zunächst, wie viele Karten \emph{mit unterschiedlichen Namen} im Spiel sind. Im Spiel sind: Geld- und Aktionskarten, die du in diesem Zug ausgespielt hast sowie evtl. bei dir ausliegende Dauerkarten (aus \emph{Seaside}). Nicht im Spiel dagegen sind Karten, die du in diesem Zug entsorgt hast. Nimm dir eine Karte aus dem Vorrat, die maximal soviel kostet, wie du Karten mit unterschiedlichen Namen im Spiel hast.

	\medskip

	Nimmst du eine Punktekarte, entsorgst du das \emph{FÜLLHORN}. Nimmst oder erhältst du eine Punktekarte auf andere Weise, entsorgst du das \emph{FÜLLHORN} nicht.}
\end{tikzpicture}
\hspace{-0.6cm}
\begin{tikzpicture}
	\card
	\cardstrip
	\cardbanner{banner/white.png}
	\cardicon{icons/coin.png}
	\cardprice{5}
	\cardtitle{Harlekin}
	\cardcontent{Jeder Mitspieler, beginnend mit deinem linken Nachbarn, deckt die oberste Karte seines Nachziehstapels auf und legt sie ab.

	\medskip

	Ist es eine Punktekarte (auch eine kombinierte), muss sich der Spieler einen \emph{FLUCH} nehmen. Wird der Vorrat an \emph{FLÜCHEN} dabei aufgebraucht, erhalten die Spieler, für die kein \emph{FLUCH} mehr vorhanden ist, keinen \emph{FLUCH}.

	\medskip

	Ist es keine Punktekarte, darfst du wählen: Entweder muss sich der Spieler eine Karte mit gleichem Namen (sofern vorhanden) aus dem Vorrat nehmen und ablegen oder du nimmst eine Karte mit gleichem Namen (sofern vorhanden) aus dem Vorrat und legst sie ab.}
\end{tikzpicture}
\hspace{-0.6cm}
\begin{tikzpicture}
	\card
	\cardstrip
	\cardbanner{banner/white.png}
	\cardicon{icons/coin.png}
	\cardprice{5}
	\cardtitle{Treibjagd}
	\cardcontent{Decke alle deine Handkarten auf. Decke dann so lange Karten vom Nachziehstapel auf, bis du die erste Karte aufdeckst, die einen Namen hat, der nicht bei deinen Handkarten dabei ist. Nimm diese Karte und die aufgedeckten Handkarten auf die Hand und lege die anderen aufgedeckten Karten ab. Kannst du (auch nach dem Mischen des Ablagestapels) eine solche Karte nicht aufdecken, nimmst du nur deine aufgedeckten Handkarten wieder auf die Hand.}
\end{tikzpicture}
\hspace{-0.6cm}
\begin{tikzpicture}
	\card
	\cardstrip
	\cardbanner{banner/green.png}
	\cardicon{icons/coin.png}
	\cardprice{6}
	\cardtitle{Festplatz}
	\cardcontent{Diese Karte ist eine Punktekarte und hat bis zum Ende des Spiels keine Funktion. Bei der Wertung des Spiels erhältst du pro 5 Karten mit unterschiedlichen Namen in deinem Kartensatz (Handkarten, Nachzieh- und Ablagestapel) 2 Siegpunkte. Ein Kartensatz mit bis zu 4 unterschiedlichen Kartennamen bringt dir zum Beispiel keine Punkte, Kartensätze mit 10 bis 14 unterschiedlichen Namen dagegen 4 \victorypoint.}
\end{tikzpicture}
\hspace{-0.6cm}
\begin{tikzpicture}
	\card
	\cardstrip
	\cardbanner{banner/white.png}
	\cardicon{icons/coin.png}
	\cardprice{0*}
	\cardtitle{Preiskarten}
	\cardcontent{\tiny{
	\vspace{1em}
	\emph{Diadem:} Das \emph{DIADEM} ist eine Geldkarte. Für jede in deinem Zug nicht verbrauchte Aktion erhältst du +\coin[1] für die Kaufphase. Hast du zum Beispiel in der Aktionsphase gar keine  Aktionskarte ausgespielt, erhältst du +\coin[1] für deine freie Aktion. Hast du zum Beispiel nur das \emph{BAUERNDORF} ausgespielt, erhältst du +\coin[2] für die beiden Aktionen des \emph{BAUERNDORFS}, da du deine freie Aktion verbraucht hast.

	\smallskip

	\emph{Ein Sack voll Gold:} Ist der Nachziehstapel leer, wenn du dir ein \emph{GOLD} nimmst, legst du das \emph{GOLD} auf die leere Stelle. Es ist dann die einzige Karte in deinem Nachziehstapel.

	\smallskip

	\emph{Gefolge:} Ist kein \emph{ANWESEN} mehr im Vorrat, erhältst du keins. Beginnend mit deinem linken Nachbarn nimmt sich jeder Mitspieler einen \emph{FLUCH}. Mitspieler, die mehr als  3 Karten auf der Hand haben, legen außerdem Karten ab, bis sie nur noch 3 Karten auf der Hand haben. Ist für einen Spieler kein \emph{FLUCH} mehr im Vorrat, erhält der Spieler keinen. Handkarten muss er dennoch ablegen, wenn er mehr als 3 Karten hat.

	\smallskip

	\emph{Prinzessin:} Die Anweisung, dass jede Karte \coin[2] weniger kostet, wenn die \emph{PRINZESSIN} im Spiel ist, betrifft die Karten auf der Hand und in den Ablage- und Nachziehstapeln aller Spieler sowie alle Karten im Vorrat. Wird die \emph{PRINZESSIN} auf einen \emph{THRONSAAL} folgend ausgespielt, kosten alle Karten trotzdem nur \coin[2] weniger, da die \emph{PRINZESSIN} nur einmal im Spiel ist.

	\smallskip

	\emph{Streitross:} Diese Karte beinhaltet 4 Anweisungen, von denen du 2 unterschiedliche wählst. Dann führst du die Anweisungen der Reihenfolge auf der Karte nach aus. Du darfst auch einen Anweisung wählen, die du nicht (oder nur teilweise) erfüllen kannst, z. B. wenn nur noch 3 \emph{SILBER} im Vorrat sind. Du darfst deinen Nachziehstapel nicht durchsehen, bevor du ihn ablegst.
	}}
\end{tikzpicture}
\hspace{-0.6cm}
\begin{tikzpicture}
	\card
	\cardstrip
	\cardbanner{banner/white.png}
	\cardtitle{\scriptsize{Empfohlene 10er Sätze\qquad}}
	\cardcontent{\emph{Kopfgeld} (Reiche Ernte + \textit{Basisspiel}):\\
	Ernte, Füllhorn, Menagerie, Treibjagd, Turnier (+ Preiskarten), \textit{Geldverleiher}, \textit{Jahrmarkt}, \textit{Keller}, \textit{Miliz}, \textit{Schmiede}

	\smallskip

	\emph{Böses Omen} (Reiche Ernte + \textit{Basisspiel}):\\
	Füllhorn, Harlekin, Nachbau, Wahrsagerin, Weiler, \textit{Abenteurer}, \textit{Bürokrat}, \textit{Laboratorium}, \textit{Spion}, \textit{Thronsaal}

	\smallskip

	\emph{Wanderzirkus} (Reiche Ernte + \textit{Basisspiel}):\\
	Bauerndorf, Festplatz, Harlekin, Junge Hexe (+ Kanzler als Bannstapel), Pferdehändler, \textit{Festmahl}, \textit{Laboratorium}, \textit{Markt}, \textit{Umbau}, \textit{Werkstatt}

	\emph{Wer zuletzt lacht} (Reiche Ernte + \textit{Die Intrige}):\\
	Bauerndorf, Ernte, Harlekin, Pferdehändler, Treibjagd, \textit{Adlige}, \textit{Handlanger}, \textit{Lakai}, \textit{Trickser}, \textit{Verwwalter}

	\smallskip

	\emph{Würze des Lebens}	(Reiche Ernte + \textit{Die Intrige}):\\
	Festplatz, Füllhorn, Junge Hexe (+ Wunschbrunnen als Bannstapel), Nachbau, Turnier (+ Preiskarten), \textit{Bergwerk}, \textit{Burghof}, \textit{Große Halle}, \textit{Kupferschmied}, \textit{Tribut} 

	\smallskip

	\emph{Kleine  Siege} (Reiche Ernte + \textit{Die Intrige}):\\
	Nachbau, Treibjagd, Turnier (+ Preiskarten), Wahrsagerin, Weiler, \textit{Große Halle}, \textit{Handlanger}, \textit{Harem}, \textit{Herzog}, \textit{Verschwörer}}
\end{tikzpicture}
\hspace{-0.6cm}
\begin{tikzpicture}
	\card
	\cardstrip
	\cardbanner{banner/white.png}
	\cardtitle{Platzhalter\quad}
\end{tikzpicture}
\hspace{0.6cm}
