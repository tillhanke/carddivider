% Basic settings for this card set
\renewcommand{\cardcolor}{darkages}
\renewcommand{\cardextension}{Erweiterung VI}
\renewcommand{\cardextensiontitle}{Dark Ages}
\renewcommand{\seticon}{darkages.png}

\clearpage
\newpage
\section{\cardextension \ - \cardextensiontitle \ (Hans Im Glück 2012)}

\begin{tikzpicture}
	\card
	\cardstrip
	\cardbanner{banner/white.png}
	\cardicon{icons/coin.png}
	\cardprice{6}
	\cardtitle{Altar}
	\cardcontent{Du musst eine Karte aus deiner Hand entsorgen, falls möglich. Dann nimmst du dir eine Karte, die bis zu \coin[5] kostet, aus dem Vorrat (auch wenn du keine Karte entsorgen konntest). Die neue Karte legst du auf deinen Ablagestapel.}
\end{tikzpicture}
\hspace{-0.6cm}
\begin{tikzpicture}
	\card
	\cardstrip
	\cardbanner{banner/white.png}
	\cardicon{icons/coin.png}
	\cardprice{1}
	\cardtitle{Armenhaus}
	\cardcontent{Wenn du diese Karte ausspielst, erhältst du zunächst +\coin[4] für die folgende Kaufphase. Dann musst du alle deine Handkarten aufdecken. Für jede Geldkarte auf deiner Hand reduzieren sich die +\coin[4] um \coin[1] (niemals jedoch unter \coin[0]). Kombinierte Geldkarten sind auch Geldkarten. Hast du z. B. 2 Kupfer auf der Hand, so bleiben dir noch +\coin[2].}
\end{tikzpicture}
\hspace{-0.6cm}
\begin{tikzpicture}
	\card
	\cardstrip
	\cardbanner{banner/white.png}
	\cardicon{icons/coin.png}
	\cardprice{5}
	\cardtitle{\footnotesize{Banditenlager}}
	\cardcontent{Du ziehst zuerst eine Karte nach, dann nimmst du dir eine Beute-Karte vom Beute-Stapel neben dem Vorrat und legst die Karte auf deinen Ablagestapel. Ist der Beute-Stapel leer, nimmst du dir keine Beute-Karte. Danach darfst du bis zu 2 weitere Aktionen ausführen.}
\end{tikzpicture}
\hspace{-0.6cm}
\begin{tikzpicture}
	\card
	\cardstrip
	\cardbanner{banner/white.png}
	\cardicon{icons/coin.png}
	\cardprice{4}
	\cardtitle{Barde}
	\cardcontent{Du ziehst zuerst eine Karte nach. Dann deckst du die obersten 3 Karten von deinem Nachziehstapel auf. Kannst du (auch nach dem Mischen des Ablagestapels) keine 3 Karten aufdecken, deckst du nur so viele Karten auf, wie möglich. Lege alle aufgedeckten Aktionskarten in beliebiger Reihenfolge zurück auf deinen Nachziehstapel und lege die übrigen aufgedeckten Karten ab. Kombinierte Aktionskarten sind auch Aktionskarten. Hast du keine Aktionskarten aufgedeckt, legst du auch keine Karten zurück auf deinen Nachziehstapel. Danach darfst du bis zu 2 weitere Aktionen ausführen.}
\end{tikzpicture}
\hspace{-0.6cm}
\begin{tikzpicture}
	\card
	\cardstrip
	\cardbanner{banner/blue.png}
	\cardicon{icons/coin.png}
	\cardprice{2}
	\cardtitle{Bettler}
	\cardcontent{Wenn du den Bettler in deiner Aktionsphase ausspielst, nimmst du dir 3 Kupfer aus dem Vorrat direkt auf deine Hand. Sind nicht mehr genügend Kupfer im Vorrat, nimmst du dir so viele, wie möglich. Wenn ein Mitspieler eine Angriffskarte ausspielt, darfst du den Bettler aus deiner Hand ablegen. (\emph{Achtung:} Der Bettler wird nicht wie üblich nur vorgezeigt und kommt dann zurück auf die Hand, sondern er wird abgelegt.) Wenn du das machst, nimmst du dir 2 Silber aus dem Vorrat. Eines davon legst du auf deinen Nachziehstapel, das andere legst du auf deinen Ablagestapel. Ist nur noch ein Silber im Vorrat so nimmst du nur dieses und legst es auf deinen Nachziehstapel. Ist kein Silber mehr im Vorrat, nimmst du dir keines.}
\end{tikzpicture}
\hspace{-0.6cm}
\begin{tikzpicture}
	\card
	\cardstrip
	\cardbanner{banner/gold.png}
	\cardicon{icons/coin.png}
	\cardprice{0*}
	\cardtitle{Beute}
	\cardcontent{Wird im Spiel mindestens eine der Königreichkarten \emph{Banditenlager}, \emph{Marodeur} oder \emph{Raubzug} verwendet, so wird auch der Beute-Stapel benötigt. Der Beute-Stapel ist niemals im Vorrat und wird \emph{neben} diesen bereit gelegt. Die Karten vom Beute-Stapel können nur durch die Anweisungen auf den 3 oben genannten Karten genommen werden. Auf andere Weise können keine Karten vom Beute-Stapel gekauft oder genommen werden. Der Botschafter (Dominion – Seaside) darf keine Karten auf den Beute-Stapel zurück legen. Der Beute-Stapel wird für die Spielende-Bedingung \emph{nicht} beachtet.
	
	\smallskip

	Die Beute ist eine Geldkarte mit Wert \coin[3], wie Gold. Wenn du die Beute ausspielst, musst du sie sofort zurück auf den Beute-Stapel legen. Du musst Geldkarten, die du auf der Hand hast, nicht ausspielen.}
\end{tikzpicture}
\hspace{-0.6cm}
\begin{tikzpicture}
	\card
	\cardstrip
	\cardbanner{banner/white.png}
	\cardicon{icons/coin.png}
	\cardprice{4}
	\cardtitle{\footnotesize{Eisenhändler}}
	\cardcontent{\emph{Errata:} Die Reihenfolge der Anweisungen ist falsch, es sollte \enquote{[...] Decke die oberste Karte von deinem Nachziehstapel auf. Du darfst sie ablegen. Unabhängig davon: Ist es eine...} heißen.
	
	\smallskip

	Du ziehst zuerst eine Karte nach, dann deckst du die oberste Karte von deinem Nachziehstapel auf. Dann entscheidest du dich, ob du die aufgedeckte Karte zurück auf den Nachziehstapel oder auf deinen Ablagestapel legst. Du erhältst einen Bonus entsprechend dem Kartentyp der aufgedeckten Karte. Ist die aufgedeckte Karte eine Karte mit kombinierten Kartentypen, erhältst du für jeden Kartentyp den angegebenen Bonus. Deckst du z. B. den Harem (Dominion – Die Intrige) auf, erhältst du +\coin[1] und +1 Karte. Danach darfst du eine weitere Aktion ausführen.}
\end{tikzpicture}
\hspace{-0.6cm}
\begin{tikzpicture}
	\card
	\cardstrip
	\cardbanner{banner/white.png}
	\cardicon{icons/coin.png}
	\cardprice{3}
	\cardtitle{Eremit}
	\cardcontent{Wenn du den Eremiten ausspielst, siehst du dir zunächst deinen Ablagestapel durch. Dann darfst du eine Karte, die keine Geldkarte ist, aus dem Ablagestapel oder aus deiner Hand entsorgen. Du musst keine Karte entsorgen und du darfst keine Geldkarte entsorgen. Kombinierte Geldkarten, wie z. B. der Harem (Dominion – Die Intrige) sind Geldkarten. Egal ob du eine Karte entsorgt hast oder nicht, musst du dir eine Karte nehmen, die bis zu \coin[3] kostet. Du nimmst diese Karte aus dem Vorrat und legst sie auf deinen Ablagestapel. Du musst eine Karte nehmen, wenn möglich. Du darfst nicht darauf verzichten.
	
	\smallskip

	Wenn du den Eremiten aus dem Spiel ablegst (normalerweise in der Aufräumphase am Ende der Runde, in der du die Karte ausgespielt hast) und du in diesem Zug keine Karte gekauft hast, entsorge den Eremiten und nimm dir einen Verrückten. Du nimmst den Verrückten vom Stapel neben dem Vorrat und legst ihn auf deinen Ablagestapel. Karten, die du auf andere Weise genommen hast, als sie zu kaufen, haben keinen Einfluss darauf, ob du den Eremiten entsorgst. Ist der Verrückten-Stapel leer, so nimmst du dir keinen. Wird der Eremit in der Aufräumphase nicht regulär abgelegt, sondern z. B. durch das Komplott (Dominion – Hinterland) zurück auf den Nachziehstapel gelegt, entsorgst du den Eremiten nicht und nimmst dir auch keinen Verrückten.}
\end{tikzpicture}
\hspace{-0.6cm}
\begin{tikzpicture}
	\card
	\cardstrip
	\cardbanner{banner/gold.png}
	\cardicon{icons/coin.png}
	\cardprice{5}
	\cardtitle{Falschgeld}
	\cardcontent{Diese Geldkarte hat den Wert \coin[1]. Du legst das Falschgeld wie üblich in der Kaufphase aus. Du erhältst +1 Kauf, dann darfst du eine Geldkarte aus deiner Hand wählen und diese zweimal ausspielen. Du erhältst also zweimal den Wert der gewählten Geldkarte (zusätzlich zu den \coin[1] Geld vom Falschgeld selbst) und führst auch zusätzliche Anweisungen auf der Karte zweimal aus. Danach musst du die gewählte Karte entsorgen. Wenn du das Falschgeld nutzt, um die Beute zweimal auszuspielen, erhältst du also \coin[6] (zusätzlich zu den \coin[1] vom Falschgeld selbst) und legst die Beute zurück auf den Beute-Stapel. Du kannst die Beute nicht entsorgen, da sie durch die Anweisung bereits anderweitig entfernt wurde. Kombinierte Geldkarten sind auch Geldkarten und können mit Falschgeld gewählt werden.}
\end{tikzpicture}
\hspace{-0.6cm}
\begin{tikzpicture}
	\card
	\cardstrip
	\cardbanner{banner/white.png}
	\cardicon{icons/coin.png}
	\cardprice{4}
	\cardtitle{Festung}
	\cardcontent{Wenn du diese Karte ausspielst, ziehst du zunächst eine Karte nach. Dann darfst du bis zu 2 weitere Aktionen ausführen.
	
	\smallskip

	Wenn du die Festung entsorgst, nimmst du die Karte zurück auf die Hand. Es spielt dabei keine Rolle, ob du die Festung in deinem eigenen Zug oder im Zug eines Mitspielers entsorgst. Die Festung gilt als entsorgt, obwohl du sie sofort zurück auf die Hand nimmst. Spielst du z. B. den Leichenkarren aus und entscheidest dich dafür, die Festung zu entsorgen, gilt die Bedingung als erfüllt und du muss den Leichenkarren nicht entsorgen.}
\end{tikzpicture}
\hspace{-0.6cm}
\begin{tikzpicture}
	\card
	\cardstrip
	\cardbanner{banner/white.png}
	\cardicon{icons/coin.png}
	\cardprice{3}
	\cardtitle{Gassenjunge}
	\cardcontent{Wenn du den Gassenjungen ausspielst, ziehst du zunächst eine Karte nach, dann muss jeder deiner Mitspieler seine Kartenhand auf 4 reduzieren. Mitspieler, die bereits 4 oder weniger Karten auf der Hand haben, legen keine Karten ab.
	
	\smallskip

	Wenn der Gassenjunge im Spiel ist und du eine weitere Angriffskarte ausspielst, darfst du (bevor du die neue Karte ausführst) den Gassenjungen entsorgen und dir einen Söldner nehmen. Du nimmst den Söldner vom Stapel neben dem Vorrat und legst ihn auf deinen Ablagestapel. Ist der Söldner-Stapel leer, so nimmst du dir keinen. Wenn du den selben Gassenjungen zweimal ausspielst, z. B. durch die Prozession, darfst du ihn nicht entsorgen um dir einen Söldner zu nehmen. Wenn du zwei einzelne Gassenjungen spielst, darfst du den ersten entsorgen und dir einen Söldner nehmen.}
\end{tikzpicture}
\hspace{-0.6cm}
\begin{tikzpicture}
	\card
	\cardstrip
	\cardbanner{banner/white.png}
	\cardicon{icons/coin.png}
	\cardprice{5}
	\cardtitle{Grabräuber}
	\cardcontent{Du musst eine der beiden Optionen wählen und diese ausführen, soweit möglich. Du darfst auch eine Option wählen, die du nicht ausführen kannst. Du darfst den Müll-Stapel jederzeit durchsehen. Wenn du dir eine Karte vom Müll-Stapel nimmst, zeigst du diese deinen Mitspielern und legst sie dann sofort auf deinen Nachziehstapel. Wenn keine Karte im Müll-Stapel ist, die \coin[3] - \coin[6] kostet, nimmst du dir keine Karte. Du darfst keine Karten mit Trank in den Kosten (Dominion – Die Alchemisten) nehmen.
	
	\smallskip

	Wenn du dich dafür entscheidest, eine Aktionskarte zu entsorgen, nimmst du dir eine Karte aus dem Vorrat und legst sie auf deinen Ablagestapel.}
\end{tikzpicture}
\hspace{-0.6cm}
\begin{tikzpicture}
	\card
	\cardstrip
	\cardbanner{banner/white.png}
	\cardicon{icons/coin.png}
	\cardprice{5}
	\cardtitle{Graf}
	\cardcontent{Diese Karte gibt dir 2 voneinander unabhängige Wahlmöglichkeiten. Du entscheidest dich zuerst, ob du 2 Handkarten ablegst oder eine Handkarte auf deinen Nachziehstapel legst oder dir ein Kupfer vom Vorrat nimmst und auf deinen Ablagestapel legst. Danach entscheidest du dich, ob du +\coin[3] möchtest oder alle deine verbliebenen Handkarten entsorgst oder dir ein Herzogtum vom Vorrat nimmst und auf deinen Ablagestapel legst. Du musst beide Wahlmöglichkeiten in dieser Reihenfolge ausführen. Du kannst z. B. zuerst 2 Handkarten ablegen und dir danach ein Herzogtum nehmen. Du darfst auch eine Option wählen, die du nicht ausführen kannst. Wenn du mehrere Karten entsorgst, die eine \enquote{Wenn du diese Karte entsorgst ...}-Anweisung haben, entsorgst du zuerst die Karten und führst die \enquote{Wenn du diese Karte entsorgst ...}-Anweisungen dann in beliebiger Reihenfolge aus.}
\end{tikzpicture}
\hspace{-0.6cm}
\begin{tikzpicture}
	\card
	\cardstrip
	\cardbanner{banner/white.png}
	\cardicon{icons/coin.png}
	\cardprice{6}
	\cardtitle{Jagdgründe}
	\cardcontent{Wenn du diese Karte ausspielst, ziehst du 4 Karten nach. Wenn du die Jagdgründe entsorgst, musst du wählen und entweder 1 Herzogtum oder 3 Anwesen vom Vorrat nehmen und auf deinen Ablagestapel legen. Du darfst auch eine Option wählen, die du nicht oder nur teilweise ausführen kannst, z. B. weil nur noch 2 Anwesen im Vorrat sind.}
\end{tikzpicture}
\hspace{-0.6cm}
\begin{tikzpicture}
	\card
	\cardstrip
	\cardbanner{banner/white.png}
	\cardicon{icons/coin.png}
	\cardprice{5}
	\cardtitle{Katakomben}
	\cardcontent{Wenn du diese Karte ausspielst, siehst du dir zuerst die obersten 3 Karten von deinem Nachziehstapel an. Dann musst du dich entscheiden, ob du entweder alle 3 Karten auf die Hand nimmst oder alle 3 Karten auf deinen Ablagestapel legst. Wenn du die 3 Karten ablegst (und nur dann), ziehst du die nächsten 3 Karten von deinem Nachziehstapel. Wenn du die Katakomben entsorgst, nimmst du dir eine Karte, die weniger kostet als die Katakomben selbst, vom Vorrat und legst sie auf deinen Ablagestapel. Dabei ist egal, ob du die Katakomben in deinem eigenen Zug oder im Zug eines Mitspielers entsorgst.}
\end{tikzpicture}
\hspace{-0.6cm}
\begin{tikzpicture}
	\card
	\cardstrip
	\cardbanner{banner/white.png}
	\cardicon{icons/coin.png}
	\cardprice{2}
	\cardtitle{Knappe}
	\cardcontent{Wenn du diese Karte ausspielst erhältst du zuerst +\coin[1]. Dann wählst du eine der 3 Optionen: +2 Aktionen, +2 Kauf oder du nimmst dir ein Silber vom Vorrat und legst es auf deinen Ablagestapel. Wenn du den Knappen entsorgst, nimmst du dir eine beliebige Angriffskarte aus dem Vorrat und legst sie auf deinen Ablagestapel. Du darfst nur sichtbare Karten aus dem Vorrat nehmen, also z. B. nur den offen liegenden Ritter.}
\end{tikzpicture}
\hspace{-0.6cm}
\begin{tikzpicture}
	\card
	\cardstrip
	\cardbanner{banner/white.png}
	\cardicon{icons/coin.png}
	\cardprice{5}
	\cardtitle{Kultist}
	\cardcontent{Wenn du diese Karte ausspielst, ziehst du zuerst 2 Karten nach. Dann muss sich, beginnend mit dem Spieler zu deiner Linken, reihum jeder Mitspieler die oberste Ruinen-Karte vom Ruinen-Stapel nehmen. Ist oder wird der Ruinen-Stapel während des Nehmens leer, können keine weiteren Ruinen mehr genommen werden. Danach darfst du einen weiteren Kultisten aus deiner Hand ausspielen. Das Ausspielen des ursprünglichen Kultisten kostet wie üblich eine Aktion. Das Ausspielen weiterer Kultisten durch die Kartenanweisung benötigt und verbraucht keine Aktion(en).
	
	\smallskip

	Wenn du einen Kultisten entsorgst, ziehst du 3 Karten nach. Dabei ist egal, ob du den Kultisten in deinem eigenen Zug oder im Zug eines Mitspielers entsorgst. Wenn du den Kultisten entsorgst, während du Karten aufdeckst, wie z. B. bei einem Ritter-Angriff, handelst du zuerst diese Aktion ab und legst die übrigen aufgedeckten Karten ab. Du ziehst also nicht die aufgedeckten Karten nach, sondern die nachfolgenden.}
\end{tikzpicture}
\hspace{-0.6cm}
\begin{tikzpicture}
	\card
	\cardstrip
	\cardbanner{banner/white.png}
	\cardicon{icons/coin.png}
	\cardprice{3}
	\cardtitle{Lagerraum}
	\cardcontent{Wenn du diese Karte ausspielst, darfst du zuerst eine beliebige Anzahl Karten (auch 0) aus deiner Hand auf den Ablagestapel legen und ebensoviele Karten nachziehen. Unabhängig davon, ob und wie viele Karten du abgelegt hast, darfst du nochmals beliebig viele deiner verbleibenden Handkarten (auch 0) ablegen und erhältst für jede dabei abgelegten Karte +\coin[1]. Weiterhin erhältst du +1 Kauf für die folgende Kaufphase.}
\end{tikzpicture}
\hspace{-0.6cm}
\begin{tikzpicture}
	\card
	\cardstrip
	\cardbanner{banner/white.png}
	\cardicon{icons/coin.png}
	\cardprice{2}
	\cardtitle{\footnotesize{Landstreicher}}
	\cardcontent{Wenn du diese Karte ausspielst, ziehst du zuerst eine Karte nach. Danach deckst du die oberste Karte von deinem Nachziehstapel auf. Ist es ein Fluch, eine Ruinen-Karte, eine Unterschlupf-Karte oder eine Punktekarte, nimm diese Karte auf die Hand. Dies gilt jeweils auch für kombinierte Kartentypen, die mindestens einen der genannten Typen enthalten. Ansonsten lege die Karte zurück auf den Nachziehstapel.}
\end{tikzpicture}
\hspace{-0.6cm}
\begin{tikzpicture}
	\card
	\cardstrip
	\cardbanner{banner/green.png}
	\cardicon{icons/coin.png}
	\cardprice{4}
	\cardtitle{Lehen}
	\cardcontent{Diese Königreichkarte ist eine Punktekarte, keine Aktionskarte. Sie hat bis zum Ende des Spiels keine Funktion. Bei der Wertung zählt sie 1 Siegpunkt für je volle 3 Silber in deinem gesamten Kartensatz (Nachziehstapel, Ablagestapel und Handkarten). Bei Spielende suchst du aus deinem gesamten Kartensatz alle Silber heraus und zählst sie. Das Ergebnis durch 3 geteilt und abgerundet ergibt die Punkte. Im Spiel zu 3. und 4. werden 12 Karten Lehen verwendet, im Spiel zu 2. werden 8 Karten Lehen verwendet. Wenn du das Lehen entsorgst (egal ob in deinem eigenen Zug oder im Zug eines Mitspielers), nimmst du dir 3 Silber vom Vorrat und legst diese auf deinen Ablagestapel. Sind nicht mehr genug Silber im Vorrat, nimmst du dir nur so viele wie möglich.}
\end{tikzpicture}
\hspace{-0.6cm}
\begin{tikzpicture}
	\card
	\cardstrip
	\cardbanner{banner/white.png}
	\cardicon{icons/coin.png}
	\cardprice{4}
	\cardtitle{\footnotesize{Leichenkarren}}
	\cardcontent{Wenn du den Leichenkarren ausspielst, musst du dich entscheiden, entweder genau eine Aktionskarte aus deiner Hand zu entsorgen oder diesen ausgespielten Leichenkarren zu entsorgen. Kombinierte Aktionskarten sind auch Aktionskarten.
	
	\smallskip

	Wenn du den Leichenkarren nimmst (durch Kauf oder durch eine andere Aktion), nimmst du dir zusätzlich die beiden obersten Ruinen-Karten vom Ruinen-Stapel und legst diese auf deinen Ablagestapel. Sind weniger als 2 Ruinen-Karten übrig, nimmst du die restlichen. Der Spieler, der den Leichenkarren nimmt, nimmt auch die Ruinen-Karten. Spielt ein Spieler die Besessenheit (Dominion – Die Alchemisten) aus und lässt sein \enquote{Opfer} einen Leichenkarren kaufen, so nimmt der Spieler, der die Besessenheit gespielt hat, am Ende den Leichenkarren und damit auch die beiden Ruinen. Verwendet ein Spieler den Fahrenden Händler (Dominion – Hinterland) und nimmt ein Silber anstatt dem Leichenkarren, so nimmt er sich keine Ruinen. Spielst du einen Botschafter (Dominion – Seaside) um Leichenkarren an deine Mitspieler zu verteilen, so nimmt sich jeder Spieler, der einen Leichenkarren nimmt, auch 2 Ruinen (wenn möglich). Wird ein Leichenkarren durch die Maskerade (Dominion – Die Intrige) weitergegeben, nimmt sich der Spieler keine Ruinen, da dies nicht als \enquote{nehmen} gilt. Weiterhin erhältst du +\coin[5] für die folgende Kaufphase.}
\end{tikzpicture}
\hspace{-0.6cm}
\begin{tikzpicture}
	\card
	\cardstrip
	\cardbanner{banner/white.png}
	\cardicon{icons/coin.png}
	\cardprice{4}
	\cardtitle{\scriptsize{Lumpensammler}}
	\cardcontent{\emph{Errata:} Der Kartentext ist falsch, es sollte \enquote{[...] Du darfst sofort deinen kompletten Nachziehstapel auf den Ablagestapel legen.} statt \enquote{[...] Du darfst sofort deinen kompletten Nachziehstapel ablegen.} heißen.
	
	\smallskip

	Du darfst deinen kompletten Nachziehstapel auf den Ablagestapel ablegen, musst dies jedoch nicht. Du musst jedoch eine Karte aus deinem Ablagestapel wählen und auf deinen Nachziehstapel legen, ausser der Ablagestapel ist leer. Ist dein Nachziehstapel leer, z. B. weil du ihn gerade komplett abgelegt hast, legst du die Karte an die Stelle des Nachziehstapels. Weiterhin erhältst du +\coin[2] für die folgende Kaufphase.
	
	\smallskip

	\emph{Anmerkung:} Durch das direkte Ablegen wird die Karte Tunnel (Dominion – Hinterland) nicht ausgelöst.}
\end{tikzpicture}
\hspace{-0.6cm}
\begin{tikzpicture}
	\card
	\cardstrip
	\cardbanner{banner/blue.png}
	\cardicon{icons/coin.png}
	\cardprice{3}
	\cardtitle{Marktplatz}
	\cardcontent{Wenn du diese Karte ausspielst ziehst du eine Karte nach. Dann darfst du eine weitere Aktion ausführen und erhältst +1 Kauf für die folgende Kaufphase.
	
	\smallskip

	Wenn du eine Karte entsorgst, darfst du den Marktplatz aus deiner Hand auf deinen Ablagestapel legen. Wenn du das machst, nimmst du 1 Gold vom Vorrat und legst es auf deinen Ablagestapel. Ist kein Gold mehr im Vorrat, nimmst du dir keines. Du darfst mehrere Karten Marktplatz aus deiner Hand ablegen, auch wenn du nur eine Karte entsorgst.}
\end{tikzpicture}
\hspace{-0.6cm}
\begin{tikzpicture}
	\card
	\cardstrip
	\cardbanner{banner/white.png}
	\cardicon{icons/coin.png}
	\cardprice{4}
	\cardtitle{Marodeur}
	\cardcontent{Wenn du diese Karte spielst, nimmst du dir zuerst eine Karte vom Beute-Stapel und legst sie auf deinen Ablagestapel. Ist der Beute-Stapel aufgebraucht, nimmst du dir keine Beute. Dann muss sich, beginnend mit dem Spieler zu deiner Linken, reihum jeder Mitspieler die oberste Karte vom Ruinen-Stapel nehmen. Ist oder wird der Ruinen-Stapel während des Nehmens leer, können keine weiteren Ruinen mehr genommen werden.}
\end{tikzpicture}
\hspace{-0.6cm}
\begin{tikzpicture}
	\card
	\cardstrip
	\cardbanner{banner/white.png}
	\cardicon{icons/coin.png}
	\cardprice{5}
	\cardtitle{Medium}
	\cardcontent{Wenn du diese Karte spielst, benennst du eine Karte (z. B. \enquote{Kupfer}, \emph{nicht} \enquote{Geld}) und deckst die oberste Karte von deinem Nachziehstapel auf. Wenn es sich um die benannte Karte handelt, nimmst du sie auf die Hand. Wenn nicht, legst du sie zurück auf den Nachziehstapel. Die benannte Karte muss eindeutig erkennbar sein, Sir Destry ist z. B. \emph{nicht} identisch mit Sir Martin. Du darfst auch eine Karte benennen, die nicht im Spiel ist.
	
	\smallskip

	Danach darfst du eine weitere Aktion ausführen und erhältst +\coin[2] für die folgende Kaufphase.}
\end{tikzpicture}
\hspace{-0.6cm}
\begin{tikzpicture}
	\card
	\cardstrip
	\cardbanner{banner/white.png}
	\cardicon{icons/coin.png}
	\cardprice{3}
	\cardtitle{Mundraub}
	\cardcontent{Wenn du diese Karte ausspielst erhältst du +1 Aktion und +1 Kauf für die folgende Kaufphase. Dann musst du eine beliebige Karte aus deiner Hand entsorgen. Wenn du keine Handkarten hast, musst du nichts entsorgen. Danach siehst du dir den Müll-Stapel durch und zählst wie viele Geldkarten mit unterschiedlichem Namen sich darin befinden. Für jede davon erhältst du +\coin[1] für die folgende Kaufphase. Befinden sich im Müll-Stapel z. B. 4 Kupfer, 1 Falschgeld und 6 Anwesen, so erhältst du +\coin[2]. Kombinierte Geldkarten (z. B. Harem, Dominion – Die Intrige) sind auch Geldkarten.}
\end{tikzpicture}
\hspace{-0.6cm}
\begin{tikzpicture}
	\card
	\cardstrip
	\cardbanner{banner/white.png}
	\cardicon{icons/coin.png}
	\cardprice{5}
	\cardtitle{Neubau}
	\cardcontent{Du benennst zuerst eine beliebige Karte. Es muss keine Punktekarte sein. Du darfst sogar eine Karte benennen, die in diesem Spiel nicht verwendet wird. Dann deckst du solange Karten von deinem Nachziehstapel auf, bis eine Punktekarte offen liegt, die nicht die zuvor benannte Karte ist. (Das Benennen einer Karte dient dazu, diese Karte zu schützen.) Wenn du auch nach dem Mischen des Ablagestapels keine solche Karte aufdecken kannst, passiert nichts weiter. Wenn du eine Punktekarte aufdeckst, die nicht die zuvor benannte Karte ist, entsorgst du diese Karte und nimmst dir eine Punktekarte, die bis zu \coin[3] mehr kostet, als die entsorgte Karte aus dem Vorrat und legst diese auf deinen Ablagestapel.}
\end{tikzpicture}
\hspace{-0.6cm}
\begin{tikzpicture}
	\card
	\cardstrip
	\cardbanner{banner/white.png}
	\cardicon{icons/coin.png}
	\cardprice{4}
	\cardtitle{Prozession}
	\cardcontent{\tiny{
	\emph{Errata:} Auf der Karte steht, dass man sich eine Karte nehmen muss, die genau \coin[1] mehr kostet. Richtig wäre: ...nimm dir eine Aktionskarte, die genau \coin[1] mehr kostet.
	
	\smallskip

	Du darfst eine Aktionskarte, die du noch auf der Hand hast, zweimal ausspielen. Du spielst die Karte zuerst aus und führst die Anweisungen komplett aus. Dann nimmst du die Karte zurück auf die Hand, spielst sie ein zweites Mal aus und führst die Anweisungen nochmal komplett aus. Das zweimalige Ausspielen der Aktionskarte kostet keine weiteren Aktionen. Das ursprüngliche Ausspielen der Prozession kostet wie üblich eine Aktion. Nachdem du die Karte zum zweiten Mal ausgeführt hast musst du sie entsorgen und dir dafür eine Aktionskarte, die genau \coin[1] mehr kostet als die entsorgte Karte, aus dem Vorrat nehmen (wenn möglich) und auf deinen Ablagestapel legen. Du musst dies auch machen, wenn du im Verlauf der Aktion den \emph{\enquote{Anschluss verloren}} hast (siehe: Neue Regeln). Du darfst andere Karten erst ausspielen, wenn die Prozession vollständig abgehandelt ist. Spielst du z. B. eine Festung auf eine Prozession, so führst du die zusätzlichen insgesamt 4 Aktionen erst aus, nachdem du die Festung entsorgt und dir eine neue Karte vom Vorrat genommen hast. Wenn du eine Prozession auf eine andere Prozession spielst, spielst du 2 weitere Aktionskarten aus deiner Hand je 2mal aus, entsorgst diese beiden Aktionskarten und nimmst dir 2 Karten vom Vorrat. Dann entsorgst du die zweite Prozession und nimmst dir auch dafür eine Karte vom Vorrat. Wenn du eine Dauer-Karte auf eine Prozession spielst, bleibt die Prozession bis zum nächsten Zug liegen, dann führst du den Effekt der Dauerkarte zweimal aus (obwohl die Dauerkarte bereits entsorgt ist).
	}}
\end{tikzpicture}
\hspace{-0.6cm}
\begin{tikzpicture}
	\card
	\cardstrip
	\cardbanner{banner/white.png}
	\cardicon{icons/coin.png}
	\cardprice{4}
	\cardtitle{Ratten}
	\cardcontent{Wenn du diese Karte ausspielst, ziehst du zuerst eine Karte nach. Dann nimmst du dir eine weitere Karte Ratten aus dem Vorrat und legst diese auf deinen Ablagestapel. Sind keine Ratten mehr im Vorrat, nimmst du dir keine. Danach musst du eine Karte aus deiner Hand entsorgen, jedoch keine Ratten. Hast du nur Ratten oder keine Karten auf der Hand, entsorgst du keine Karte. Hast du nur Ratten auf der Hand, musst du deine Kartenhand den anderen Spielern vorzeigen.
	
	\smallskip

	Wenn du die Ratten entsorgst, ziehst du sofort eine Karte nach. Dabei ist egal, ob du die Ratten in deinem eigenen Zug oder im Zug eines Mitspielers entsorgst.}
\end{tikzpicture}
\hspace{-0.6cm}
\begin{tikzpicture}
	\card
	\cardstrip
	\cardbanner{banner/white.png}
	\cardicon{icons/coin.png}
	\cardprice{5}
	\cardtitle{Raubzug}
	\cardcontent{Du entsorgst diese Karte direkt nachdem du sie ausgespielt hast. Dann muss jeder Mitspieler, beginnend mit dem Spieler zu deiner Linken, seine komplette Kartenhand aufdecken. Du wählst bei jedem Mitspieler eine seiner Handkarten, die dieser ablegen muss. Danach nimmst du dir 2 Karten vom Beute-Stapel neben dem Vorrat und legst diese auf deinen Ablagestapel. Sind nicht mehr genügend Karten im Beute-Stapel, nimmst du nur so viele wie möglich.}
\end{tikzpicture}
\hspace{-0.6cm}
\begin{tikzpicture}
	\card
	\cardstrip
	\cardbanner{banner/white.png}
	\cardicon{icons/coin.png}
	\cardprice{5}
	\cardtitle{Ritter}
	\cardcontent{\miniscule{
	\vspace{1em}
	Im Ritter-Stapel befinden sich bei Spielbeginn 10 Karten mit jeweils individuellem Namen. Alle Ritter haben die selbe Basisfähigkeit, zusätzlich jedoch noch eine individuelle Anweisung. Wenn die Ritter als Königreichkarte für ein Spiel gewählt werden, werden die Karten mit dem Typ \enquote{Ritter} gemischt, als verdeckter Stapel in den Vorrat gelegt und die oberste Karte davon aufgedeckt. Siehe hierzu auch unter Spielvorbereitung \enquote{gemischte Stapel}. Auf den Ritter-Stapel wird kein Marker für die Handelsroute (Dominion – Blütezeit) gelegt, auch wenn die oberste Karte eine Punktekarte ist. Bis auf diese Ausnahmen wird der Ritter-Stapel behandelt, wie jeder andere der 10 Kartenstapel. Er ist Teil des Vorrats und wird bei der Spielendebedingung beachtet. Die Anweisungen auf den Ritterkarten werden wie üblich von oben nach unten ausgeführt. Sir Michael z. B. lässt die Mitspieler Karten ablegen, bevor seine Basisfähigkeit ausgeführt wird. Die Basisfähigkeit, die alle Ritter gemeinsam haben, wird wie folgt ausgeführt: Beginnend mit dem Spieler links von dir, muss jeder Mitspieler die obersten beiden Karten von seinem Nachziehstapel aufdecken. Hat der Mitspieler keine Karte aufgedeckt, die 3 - 6 Geld kostet, muss er keine Karte entsorgen. Er legt beide aufgedeckten Karten ab. Kostet genau eine der beiden aufgedeckten Karten 3 - 6 Geld, so muss er diese Karte entsorgen, die andere Karte legt er ab. Kosten beide aufgedeckten Karten 3 - 6 Geld, so darf der Mitspieler entscheiden, welche der beiden Karten er entsorgen muss. Die andere aufgedeckte Karte legt er ab. Entsorgt ein Mitspieler auf diese Weise eine andere Ritter-Karte, so musst auch du den Ritter entsorgen, der den Angriff ausgelöst hat. Die Mitspieler dürfen keine Karten mit Trank in den Kosten (Dominion – Die Alchemisten) entsorgen. Sollen die Ritter im Schwarzmarktstapel verwendet werden, so wird einer der 10 Ritter zufällig gezogen und in den Schwarzmarktstapel eingemischt.
	
	\smallskip

	Die individuellen Fähigkeiten der einzelnen Ritter sind selbsterklärend. Darum hier nur kurze Anmerkungen zu einzel- nen Rittern. \emph{Sir Martin:} Es ist beabsichtigt, dass Sir Martin nur 4 kostet. Sir Martin ist der einzige Ritter, den ein Spieler z. B. durch die Waffenkammer nehmen kann (natürlich nur, wenn er offen liegt). \emph{Sir Vander:} Das Gold nimmst du aus dem Vorrat und legst es auf deinen Ablagestapel. Es ist egal, ob du Sir Vander in deinem eigenen Zug oder im Zug eines Mitspielers entsorgst. \emph{Dame Natalie:} Die Karte darfst (emph{Errata:} auf der Karte ist die Anweisung fälschlicherweise nicht optional) du aus dem Vorrat nehmen und sie dann auf deinen Ablagestapel legen. \emph{Dame Anna:} Du musst keine Karten entsorgen. \emph{Dame Josephine} ist eine Punktekarte. Der Ritter-Stapel ist jedoch dadurch kein Punktestapel. Wird die Handelsroute (Dominion – Blütezeit) verwendet, wird kein Marker auf den Ritter-Stapel gelegt.
	}}
\end{tikzpicture}
\hspace{-0.6cm}
\begin{tikzpicture}
	\card
	\cardstrip
	\cardbanner{banner/brown.png}
	\cardicon{icons/coin.png}
	\cardprice{0}
	\cardtitle{Ruinen}
	\cardcontent{Der Ruinen-Stapel besteht aus bis zu 5 unterschiedlichen Karten. Siehe hierzu auch unter Spielvorbereitung \enquote{gemischte Stapel} und \enquote{Ruinen}. Der Ruinen-Stapel ist Teil des Vorrats. Die offenliegende Karte (und nur diese) kann bzw. muss durch eine entsprechende Kartenanweisung genommen oder auch gekauft werden.
	
	\smallskip

	Wenn du die Überlebenden ausspielst, siehst du dir die obersten beiden Karten von deinem Nachziehstapel an und entscheidest dich dann, ob du beide Karten auf deinen Ablagestapel legst oder ob du beide Karten in beliebiger Reihenfolge zurück auf deinen Nachziehstapel legst. Du darfst nicht eine Karte ablegen und die andere zurück auf den Nachziehstapel legen.
	
	\smallskip

	Die übrigen 4 Ruinen-Karten sind selbsterklärend.}
\end{tikzpicture}
\hspace{-0.6cm}
\begin{tikzpicture}
	\card
	\cardstrip
	\cardbanner{banner/white.png}
	\cardicon{icons/coin.png}
	\cardprice{5}
	\cardtitle{\scriptsize{Schrotthändler}}
	\cardcontent{Wenn du diese Karte ausspielst, ziehst du zuerst eine Karte nach. Dann musst du eine Karte entsorgen, wenn du mindestens eine Karte auf der Hand hast. Danach darfst du eine weitere Aktion ausführen und erhältst +\coin[1] für die folgende Kaufphase.}
\end{tikzpicture}
\hspace{-0.6cm}
\begin{tikzpicture}
	\card
	\cardstrip
	\cardbanner{banner/white.png}
	\cardicon{icons/coin.png}
	\cardprice{5}
	\cardtitle{Schurke}
	\cardcontent{Wenn im Müll-Stapel mindestens eine Karte ist, die \coin[3] - \coin[6] kostet, musst du eine davon nehmen und auf deinen Ablagestapel legen. Du darfst nicht darauf verzichten. Du darfst den Müllstapel durchsehen, bevor du den Schurken ausspielst. Du zeigst die Karte, die du nimmst deinen Mitspielern. Du darfst keine Karten mit Trank in den Kosten (Dominion – Die Alchemisten) nehmen.
	
	\smallskip

	Wenn keine Karte im Müll-Stapel ist, die \coin[3] - \coin[6] kostet, nimmst du dir keine Karte. Stattdessen muss jeder Mitspieler, beginnend mit dem Spieler links von dir, die obersten beiden Karten von seinem Nachziehstapel aufdecken und eine Karte davon entsorgen, die \coin[3] - \coin[6] kostet. Hat der Mitspieler keine Karte aufgedeckt, die 3 - 6 Geld kostet, muss er keine Karte entsorgen. Er legt beide aufgedeckten Karten ab. Kostet genau eine der beiden aufgedeckten Karten \coin[3] - \coin[6], so muss er diese Karte entsorgen, die andere Karte legt er ab. Kosten beide aufgedeckten Karten \coin[3] - \coin[6], so darf der Mitspieler entscheiden, welche der beiden Karten er entsorgen muss. Die andere aufgedeckte Karte legt er ab.}
\end{tikzpicture}
\hspace{-0.6cm}
\begin{tikzpicture}
	\card
	\cardstrip
	\cardbanner{banner/white.png}
	\cardicon{icons/coin.png}
	\cardprice{0*}
	\cardtitle{Söldner}
	\cardcontent{Wird der Gassenjunge im Spiel verwendet, so wird auch der Söldner-Stapel benötigt. Der Söldner-Stapel wird \emph{neben} dem Vorrat bereit gelegt. Die Karten vom Söldner-Stapel können nur durch die Anweisung auf dem Gassenjunge genommen werden. Auf andere Weise können keine Karten vom Söldner-Stapel gekauft oder genommen werden. Der Botschafter (Dominion – Seaside) darf keine Karten auf den Söldner-Stapel zurück legen. Der Söldner-Stapel wird für die Spielende-Bedingung nicht beachtet.
	
	Wenn du den Söldner ausspielst, darfst du 2 Karten aus deiner Hand entsorgen. Wenn du das machst, musst du zuerst 2 Karten nachziehen und erhältst +\coin[2] für die folgende Kaufphase. Dann müssen alle deine Mitspieler, beginnend mit dem Spieler links von dir, solange Handkarten ablegen, bis sie nur noch höchstens 3 Karten auf der Hand haben. Mitspieler, die bereits 3 oder weniger Karten auf der Hand haben, legen keine Karten mehr ab. Spieler, die mit dem Bettler auf diesen Angriff reagieren wollen, müssen diesen ablegen, bevor du dich entscheidest 2 Karten zu entsorgen oder nicht. Wenn du dich dafür entscheidest, 2 Karten zu entsorgen, jedoch nur noch 1 Karte auf der Hand hast. Entsorgst du nur diese 1 Karte. Da die nachfolgenden Anweisungen jedoch an das entsorgen von 2 Karten gebunden ist, ziehst du keine Karten nach und erhältst auch kein virtuelles Geld. Wenn die Karten, die du entsorgst, Entsorgungs-Anweisungen haben, entsorgst du zuerst beide und führst die Anweisungen dann in beliebiger Reihenfolge aus.}
\end{tikzpicture}
\hspace{-0.6cm}
\begin{tikzpicture}
	\card
	\cardstrip
	\cardbanner{banner/red.png}
	\cardicon{icons/coin.png}
	\cardprice{1}
	\cardtitle{\footnotesize{Unterschlupf}}
	\cardcontent{\tiny{
	\vspace{1em}
	Unterschlupfkarten haben keinen Stapel (weder im Vorrat noch ausserhalb des Vorrats), sie können niemals gekauft werden.
	
	\smallskip

	\emph{Hütte:} Diese Karte ist ein Unterschlupf, siehe \enquote{Besondere Karten} und eine Reaktionskarte. Die Hütte ist niemals im Vorrat. Wenn du eine Punktekarte kaufst und die Hütte auf der Hand hast, darfst du die Hütte entsorgen. Dies gilt auch, wenn du eine kombinierte Punktekarte kaufst. Du erhältst keine weiteren Vorteile durch das Entsorgen der Hütte. Du wirst nur eine Karte los.
	
	\smallskip

	\emph{Totenstadt:} Diese Karte ist ein Unterschlupf, siehe \enquote{Besondere Karten} und eine Aktionskarte. Die Totenstadt ist niemals im Vorrat. Wenn du diese Karte ausspielst, erhältst du +2 Aktionen.
	
	\smallskip

	\emph{Verfallenes Anwesen:} Diese Karte ist ein Unterschlupf, siehe \enquote{Besondere Karten} und eine Punktekarte mit dem Wert 0 Siegpunkte. Das Verfallene Anwesen ist niemals im Vorrat. Wenn du diese Karte entsorgst, ziehst du sofort eine Karte nach, auch wenn du gerade eine andere Aktionskarte ausführst. Wenn du das Verfallene Anwesen z. B. durch den Altar entsorgst, ziehst du zuerst eine Karte nach und nimmst dir danach eine Karte vom Vorrat. Du darfst das Verfallene Anwesen nicht \enquote{freiwillig} entsorgen, sondern benötigst wie üblich eine andere Karte mit der Anweisung, die dir erlaubt, eine Karte zu entsorgen.
	}}
\end{tikzpicture}
\hspace{-0.6cm}
\begin{tikzpicture}
	\card
	\cardstrip
	\cardbanner{banner/white.png}
	\cardicon{icons/coin.png}
	\cardprice{0*}
	\cardtitle{Verrückter}
	\cardcontent{Wird der Eremit im Spiel verwendet, so wird auch der Verrückten-Stapel benötigt. Der Verrückten-Stapel wird \emph{neben} dem Vorrat bereit gelegt. Die Karten vom Verrückten-Stapel können nur durch die Anweisung auf dem Eremiten genommen werden. Auf andere Weise können keine Karten vom Verrückten-Stapel gekauft oder genommen werden. Der Botschafter (Dominion – Seaside) darf keine Karten auf den Verrückten-Stapel zurück legen. Der Verrückten-Stapel wird für die Spielende-Bedingung \emph{nicht} beachtet.
	
	\smallskip

	Diese Karte ist nicht Teil des Vorrats. Sie kann nur durch die Anweisung auf dem Eremiten genommen werden. Auf eine andere Art kann der Verrückte nicht genommen oder gekauft werden. Wenn du den Verrückten ausspielst, erhältst du +2 Aktionen. Dann legst du den Verrückten normalerweise auf den Verrückten-Stapel zurück und ziehst pro Karte, die du noch auf der Hand hast, eine Karte nach (du verdoppelst also die Anzahl deiner Handkarten). Es kann jedoch vorkommen, dass du den Verrückten nicht zurück legen kannst, weil du den \enquote{Anschluss verloren} hast (siehe Neue Regeln), z. B. weil du den Verrückten auf eine Prozession oder einen Thronsaal (Dominion – Basisspiel) ausgespielt hast.}
\end{tikzpicture}
\hspace{-0.6cm}
\begin{tikzpicture}
	\card
	\cardstrip
	\cardbanner{banner/white.png}
	\cardicon{icons/coin.png}
	\cardprice{5}
	\cardtitle{Vogelfreie}
	\cardcontent{\miniscule{
	\vspace{1em}
	Wenn du diese Karte ausspielst, wählst du eine Karte aus dem Vorrat, die weniger kostet, als die Vogelfreien selbst (also normalerweise bis zu 4 Geld). Dann behandelst du die vor dir liegende Karte Vogelfreie genau so, als wäre es die gewählte Karte. Normalerweise führst du einfach nur die Anweisungen auf der gewählten Karte aus. Wählst du z. B. die Festung, so ziehst du eine Karte nach und führst danach bis zu 2 weitere Aktionen aus. Die Vogelfreien nehmen auch die Kosten, den Namen und den oder die Kartentypen der gewählten Karte an.
	
	\smallskip

	Wenn du mit den Vogelfreien eine Karte wählst, die sich selbst entsorgt, so entsorgst du die Vogelfreien. Wenn die Karte nicht mehr vor dir liegt, wird sie wieder zu der Karte Vogelfreie selbst. Wenn du die Vogelfreien als Dauer-Karte (Dominion – Seaside) verwendest, bleibt die Karte bis zu deinem nächsten Zug im Spiel. Wenn du die Vogelfreien als Thronsaal (Dominion – Basisspiel), Königshof (Dominion – Blütezeit) oder Prozession verwendest um eine Dauer-Karte mehrmals auszuspielen, bleiben die Vogelfreien ebenso im Spiel. Wenn du die Vogelfreien mehrmals ausspielst (z. B. durch den Thronsaal), wählst du nur beim ersten Ausspielen eine Karte. Die Vogelfreien bleiben diese Karte auch beim nachfolgenden Ausspielen. Nutzt du z. B. die Prozession, um die Vogelfreien zweimal auszuspielen und kopierst die Festung, so ziehst du insgesamt 2 Karten nach und erhältst +4 Aktionen. Danach entsorgst du die ausliegende Karte, nimmst sie aber sofort wieder auf die Hand zurück, da es noch immer eine Festung ist. Zurück auf deiner Hand ist die Karte aus dem Spiel und wird wieder zu den Vogelfreien. Nun nimmst du dir noch eine Karte aus dem Vorrat, die genau 1 Geld mehr kostet, als die entsorgte Karte. Die Karte, die du entsorgt hast, ist inzwischen wieder die Karte Vogelfreie 5 Geld. Du nimmst dir also eine Karte die 6 Geld kostet. Wenn du die Vogelfreien als eine Karte nutzt, die etwas in der Aufräumphase macht (z. B. der Eremit), so führst du diese Anweisung aus. Wenn du das Füllhorn (Dominion – Reiche Ernte) ausspielst, werden die Vogelfreien als die Karte, die sie kopieren angesehen. Wenn du z. B. mit einer Karte Vogelfreie die Festung wählst und mit einer weiteren Karte Vogelfreie den Lumpensammler, nimmst du dir eine Karte, die bis zu 3 Geld kostet. Du darfst mit den Vogelfreien nur eine Karte wählen, die sichtbar im Vorrat liegt. Du darfst keine Karte wählen, deren Stapel leer ist. Du darfst keine Karte wählen, die nicht im Vorrat ist, wie z. B. den Söldner. In gemischten Stapeln, wie z. B. Ruinen oder Ritter, darfst du nur die oben liegende wählen.
	}}
\end{tikzpicture}
\hspace{-0.6cm}
\begin{tikzpicture}
	\card
	\cardstrip
	\cardbanner{banner/white.png}
	\cardicon{icons/coin.png}
	\cardprice{4}
	\cardtitle{\footnotesize{Waffenkammer}}
	\cardcontent{Wenn du diese Karte ausspielst, nimmst du dir eine Karte, die bis zu \coin[4] kostet vom Vorrat. Du legst diese sofort verdeckt auf deinen Nachziehstapel, anstatt auf den Ablagestapel wie üblich.}
\end{tikzpicture}
\hspace{-0.6cm}
\begin{tikzpicture}
	\card
	\cardstrip
	\cardbanner{banner/white.png}
	\cardicon{icons/coin.png}
	\cardprice{3}
	\cardtitle{Weiser}
	\cardcontent{Wenn du diese Karte ausspielst, erhältst du immer +1 Aktion, dann deckst du solange Karten von deinem Nachziehstapel auf, bis du eine Karte aufdeckst, die mindestens \coin[3] kostet und nimmst diese Karte auf die Hand. Lege die übrigen durch diese Aktion aufgedeckten Karten ab. Wenn du auch nach dem Mischen deines Ablagestapels keine Karte, die mindestens \coin[3] kostet, aufdecken kannst, legst du alle aufgedeckten Karten ab und nimmst keine Karte auf die Hand. Deckst du z. B. ein Kupfer, dann einen Fluch und dann eine Provinz auf, nimmst du die Provinz auf die Hand und legst Kupfer und Fluch ab.}
\end{tikzpicture}
\hspace{-0.6cm}
\begin{tikzpicture}
	\card
	\cardstrip
	\cardbanner{banner/white.png}
	\cardtitle{\scriptsize{Empfohlene 10er Sätze\qquad}}
	\cardcontent{\emph{Dark Ages:}
	
	\smallskip

	\emph{Leichenzug:} \\ 
	Festung, Jagdgründe, Katakomben, Kultist, Marktplatz, Mundraub, Prozession, Ritter, Vogelfreie, Waffenkammer

	\smallskip 
	
	\emph{Spiel mit dem Teufel} \\ 
	Banditenlager, Grabräuber, Lagerraum, Landstreicher, Lumpensammler, Medium, Ratten, Raubzug, Schrotthändler, Weiser

	\smallskip 
	
	\emph{Dark Ages und Basisspiel:} 

	\smallskip 
	
	\emph{Auf und Ab:} \\ 
	Armenhaus, Barde, Eremit, Jagdgründe, Medium / Geldverleiher, Hexe, Keller, Thronsaal, Werkstatt

	\smallskip 
	
	\emph{Ritterspiele:} \\ 
	Altar, Knappe, Lumpensammler, Ratten, Ritter / Bibliothek, Gärten, Jahrmarkt, Laboratorium, Umbau}
\end{tikzpicture}
\hspace{-0.6cm}
\begin{tikzpicture}
	\card
	\cardstrip
	\cardbanner{banner/white.png}
	\cardtitle{\scriptsize{Empfohlene 10er Sätze\qquad}}
	\cardcontent{\emph{Dark Ages und Die Intrige:}
	
	\smallskip

	\emph{Prophezeiung:} \\ 
	Eisenhändler, Landstreicher, Medium, Neubau, Waffenkammer / Adelige, Baron, Große Halle, Verschwörer, Wunschbrunnen

	\smallskip 
	
	\emph{Invasion:} \\ 
	Bettler, Gassenjunge, Knappe, Marodeur, Schurke / Anbau, Bergwerksdorf, Harem, Kerkermeister, Trickser
	
	\smallskip 
	
	\emph{Dark Ages und Seaside:} 

	\smallskip 
	
	\emph{Nasses Grab:} \\ 
	Eremit, Gassenjunge, Grabräuber, Graf, Lumpensammler / Eingeborenendorf, Müllverwerter, Piratenschiff, Schatzkarte, Schatzkammer

	\smallskip 
	
	\emph{Einfaches Volk:} \\ 
	Armenhaus, Gassenjunge, Landstreicher, Lehen, Leichenkarren / Fischerdorf, Hafen, Insel, Ausguck, Lagerhaus}
\end{tikzpicture}
\hspace{-0.6cm}
\begin{tikzpicture}
	\card
	\cardstrip
	\cardbanner{banner/white.png}
	\cardtitle{\scriptsize{Empfohlene 10er Sätze\qquad}}
	\cardcontent{\emph{Dark Ages und die Alchemisten:}
	
	\smallskip

	\emph{Seuchenherd:} \\ 
	Barde, Kultist, Lehen, Marktplatz, Ratten, Waffenkammer / Lehrling, Vision, Verwandlung, Weinberg

	\smallskip 
	
	\emph{Klagelied:} \\ 
	Bettler, Eisenhändler, Falschgeld, Katakomben, Mundraub, Raubzug / Apotheker, Golem, Kräuterkundiger, Universität
	
	\smallskip 
	
	\emph{Dark Ages und Blütezeit:}

	\smallskip 

	\emph{Des einen Müll ...:} \\ 
	Falschgeld, Grabräuber, Marktplatz, Mundraub, Schurke / Abenteuer, Denkmal, Großer Markt, Stadt, Talisman
	
	\smallskip 
	
	\emph{Ehre unter Dieben:} \\ 
	Banditenlager, Knappe, Neubau, Prozession, Schurke / Kunstschmiede, Hort, Hausierer, Steinbruch, Wachturm}
\end{tikzpicture}
\hspace{-0.6cm}
\begin{tikzpicture}
	\card
	\cardstrip
	\cardbanner{banner/white.png}
	\cardtitle{\scriptsize{Empfohlene 10er Sätze\qquad}}
	\cardcontent{\emph{Dark Ages und Reiche Ernte:}
	
	\smallskip

	\emph{Dunkler Karneval:} \\ 
	Eremit, Festung, Kultist, Ritter, Schrotthändler, Vogelfreie / Festplatz, Füllhorn, Menagerie, Weiler
	
	\smallskip 
	
	\emph{Auf den Sieger:} \\ 
	Banditenlager, Barde, Falschgeld, Leichenkarren, Marodeur, Raubzug / Ernte, Treibjagd Nachbau, Turnier
	
	\smallskip 
	
	\emph{Dark Ages und Hinterland:} 
	
	\smallskip 
	
	\emph{Fern von Zuhause:} \\ 
	Barde, Bettler, Graf, Lehen, Marodeur / Botschaft, Feilscher, Aufbau, Kartograph, Katzengold
	
	\smallskip 
	
	\emph{Expeditionen:} \\ 
	Altar, Armenhaus, Eisenhändler, Katakomben, Lagerraum / Fruchtbares Land, Fernstraße, Gewürzhändler, Tunnel, Wegkreuzung}
\end{tikzpicture}
\hspace{-0.6cm}
\begin{tikzpicture}
	\card
	\cardstrip
	\cardbanner{banner/white.png}
	\cardtitle{\scriptsize{Neue Regeln (1/5)}\qquad}
	\cardcontent{\tiny{
	\vspace{1em}
	\emph{\enquote{ausspielen}:} Ein Spieler, der eine Karte ausspielt, legt diese offen vor sich aus und führt dann die Anweisungen von oben nach unten aus. Auch wenn die Karte nicht vor dem Spieler liegen bleibt (sondern z. B. wie die Beute oder der Verrückte sofort auf den jeweiligen Stapel zurück gelegt werden), werden die Anweisungen soweit möglich vollständig ausgeführt.
	
	\smallskip

	\emph{\enquote{Im Spiel} (\enquote{ausgespielt}):} Aktions- und Geldkarten, die ein Spieler offen vor sich ausgelegt hat, sind \enquote{im Spiel} bis sie aufgeräumt werden (üblicherweise in der Aufräumphase abgelegt werden). Beiseite gelegte und entsorgte Karten, sowie Karten, welche die Spieler auf der Hand haben und Karten im Vorrat und in den Nachzieh- und Ablagestapeln der Spieler sind nicht \enquote{im Spiel}. Das Aufdecken einer Reaktionskarte bringt diese nicht \enquote{ins Spiel}.
	
	\smallskip

	\emph{Kaufphase:} In der Kaufphase muss der Spieler seine Geldkarten einzeln auslegen. Er darf dabei jedoch die Reihenfolge selbst bestimmen. Legt ein Spieler eine Geldkarte mit zusätzlichen Anweisungen aus, führt er diese Anweisungen nach Möglichkeit aus, bevor er eine weitere Geldkarte auslegt. Der Spieler muss alle Geldkarten, die er in dieser Runde auslegen möchte auslegen, bevor er eine Karte kauft (auch wenn er mehrere Käufe zur Verfügung hat). Der Spieler darf in dieser Kaufphase keine Geldkarten mehr auslegen, nachdem er eine Karte gekauft hat.
	
	\smallskip

	\emph{Geldwert:} Geldkarten \enquote{produzieren} ihren Wert im Moment, in dem sie ausgelegt werden. Werden Geldkarten noch während der Kaufphase verändert oder abgelegt (z. B. Beute), bleibt der \enquote{produzierte} Geldwert für diese Kaufphase trotzdem erhalten.
	}}
\end{tikzpicture}
\hspace{-0.6cm}
\begin{tikzpicture}
	\card
	\cardstrip
	\cardbanner{banner/white.png}
	\cardtitle{\scriptsize{Neue Regeln (2/5)}\qquad}
	\cardcontent{\tiny{
	\vspace{1em}
	\emph{Geldkarten mit zusätzlichen Anweisungen:} Dark Ages beinhaltet 2 Geldkarten mit zusätzlichen Anweisungen, Falschgeld und Beute. Es handelt sich dabei nicht um Basiskarten, die in jedem Spiel verwendet werden (wie z. B. Gold). Diese Geldkarten können wie üblich in der Kaufphase ausgelegt werden. Sobald ein Spieler eine Geldkarte mit Anweisungen auslegt, führt er die Anweisung auf der Karte aus. Diese Geldkarten sind, wie jede andere Geldkarte, von Anweisungen betroffen, die sich auf Geldkarten beziehen, z. B. Dieb (Dominion-Basisspiel).
	
	\smallskip

	\emph{Wenn mehrere Anweisungen gleichzeitig ausgeführt werden können} darfst du entscheiden, in welcher Reihenfolge du diese Anweisungen ausführst. Entsorgst du z. B. eine Karte Ratten und legst den Marktplatz aus deiner Hand ab, so entscheidest du, ob du zuerst die Anweisung der Ratten oder die Anweisung auf dem Marktplatz ausführst. Wenn mehrere Spieler von einer Anweisung betroffen sind (meistens durch Angriffskarten), so wird die Anweisung normalerweise in Spielerreihenfolge ausgeführt.
	
	\smallskip

	\emph{Mehrfache Reaktionen:} Jeder Spieler darf auf ein einziges Ereignis (meist ein Angriff ) auch mit mehreren Reaktionskarten reagieren, wenn er diese auf der Hand hat. Wie üblich muss jedoch zuerst eine Karte vollständig ausgeführt werden, bevor der Spieler eine weitere Reaktionskarte aus seiner Hand aufdeckt bzw. ablegt. Ein Spieler darf z. B. als Reaktion auf einen Angriff die Geheimkammer (Dominion – Die Intrige) aufdecken. Nachdem er die Geheimkammer vollständig ausgeführt hat, darf er auch noch den Bettler aus seiner Hand ablegen (auch wenn er diesen gerade durch die Geheimkammer nachgezogen hat) und auch diesen ausführen.
	}}
\end{tikzpicture}
\hspace{-0.6cm}
\begin{tikzpicture}
	\card
	\cardstrip
	\cardbanner{banner/white.png}
	\cardtitle{\scriptsize{Neue Regeln (3/5)}\qquad}
	\cardcontent{\tiny{
	\vspace{1em}
	\emph{Anschluss verloren:} In einigen Fällen ist es möglich, dass eine Kartenanweisung verlangt, eine Karte zu bewegen (z. B. abzulegen), die Karte jedoch bereits durch eine andere Anweisung anderweitig bewegt (z. B. entsorgt) wurde. Das hat zur Folge, dass du die Karte nicht bewegen kannst. Du führst die Anweisungen auf der Karte trotzdem aus (soweit möglich). Wenn du z. B. auf eine Prozession einen Verrückten spielst, legst du den Verrückten zurück auf den Verrückten-Stapel. Du erhältst +2 Aktionen und ziehst Karten nach. Nun musst du den Verrückten zum zweiten Mal ausspielen. Die Karte liegt jedoch bereits wieder auf dem Verrückten-Stapel. Du erhältst trotzdem nochmals die +2 Aktionen, kannst die Karte jedoch nicht mehr zurück auf den Stapel legen (sie liegt bereits dort). Du ziehst also keine Karten nach, weil diese Anweisung nur aktiv wird, wenn du die Karte auf den Verrückten-Stapel zurück legst. Danach müsstest du den Verrückten entsorgen, da dieser bereits auf seinen Stapel zurück gelegt ist, kannst du ihn nicht auf den Müllstapel legen. Du nimmst dir jedoch trotzdem eine Karte die genau 1 Geld kostet, wenn möglich.
	}}
\end{tikzpicture}
\hspace{-0.6cm}
\begin{tikzpicture}
	\card
	\cardstrip
	\cardbanner{banner/white.png}
	\cardtitle{\scriptsize{Neue Regeln (4/5)}\qquad}
	\cardcontent{\tiny{
	\vspace{1em}
	\emph{\enquote{Wenn du diese Karte entsorgst: ...}:} Immer, wenn du eine deiner Karten entsorgst, die eine solche Anweisung haben, führst du diese Anweisung aus. Es ist egal, ob du die Karte in deinem Zug oder im Zug eines Mitspielers (z. B. durch eine Angriffskarte) entsorgst. Jeder Spieler entsorgt immer nur seine eigenen Karten. Im Falle eines Angriffs entsorgt also nicht der Spieler, der die Angriffskarte ausgespielt hat, die Karten seiner Mitspieler, sondern immer nur der durch den Angriff betroffene Spieler. Der Spieler, der seine Karte entsorgt, führt dann auch die Anweisung aus, direkt nachdem er die Karte auf den Müll-Stapel gelegt hat. Dies kann dazu führen, dass der Spieler die Anweisung ausführt, bevor eine andere Aktionskarte vollständig ausgeführt wurde. Spielt ein Spieler z. B. einen Grabräuber und nutzt diesen, um einen Kultisten zu entsorgen, so zieht er zuerst 3 Karten für den Kultisten nach und nimmt danach eine Karte durch den Grabräuber. In einigen Fällen werden Karten aus dem Stapel des Spielers entfernt, ohne entsorgt zu werden, z. B. durch den Botschafter (Dominion – Seaside) oder die Maskerade (Dominion – Die Intrige). Wenn durch eine Anweisung mehrere Karten gleichzeitig entsorgt werden, so entsorgst du zunächst die Karten und führst dann die Entsorgungs-Anweisungen in beliebiger Reihenfolge aus.
	}}
\end{tikzpicture}
\hspace{-0.6cm}
\begin{tikzpicture}
	\card
	\cardstrip
	\cardbanner{banner/white.png}
	\cardtitle{\scriptsize{Neue Regeln (5/5)}\qquad}
	\cardcontent{\tiny{
	\vspace{1em}
	\emph{Nur zur Erinnerung:} Immer wenn ein Spieler Karten nachziehen müsste, sein Nachziehstapel je- doch leer ist, mischt der Spieler seinen Ablagestapel und legt ihn als neuen Nachziehstapel bereit.
	
	\smallskip

	Müsste der Spieler mehr Karten nachziehen, als sein Nachziehstapel noch enthält, so zieht er zuerst die verbleibenden Karten seines Nachziehstapels, mischt dann seinen Ablagestapel und zieht die restlichen Karten nach. Kann er, auch nachdem er seinen Ablagestapel gemischt und als neuen Nachziehstapel bereit gelegt hat, nicht genügend Karten nachziehen, so zieht er nur so viele Karten, wie möglich.
	
	\smallskip

	Gleiches gilt, wenn der Spieler Karten von seinem Nachziehstapel aufdecken oder ansehen muss. Muss ein Spieler Karten auf seinen Nachziehstapel legen und ist sein Nachziehstapel in diesem Moment leer, so legt er diese Karten an die Stelle seines Nachziehstapels. Er mischt seinen Ablagestapel nicht.
	}}
\end{tikzpicture}
\hspace{-0.6cm}
\begin{tikzpicture}
	\card
	\cardstrip
	\cardbanner{banner/white.png}
	\cardtitle{Platzhalter\quad}
\end{tikzpicture}
\hspace{0.6cm}
