% Basic settings for this card set
\renewcommand{\cardcolor}{intrigue}
\renewcommand{\cardextension}{Second Edition}
\renewcommand{\cardextensiontitle}{Die Intrige}
\renewcommand{\seticon}{intrigue1.png}

% \clearpage
% \newpage
\section{\cardextension \ - \cardextensiontitle \ (Rio Grande Games 2017)}

\begin{tikzpicture}
	\card
	\cardstrip
	\cardbanner{banner/white.png}
	\cardicon{icons/coin.png}
	\cardprice{2}
	\cardtitle{Burghof}
	\cardcontent{Du ziehst 3 Karten von deinem Nachziehstapel und nimmst sie auf die Hand. Dann wählst du eine beliebige Karte aus deiner Hand und legst sie verdeckt auf den Nachziehstapel.}
\end{tikzpicture}
\hspace{-0.6cm}
\begin{tikzpicture}
	\card
	\cardstrip
	\cardbanner{banner/white.png}
	\cardicon{icons/coin.png}
	\cardprice{2}
	\cardtitle{Handlanger}
	\cardcontent{Du darfst von den vier Anweisungen der Karte \emph{genau 2} auswählen und diese für deinen Zug nutzen. Du musst zwei verschiedene Anweisungen wählen und darfst nicht z. B. zwei Karten ziehen oder 2 zusätzliche Käufe tätigen. Du musst dich sofort entscheiden, welche zwei Anweisungen du nutzen möchtest. Du darfst nicht eine Karte nachziehen und dich erst dann entscheiden, welche zweite Anweisung du ausführen möchtest.}
\end{tikzpicture}
\hspace{-0.6cm}
\begin{tikzpicture}
	\card
	\cardstrip
	\cardbanner{banner/white.png}
	\cardicon{icons/coin.png}
	\cardprice{3}
	\cardtitle{Armenviertel}
	\cardcontent{Du darfst in deiner Aktionsphase zwei zusätzliche Aktionen ausführen. Zeige deine Kartenhand vor. Wenn du keine Aktionskarte (oder Aktions-/Punktekarte) auf der Hand hast, ziehst du zwei Karten nach. Sollten sich darunter Aktionskarten befinden, darfst du diese auch gleich nutzen.}
\end{tikzpicture}
\hspace{-0.6cm}
\begin{tikzpicture}
	\card
	\cardstrip
	\cardbanner{banner/white.png}
	\cardicon{icons/coin.png}
	\cardprice{3}
	\cardtitle{Maskerade}
	\cardcontent{Ziehe zwei Karten. Anschließend wählen alle Spieler (einschließlich dir selbst) eine beliebige Karte aus ihrer Hand und legen sie verdeckt neben ihrem linken Nachbarn ab. Wenn alle Karten verteilt sind, nimmt jeder Spieler die erhaltene Karte auf die Hand. Da die \emph{MASKERADE} keine Angriffskarte ist, dürfen die anderen Spieler keine Reaktionskarten vorzeigen. Danach darfst du eine Karte aus deiner Hand entsorgen.}
\end{tikzpicture}
\hspace{-0.6cm}
\begin{tikzpicture}
	\card
	\cardstrip
	\cardbanner{banner/white.png}
	\cardicon{icons/coin.png}
	\cardprice{3}
	\cardtitle{Trickser}
	\cardcontent{Alle Mitspieler müssen die oberste Karte ihres Nachziehstapels aufdecken und diese entsorgen. Du wählst für jeden Mitspieler jeweils eine Karte aus dem Vorrat, die genauso viel kostet, wie die entsorgte Karte und gibst sie dem Mitspieler. Dieser legt die neue Karte auf seinen Ablagestapel. Befindet sich keine Karte mit den gleichen Kosten im Vorrat, muss der Mitspieler seine Karte trotzdem entsorgen, erhält dafür aber keine neue Karte. Du darfst dem Mitspieler auch eine gleiche Karte wie die, die er entsorgt hat, zurückgeben. }
\end{tikzpicture}
\hspace{-0.6cm}
\begin{tikzpicture}
	\card
	\cardstrip
	\cardbanner{banner/white.png}
	\cardicon{icons/coin.png}
	\cardprice{3}
	\cardtitle{Verwalter}
	\cardcontent{Du wählst von den drei Anweisungen der Karte \emph{genau 1} aus und führst diese dann komplett aus. Wenn du dich entscheidest, zwei Karten zu entsorgen, du aber nur eine Karte auf der Hand hast, musst du diese entsorgen.}
\end{tikzpicture}
\hspace{-0.6cm}
\begin{tikzpicture}
	\card
	\cardstrip
	\cardbanner{banner/white.png}
	\cardicon{icons/coin.png}
	\cardprice{3}
	\cardtitle{\scriptsize{Wunschbrunnen}}
	\cardcontent{Zuerst ziehst du eine Karte. Du darfst in der Aktionsphase eine zusätzliche Aktion ausführen. Nenne eine Karte (z.B. \emph{KUPFER}) und decke die oberste Karte deines Nachziehstapels auf. Handelt es sich dabei um die von dir genannte Karte, nimmst du sie auf die Hand. Ansonsten legst du sie zurück auf den Nachziehstapel.}
\end{tikzpicture}
\hspace{-0.6cm}
\begin{tikzpicture}
	\card
	\cardstrip
	\cardbanner{banner/white.png}
	\cardicon{icons/coin.png}
	\cardprice{4}
	\cardtitle{Baron}
	\cardcontent{Du \emph{darfst} ein Anwesen (sofern du gerade eins auf der Hand hast) ablegen und erhältst dafür +\coin[4] für die Kaufphase. Wenn du das nicht tun kannst (weil du kein Anwesen auf der Hand hast) oder willst, musst du dir ein Anwesen nehmen, solange noch welche im Vorrat sind.}
\end{tikzpicture}
\hspace{-0.6cm}
\begin{tikzpicture}
	\card
	\cardstrip
	\cardbanner{banner/white.png}
	\cardicon{icons/coin.png}
	\cardprice{4}
	\cardtitle{Bergwerk}
	\cardcontent{Du ziehst zuerst eine Karte nach und \emph{darfst} dann diese Karte entsorgen, bevor du ggf. weitere Aktionen ausspielst. Du erhältst dafür +\coin[2]. Wenn du das \emph{BERGWERK} auf einen \emph{THRONSAAL} spielst, erhältst du den Bonus für das Entsorgen nur einmal, da du die Karte nur einmal entsorgen kannst. Die anderen Anweisungen (+ 1 Karte sowie + 2 Aktionen) werden durch den \emph{THRONSAAL} dagegen verdoppelt.}
\end{tikzpicture}
\hspace{-0.6cm}
\begin{tikzpicture}
	\card
	\cardstrip
	\cardbanner{banner/white.png}
	\cardicon{icons/coin.png}
	\cardprice{4}
	\cardtitle{Brücke}
	\cardcontent{Die Kosten aller Karten (auch Handkarten, Karten aus den Nachzieh- und Ablagestapeln) werden in diesem Spielzug für alle Belange um \coin[1] reduziert (nicht aber unter \coin[0]). Dieser Effekt ist kumulativ, d. h. die Kosten pro Karte können durch das Ausspielen bestimmter Karten (z. B. den \emph{THRONSAAL}) auch um \coin[2] oder mehr reduziert werden.}
\end{tikzpicture}
\hspace{-0.6cm}
\begin{tikzpicture}
	\card
	\cardstrip
	\cardbanner{banner/white.png}
	\cardicon{icons/coin.png}
	\cardprice{4}
	\cardtitle{Eisenhütte}
	\cardcontent{Du nimmst dir eine beliebige Karte vom Vorrat, die maximal \coin[4] kostet. Durch das Ausspielen bestimmter Aktionskarten (z. B. die \emph{BRÜCKE}) können die Kosten der Karten reduziert werden.

	\medskip

	Je nachdem, ob du dich für eine Aktions-, Geld- oder Punktekarte entschieden hast, erhältst du einen anderen Bonus. Solltest du dich für eine kombinierte Karte entscheiden, erhältst du die Boni beider Kartentypen.}
\end{tikzpicture}
\hspace{-0.6cm}
\begin{tikzpicture}
	\card
	\cardstrip
	\cardbanner{banner/white.png}
	\cardicon{icons/coin.png}
	\cardprice{4}
	\cardtitle{Verschwörer}
	\cardcontent{Wenn du zu dem Zeitpunkt an dem du den \emph{VERSCHWÖRER} spielst, bereits mindestens 3 Aktionskarten (inklusive diesem \emph{VERSCHWÖRER}) ausgespielt hast, erhältst du den Bonus. Wenn du erst im weiteren Verlauf deiner Aktionsphase diese Bedingung erfüllst, erhältst du den Bonus nicht. Aktionskarten, die z. B. durch den \emph{THRONSAAL} doppelt ausgespielt werden dürfen, gelten als 2 ausgespielte Karten.}
\end{tikzpicture}
\hspace{-0.6cm}
\begin{tikzpicture}
	\card
	\cardstrip
	\cardbanner{banner/white.png}
	\cardicon{icons/coin.png}
	\cardprice{5}
	\cardtitle{Anbau}
	\cardcontent{Nachdem du dir eine Karte genommen und eine zusätzliche Aktion erhalten hast, musst du eine Karte aus deiner Hand entsorgen sofern du noch Handkarten hast. Du nimmst dir dafür eine Karte vom Vorrat, die \emph{genau} \coin[1] mehr kostet als die entsorgte Karte. Wenn keine solche Karte vorhanden ist, musst du zwar eine Karte entsorgen, erhältst aber keine Karte vom Vorrat. }
\end{tikzpicture}
\hspace{-0.6cm}
\begin{tikzpicture}
	\card
	\cardstrip
	\cardbanner{banner/white.png}
	\cardicon{icons/coin.png}
	\cardprice{5}
	\cardtitle{\footnotesize{Handelsposten}}
	\cardcontent{Wenn du nur eine Karte auf der Hand hast, musst du sie entsorgen, erhältst dafür aber kein Silber. Wenn du zwei oder mehr Karten auf der Hand hast, musst du genau zwei Karten entsorgen und nimmst dir dafür ein Silber direkt auf die Hand. Sollte kein Silber mehr im Vorrat sein, musst du die Karten trotzdem entsorgen, erhältst aber kein Silber. }
\end{tikzpicture}
\hspace{-0.6cm}
\begin{tikzpicture}
	\card
	\cardstrip
	\cardbanner{banner/green.png}
	\cardicon{icons/coin.png}
	\cardprice{5}
	\cardtitle{Herzog}
	\cardcontent{Diese Karte ist die einzige reine Punktekarte unter den Königreichkarten. Sie hat bis zum Ende des Spiels keine Funktion. Bei Spielende erhält der Spieler, der diese Karte in seinem Kartensatz (Nachziehstapel, Handkarten und Ablagestapel) hat, für jedes \emph{HERZOGTUM} im Kartensatz 1 Siegpunkt. Wer mehrere \emph{HERZÖGE} besitzt, erhält für jeden \emph{HERZOG} die entsprechende Anzahl Siegpunkte.} 
\end{tikzpicture}
\hspace{-0.6cm}
\begin{tikzpicture}
	\card
	\cardstrip
	\cardbanner{banner/white.png}
	\cardicon{icons/coin.png}
	\cardprice{5}
	\cardtitle{\footnotesize{Kerkermeister}}
	\cardcontent{Jeder Mitspieler (beginnend bei deinem linken Nachbarn) muss entweder zwei Karten ablegen oder einen \emph{FLUCH} vom Stapel nehmen. Ein Spieler kann sich entscheiden, die Karten abzulegen, auch wenn er nur eine oder gar keine Karte auf der Hand hat. Er legt dann nur so viele Karten ab, wie er kann. Er kann sich auch entscheiden, einen \emph{FLUCH} zu nehmen, wenn es keine \emph{FLÜCHE} mehr im Vorrat gibt.}
\end{tikzpicture}
\hspace{-0.6cm}
\begin{tikzpicture}
	\card
	\cardstrip
	\cardbanner{banner/white.png}
	\cardicon{icons/coin.png}
	\cardprice{5}
	\cardtitle{Lakai}
	\cardcontent{Du entscheidest dich für eine der beiden Optionen: Entweder erhältst du +\coin[2] in diesem Zug oder du legst alle deine Handkarten ab und ziehst vier neue Karten nach. Wenn du die zweite Option wählst, müssen außerdem alle Mitspieler, die fünf oder mehr Karten auf der Hand haben (alle anderen sind nicht betroffen), diese ablegen und ebenfalls vier Karten nachziehen. Jeder Spieler (auch wenn er von dem Angriff nicht betroffen ist) kann eine oder mehrere Reaktionskarten vorzeigen, wenn du den \emph{LAKAIEN} spielst.}
\end{tikzpicture}
\hspace{-0.6cm}
\begin{tikzpicture}
	\card
	\cardstrip
	\cardbanner{banner/whitegreen.png}
	\cardicon{icons/coin.png}
	\cardprice{6}
	\cardtitle{Adelige}
	\cardcontent{Diese Karte ist eine kombinierte Aktions- und Punktekarte. Sie kann in der Aktionsphase eingesetzt werden und bringt außerdem bei Spielende 2 Siegpunkte. Wenn du diese Karte ausspielst, musst du dich entscheiden, ob du entweder 3 Karten ziehst oder  2 weitere Aktionen ausspielen willst. Du darfst die Anweisungen aber nicht mischen oder aufteilen. Wenn du die Karte das nächste Mal auf der Hand hast und ausspielst, darfst du natürlich eine andere Wahl treffen.}
\end{tikzpicture}
\hspace{-0.6cm}
\begin{tikzpicture}
	\card
	\cardstrip
	\cardbanner{banner/goldgreen.png}
	\cardicon{icons/coin.png}
	\cardprice{6}
	\cardtitle{Harem}
	\cardcontent{Diese Karte ist eine kombinierte Geld- und Punktekarte. Sie wird während des Zugs wie eine normale Geldkarte eingesetzt und bringt außerdem bei Spielende 2 Siegpunkte.}
\end{tikzpicture}
\hspace{-0.6cm}
\begin{tikzpicture}
	\card
	\cardstrip
	\cardbanner{banner/white.png}
	\cardicon{icons/coin.png}
	\cardprice{2}
	\cardtitle{\scriptsize{Herumtreiberin}}
	\cardcontent{Die Karte, die du entsorgst oder vom Müllstapel nimmst, muss den Typ AKTION beinhalten, d. h. sie kann auch eine kombinierte Aktionskarte (z. B. \emph{MÜHLE}) sein. Genommene Karten werden auf den Ablagestapel gelegt – es sei denn, auf der Karte steht etwas anderes. Wird eine Karte entsorgt, die einen speziellen Effekt beim Entsorgen hat, tritt dieser ein.}
\end{tikzpicture}
\hspace{-0.6cm}
\begin{tikzpicture}
	\card
	\cardstrip
	\cardbanner{banner/blue.png}
	\cardicon{icons/coin.png}
	\cardprice{4}
	\cardtitle{Diplomatin}
	\cardcontent{Diese Karte ist eine Aktions- und Reaktionskarte. Wird sie als Aktion in der Aktionsphase ausgespielt, nimmst du 2 Karten. Hast du dann 5 oder weniger Karten auf der Hand, erhältst du außerdem +2 Aktionen. 
	
	\medskip

	Spielt ein Mitspieler eine Angriffskarte aus und du hast zu diesem Zeitpunkt 5 oder mehr Karten auf der Hand, darfst du diese Karte – bevor der ausgespielte Angriff ausgeführt wird – aus der Hand aufdecken. Wenn du das tust, nimmst du diese \emph{DIPLOMATIN} wieder auf die Hand, ziehst 2 Karten und legst dann 3 Karten (auch möglich inklusive dieser \emph{DIPLOMATIN}) ab. Hast du dann immer noch 5 oder mehr Karten sowie eine \emph{DIPLOMATIN} auf der Hand, darfst du die \emph{DIPLOMATIN} noch einmal aufdecken – und dies so oft wiederholen wie du möchtest und die Bedingung der 5 oder mehr Karten auf der Hand erfüllt ist. Erst dann wird der Angriff ausgeführt. Hast du mehrere Reaktionskarten auf der Hand, mit denen du auf das Ausspielen einer Angriffskarte reagieren kannst, darfst du diese \emph{nacheinander} in beliebiger Reihenfolge aufdecken.}
\end{tikzpicture}
\hspace{-0.6cm}
\begin{tikzpicture}
	\card
	\cardstrip
	\cardbanner{banner/white.png}
	\cardicon{icons/coin.png}
	\cardprice{4}
	\cardtitle{Geheimgang}
	\cardcontent{Du ziehst 2 Karten und erhältst +1 Aktion. Dann nimmst du eine beliebige Karte aus deiner Hand (auch ggf. eine, die du gerade gezogen hast) und legst sie an eine beliebige Stelle in deinen Nachziehstapel. Du darfst sie oben drauf, unten drunter oder irgendwo in die Mitte legen. Du darfst dabei die Karten deines Nachziehstapels zählen, aber nicht ansehen. Befinden sich keine Karten in deinem Nachziehstapel, wird die zurückgelegte Karte zur einzigen Karte in deinem Nachziehstapel.}
\end{tikzpicture}
\hspace{-0.6cm}
\begin{tikzpicture}
	\card
	\cardstrip
	\cardbanner{banner/whitegreen.png}
	\cardicon{icons/coin.png}
	\cardprice{4}
	\cardtitle{Mühle}
	\cardcontent{Diese Karte ist eine kombinierte Aktions- und Punktekarte. Als Punktekarte bringt sie beim Zählen der Punkte 1\victorypoint. Spielst du die \emph{MÜHLE} als Aktionskarte aus, ziehst du 1 Karte und erhältst + 1 Aktion. Du darfst 2 Karten aus deiner Hand ablegen. Wenn du das tust, erhältst du +\coin[2]. Tust du das nicht (weil du zum Beispiel nicht genügend Karten auf der Hand hast), erhältst du nichts. Nur, wenn du nicht mehr als eine Karte auf der Hand hast, darfst du genau eine Karte ablegen, erhältst dafür aber kein \coin.}
\end{tikzpicture}
\hspace{-0.6cm}
\begin{tikzpicture}
	\card
	\cardstrip
	\cardbanner{banner/white.png}
	\cardicon{icons/coin.png}
	\cardprice{5}
	\cardtitle{Austausch}
	\cardcontent{Entsorge zuerst eine Karte aus deiner Hand. Dann nimmst du dir eine Karte vom Vorrat, die maximal \coin[2] mehr kostet als die entsorgte Karte. Eine Karte kostet nur dann maximal \coin[2] mehr, wenn die restlichen Kosten (z. B. \hex aus Empires oder \potion aus Alchemie) gleich oder niedriger sind. Wenn die genommene Karte eine Aktions- und/oder Geldkarte ist, legst du die Karte oben auf deinen Nachziehstapel. Ansonsten legst du die Karte auf den Ablagestapel. Ist die genommene Karte eine Punktekarte, nimmt sich jeder Mitspieler – beginnend bei deinem linken Mitspieler – einen Fluch. Ist die genommene Karte eine Punktekarte sowie eine Aktions- oder Geldkarte (z. B. \emph{MÜHLE}), legst du die Karte oben auf deinen Nachziehstapel und jeder Mitspieler muss sich einen Fluch nehmen.}
\end{tikzpicture}
\hspace{-0.6cm}
\begin{tikzpicture}
	\card
	\cardstrip
	\cardbanner{banner/white.png}
	\cardicon{icons/coin.png}
	\cardprice{5}
	\cardtitle{Höflinge}
	\cardcontent{Decke eine Karte aus deiner Hand auf. Zähle dann die Typen, denen diese Karte angehört – also AKTION, GELD, REAKTION, ANGRIFF, PUNKTE, FLUCH etc. Pro Typ, dem die Karte angehört, entscheidest du dich für eine der vier angegebenen Optionen. Dabei darfst du keine der Optionen doppelt auswählen. 
	
	\medskip

	Wenn du zum Beispiel eine \emph{PATROUILLE} (AKTION) aufdeckst, darfst du eine Option auswählen, deckst du einen \emph{KARAWANENWÄCHTER} aus Abenteuer (AKTION – DAUER – REAKTION) auf, darfst du 3 unterschiedliche Optionen wählen. Entscheidest du dich für das Gold, legst du dieses auf den Ablagestapel. Kannst du keine Handkarte aufdecken, erhältst du nichts.}
\end{tikzpicture}
\hspace{-0.6cm}
\begin{tikzpicture}
	\card
	\cardstrip
	\cardbanner{banner/white.png}
	\cardicon{icons/coin.png}
	\cardprice{5}
	\cardtitle{Patrouille}
	\cardcontent{Ziehe zuerst 3 Karten. Decke dann die obersten 4 Karten deines Nachziehstapels auf. So aufgedeckte Punktekarten (auch ggf. kombinierte) und Flüche nimmst du alle auf die Hand. Die restlichen Karten legst du in beliebiger Reihenfolge zurück auf den Nachziehstapel.}
\end{tikzpicture}
\hspace{-0.6cm}
\begin{tikzpicture}
	\card
	\cardstrip
	\cardbanner{banner/white.png}
	\cardtitle{\scriptsize{Empfohlene 10er Sätze\qquad}}
	\cardcontent{\emph{Siegestanz:}\\
	Adlige, Austausch, Baron, Eisenhütte, Harem, Herzog, Höflinge, Maskerade, Mühle, Patrouille

	\smallskip

	\emph{Verschwörung:}\\
	Bergwerk, Eisenhütte, Geheimgang, Handelsposten, Handlanger, Herumtreiberin, Kerkermeister, Trickser, Verschwörer, Verwalter 

	\smallskip

	\emph{Beste Wünsche:}\\
	Anbau, Armenviertel, Baron, Burghof, Diplomatin, Geheimgang, Herzog, Kerkermeister, Verschwörer, Wunschbrunnen

	\smallskip

	\emph{Untergebene} (Intrige + \textit{Basisspiel}):\\
	Adlige, Diplomatin, Handlanger, Höflinge, Lakai, \textit{Bibliothek}, \textit{Jahrmarkt}, \textit{Keller}, \textit{Torwächterin}, \textit{Vasall}

	\smallskip

	\emph{Das große Ganze} (Intrige + \textit{Basisspiel}):\\
	Armenviertel, Bergwerk, Brücke, Mühle, Patrouille, \textit{Markt}, \textit{Miliz}, \textit{Ratsversammlung}, \textit{Töpferei}, \textit{Werkstatt}

	\smallskip

	\emph{Dekonstruktion} (Intrige + \textit{Basisspiel}):\\
	Austausch, Diplomatin, Harem, Herumtreiberin, Trickser, \textit{Banditin}, \textit{Dorf}, \textit{Mine}, \textit{Thronsaal}, \textit{Umbau}}
\end{tikzpicture}
\hspace{-0.6cm}
\begin{tikzpicture}
	\card
	\cardstrip
	\cardbanner{banner/white.png}
	\cardtitle{Platzhalter\quad}
\end{tikzpicture}
\hspace{-0.6cm}