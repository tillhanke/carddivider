% Basic settings for this card set
\renewcommand{\cardcolor}{hinterlands}
\renewcommand{\cardextension}{Update Pack}
\renewcommand{\cardextensiontitle}{Hinterland}
\renewcommand{\seticon}{hinterlands.png}

\clearpage
\newpage
\section{\cardextension \ - \cardextensiontitle \ (Rio Grande Games 2022)}

\begin{tikzpicture}
	\card
	\cardstrip
	\cardbanner{banner/blue.png}
	\cardicon{icons/coin.png}
	\cardprice{3}
	\cardtitle{Wachhund}
	\cardcontent{Wenn du diese Karte spielst, ziehst du 2 Karten und zählst dann die Karten auf deiner Hand. Sind das 5 oder weniger, ziehst du zwei weitere Karten. Wenn ein:e Mitspieler:in eine Angriffskarte spielt, darfst du diese Karte spielen, bevor der Angriff ausgeführt wird. Dann wird der Angriff ausgeführt (außer du verhinderst dies auf andere Weise). Wenn du diese Karte als Reaktion nutzt, spielst du sie, das heißt sie kommt ins Spiel und du folgst ihren Anweisungen. Wenn du sie während des Zuges eines Mitspielers/einer Mitspielerin spielst, legst du sie während jener Aufräumphase ab.}
\end{tikzpicture}
\hspace{-0.6cm}
\begin{tikzpicture}
	\card
	\cardstrip
	\cardbanner{banner/white.png}
	\cardicon{icons/coin.png}
	\cardprice{4}
	\cardtitle{Nomaden}
	\cardcontent{Wenn du diese Karte während des Zuges eines Mitspielers/einer Mitspielerin nimmst oder entsorgst, erhältst du trotzdem die +\coin[2], du kannst dies aber normalerweise nicht nutzen.}
\end{tikzpicture}
\hspace{-0.6cm}
\begin{tikzpicture}
	\card
	\cardstrip
	\cardbanner{banner/blue.png}
	\cardicon{icons/coin.png}
	\cardprice{4}
	\cardtitle{Pfad}
	\cardcontent{Wenn du diese Karte spielst, erhältst du einfach nur \emph{+1 Karte} und \emph{+1 Aktion}.	Wenn du diese Karte außerhalb der Aufräumphase nimmst, entsorgst oder ablegst (vgl. den Abschnitt: \emph{Wenn du ... ablegst} der neuen Anweisungen), darfst du sie spielen. Die Karte zu spielen bedeutet sie ins Spiel zu bringen und ihre Anweisungen auszuführen. Wenn du den \emph{PFAD} im Zug eines Mitspielers / einer Mitspielerin spielst, kannst du die \emph{+1 Aktion} nicht nutzen und legst den \emph{PFAD} in jener Aufräumphase ab. Wenn du einen \emph{PFAD} entsorgst, kommt er wieder ins Spiel und wird in dieser Aufräumphase abgelegt. Dies zählt trotzdem als \enquote{entsorgen}, aber nicht als \enquote{nehmen}. Wenn du einen \emph{PFAD} mithilfe des \emph{UMBAUS} (aus dem \emph{Basisspiel}) aus der Hand entsorgst und dir dafür ein \emph{GOLD} nimmst, darfst du den entsorgten \emph{PFAD} trotzdem spielen und so weiter.}
\end{tikzpicture}
\hspace{-0.6cm}
\begin{tikzpicture}
	\card
	\cardstrip
	\cardbanner{banner/blue.png}
	\cardicon{icons/coin.png}
	\cardprice{4}
	\cardtitle{Weberin}
	\cardcontent{Entweder nimmst du 2 \emph{SILBER} oder eine Karte, die bis zu \coin[4] kostet (das kann auch ein \emph{SILBER} sein). Wenn du diese Karte außerhalb der Aufräumphase ablegst (vgl. den Abschnitt: \emph{Wenn du ... ablegst} der neuen Anweisungen), darfst du sie spielen. Die Karte zu spielen bedeutet sie ins Spiel zu bringen and ihre Anweisungen zu befolgen. Wenn du die \emph{WEBERIN} im Zug eines Mitspielers/einer Mitspielerin spielst, legst du sie in jener Aufräumphase ab.}
\end{tikzpicture}
\hspace{-0.6cm}
\begin{tikzpicture}
	\card
	\cardstrip
	\cardbanner{banner/white.png}
	\cardicon{icons/coin.png}
	\cardprice{5}
	\cardtitle{Berserker}
	\cardcontent{Wenn du einen \emph{BERSERKER} nimmst, während du eine Aktionskarte im Spiel hast, spielst du den genommenen \emph{BERSERKER}, das heißt die genommene Karte kommt ins Spiel. Damit (oder wenn du einen \emph{BERSERKER} aus der Hand spielst) wird die Anweisung über der Trennlinie aktiviert, d. h. du nimmst eine billigere Karte und dann legen alle Mitspielerinnen ihre Handkarten bis auf 3 ab. Karten wie der \emph{BURGGRABEN} können von den Mitspielerinnen genutzt werden. Nimmst du einen \emph{BERSERKER} und hast keine Aktionskarte im Spiel, spielst du den \emph{BERSERKER} nicht.}
\end{tikzpicture}
\hspace{-0.6cm}
\begin{tikzpicture}
	\card
	\cardstrip
	\cardbanner{banner/white.png}
	\cardicon{icons/coin.png}
	\cardprice{5}
	\cardtitle{Hexenhütte}
	\cardcontent{Du musst beide abgelegte Karten aufdecken, auch wenn sie nicht beide Aktionskarten sind. Wenn sie beide Aktionskarten sind - auch wenn sie zusätzlich noch andere Typen enthalten - nehmen alle Mitspieler:innen einen \emph{FLUCH}.}
\end{tikzpicture}
\hspace{-0.6cm}
\begin{tikzpicture}
	\card
	\cardstrip
	\cardbanner{banner/gold.png}
	\cardicon{icons/coin.png}
	\cardprice{5}
	\cardtitle{Hexenkessel}
	\cardcontent{Der Effekt, dass jeder Mitspieler einen \emph{FLUCH} nehmen muss, liegt in der Zukunft, muss aber nicht eintreten. Wenn du in einem Zug, \emph{bevor} du den \emph{HEXENKESSEL} spielst, schon drei Aktionskarten genommen hast, nehmen die Mitspieler:innen keinen \emph{FLUCH}. Es ist egal, wie viele Nicht-Aktionskarten du nimmst, oder wie viele Aktionskarten du nach dem Spielen des \emph{HEXENKESSELS} noch nimmst, einzig relevant ist die Gesamtanzahl deiner genommenen Aktionskarten im aktuellen Zug. Geschieht das Nehmen deiner 3. Aktionskarte dieses Zuges, nachdem du einen \emph{HEXENKESSEL} gespielt hast, nehmen alle Mitspieler:innen einen \emph{FLUCH}. Dies ist kumulativ, d. h. wenn du mehrere \emph{HEXENKESSEL} spielst, nehmen alle Mitspielerinnen pro \emph{HEXENKESSEL} einen \emph{FLUCH}. Der \emph{HEXENKESSEL} ist eine Geldkarte, d.h. du spielst ihn regulär in deiner Kaufphase. Da er aber auch eine Angriffskarte ist, dürfen Reaktionskarten wie der \emph{WACHHUND} und der \emph{BURGGRABEN} genutzt werden.}
\end{tikzpicture}
\hspace{-0.6cm}
\begin{tikzpicture}
	\card
	\cardstrip
	\cardbanner{banner/white.png}
	\cardicon{icons/coin.png}
	\cardprice{5}
	\cardtitle{Radmacherin}
	\cardcontent{Du darfst jeden beliebigen Kartentyp ablegen, aber die genommene Karte muss eine Aktionskarte sein. Wenn du eine Aktionskarte ablegst, kannst du eine gleiche Aktionskarte nehmen.}
\end{tikzpicture}
\hspace{-0.6cm}
\begin{tikzpicture}
	\card
	\cardstrip
	\cardbanner{banner/white.png}
	\cardicon{icons/coin.png}
	\cardprice{5}
	\cardtitle{Souk}
	\cardcontent{Wenn du zum Beispiel einen \emph{SOUK} spielst, während du noch 3 andere Karten auf deiner Hand hast, erhältst du \emph{+1 Kauf} und \coin[7] und verlierst dann \coin[3], für insgesamt \emph{+1 Kauf} und +\coin[4]. Du kannst nicht weniger als \coin[0] haben, aber wenn du vorher schon Geldwerte bekommen hast (z.B. durch andere Aktionskarten), kannst du diese wieder verlieren. Wenn du einen \emph{SOUK} nimmst, kannst du bis zu 2 deiner Handkarten entsorgen, du musst aber nicht.}
\end{tikzpicture}
\hspace{-0.6cm}
\begin{tikzpicture}
	\card
	\cardstrip
	\cardbanner{banner/white.png}
	\cardtitle{\scriptsize{Empfohlene 10er Sätze\qquad}}
	\cardcontent{\emph{Einführung:} \\ 
	Aufbau, Gewürzhändler, Kartograph, Lebenskünstler, Markgraf, Nomaden, Oase, Stallungen, Weberin, Wegkreuzung 

	\smallskip 
	
	\emph{Gelegenheiten:} \\ 
	Fahrender Händlerin, Feilscher, Fernstraße, Grenzdorf, Hexenkessel, Katzengold, Komplott, Pfad, Radmacherin, Souk 

	\smallskip

	\emph{Glückliche Pfade} (+ \textit{Basisspiel (2. Edition)}):\\ 
	Berserker, Fernstraße, Nomaden, Oase, Pfad, \textit{Bibliothek}, \textit{Geldverleiher}, \textit{Keller}, \textit{Thronsaal}, \textit{Werkstatt}

	\smallskip 
	
	\emph{Abenteuerfahrt}  (+ \textit{Basisspiel (2. Edition)}):\\ 
	Hexenhütte, Katzengold, Souk, Wachhund, Wegkreuzung, \textit{Jahrmarkt}, \textit{Laboratorium}, \textit{Torwächterin}, \textit{Umbau}, \textit{Vasall}

	\smallskip

	\emph{Geld für nichts} (+ \textit{Intrige (2. Edition)}):\\ 
	Kartograph, Lebenskünstler, Radmacherin, Tunnel, Weberin, \textit{Armenviertel}, \textit{Austausch}, \textit{Handlanger}, \textit{Kerkermeister}, \textit{Patrouille}

	\smallskip 
	
	\emph{Der Ball des Herzogs}  (+ \textit{Intrige (2. Edition)}):\\ 
	Gasthaus, Komplott, Pfad, Radmacherin, Wachhund, \textit{Anbau}, \textit{Harem}, \textit{Herzog}, \textit{Maskerade}, \textit{Verschwörer}}
\end{tikzpicture}
\hspace{-0.6cm}
\begin{tikzpicture}
	\card
	\cardstrip
	\cardbanner{banner/white.png}
	\cardtitle{\scriptsize{Empfohlene 10er Sätze\qquad}}
	\cardcontent{\emph{Reisende}  (+ \textit{Seaside (2. Edition)}):\\ 
	Fruchtbares Land, Kartograph, Souk, Stallungen, Wegkreuzung, \textit{Ausguck}, \textit{Beutelschneider}, \textit{Handelsschiff}, \textit{Insel}, \textit{Lagerhaus}

	\smallskip 
	
	\emph{Läufer}  (+ \textit{Seaside (2. Edition)}):\\ 
	Berserker, Hexenkessel, Nomaden, Radmacherin, Wachhund, \textit{Bazar}, \textit{Blockade}, \textit{Karawane}, \textit{Schmuggler}, \textit{Seefahrerin}
	
	\smallskip
	
	\emph{Händler und Diebe} (+ \textit{Abenteuer}):\\ 
	Berserker, Fahrende Händlerin, Feilscher, Gewürzhändler, Wachhund, \textit{Geisterwald}, \textit{Hafenstadt}, \textit{Pagin}, \textit{Verlorene Stadt}, \textit{Weinhändler}, \textit{\underline{Überfall}}

	\smallskip 
	
	\emph{Reisen}  (+ \textit{Abenteuer}):\\ 
	Fernstraße Gasthaus, Kartograph, Pfad, Wegkreuzung, \textit{Brückentroll}, \textit{Ferne Lande}, \textit{Kundschafterin}, \textit{Riese}, \textit{Wildhüterin}, \textit{\underline{Erbschaft}}, \textit{\underline{Expedition}}
	
	\smallskip
	
	\emph{Auf zur Party} (+ \textit{Nocturne}):\\ 
	Gasthaus, Kartograph, Komplott, Oase, Souk, \textit{Druidin (Geschenk des Berges, Gesch. d. Himmels, Gesch. d. Sonne)}, \textit{Getreuer Hund}, \textit{Konklave}, \textit{Schuster}, \textit{Werwolf}

	\smallskip 
	
	\emph{Schafe zählen}  (+ \textit{Nocturne}):\\ 
	Fruchtbares Land, Oase, Tunnel, Weberin, Wegkreuzung, \textit{Geheime Höhle}, \textit{Kobold}, \textit{Krypta}, \textit{Puka}, \textit{Schäferin}}
\end{tikzpicture}
\hspace{-0.6cm}
\begin{tikzpicture}
	\card
	\cardstrip
	\cardbanner{banner/white.png}
	\cardtitle{\scriptsize{Empfohlene 10er Sätze\qquad}}
	\cardcontent{\emph{Einfache Pläne}  (+ \textit{Empires}):\\ 
	Feilscher, Grenzdorf, Hexenkessel, Radmacherin, Stallungen, \textit{Forum}, \textit{Katapult/Felsen}, \textit{Patrizier/Handelsplatz}, \textit{Villa}, \textit{Zauberin}, \textit{\underline{Spende}}, \textit{\underline{Labyrinth}}

	\smallskip 
	
	\emph{Expansion}  (+ \textit{Empires}):\\ 
	Fruchtbares Land, Gewürzhändler, Oase, Stallungen, Tunnel, \textit{Feldlager/Diebesgut}, \textit{Ingenieurin}, \textit{Legionär}, \textit{Schlösser}, \textit{Zauber}, \textit{\underline{Brunnen}}, \textit{\underline{Schlachtfeld}}
	
	\smallskip
	
	\emph{Versüßte Deals} (+ \textit{Renaissance}):\\ 
	Aufbau, Feilscher, Gewürzhändler, Hexenhütte, Wachhund, \textit{Bergdorf}, \textit{Diener}, \textit{Fahnenträger}, \textit{Gewürze}, \textit{Seidenhändlerin}, \textit{\underline{Speicher}}

	\smallskip 
	
	\emph{Ein gesparter Penny}  (+ \textit{Renaissance}):\\ 
	Berserker, Fahrende Händlerin, Grenzdorf, Oase, Souk, \textit{Freibeuterin}, \textit{Goldmünze}, \textit{Patron}, \textit{Seher}, \textit{Zepter}, \textit{\underline{Gildenhaus}}, \textit{\underline{Kaserne}}}
\end{tikzpicture}
\hspace{-0.6cm}
\begin{tikzpicture}
	\card
	\cardstrip
	\cardbanner{banner/white.png}
	\cardtitle{\scriptsize{Empfohlene 10er Sätze\qquad}}
	\cardcontent{\emph{Big Blue} (+ \textit{Menagerie}):\\ 
	Fahrende Händlerin, Hexenhütte, Pfad, Tunnel, Weberin, \textit{Dorfanger}, \textit{Falknerin}, \textit{Hirtenhund}, \textit{Schlitten}, \textit{Schwarze Katze}, \textit{\underline{Verbannung}}, \textit{\underline{Weg der Schildkröte}}

	\smallskip 
	
	\emph{Kreuzung}  (+ \textit{Menagerie}):\\ 
	Aufbau, Fruchtbares Land, Nomaden, Radmacherin, Stallungen, \textit{Drahtzieher}, \textit{Herberge}, \textit{Kardinal}, \textit{Nachschub}, \textit{Pferdestall}, \textit{\underline{Weg der Maus (mit Wegkreuzung)}}, \textit{\underline{Wagnis}}
	
	\smallskip

	\emph{Längster Tunnel}  (+ \textit{Verbündete}):\\ 
	Feilscher, Lebenskünstler, Markgraf, Pfad, Tunnel, \textit{Hauptstadt}, \textit{Schreinerin}, \textit{Tand}, \textit{Vertrag}, \textit{Wirtin}, \textit{\underline{Gefolgschaft der Schreiber}}

	\smallskip 
	
	\emph{Expertise}  (+ \textit{Verbündete}):\\ 
	Fernstraße, Gasthaus, Gewürzhändler, Grenzdorf, Wegkreuzung, \textit{Barbar}, \textit{Bürger}, \textit{Spezialistin}, \textit{Untergebener}, \textit{Wegelagerer}, \textit{\underline{Freimauerloge}}}
\end{tikzpicture}
\hspace{-0.6cm}
\begin{tikzpicture}
	\card
	\cardstrip
	\cardbanner{banner/white.png}
	\cardtitle{Platzhalter\quad}
\end{tikzpicture}
\hspace{0.6cm}
