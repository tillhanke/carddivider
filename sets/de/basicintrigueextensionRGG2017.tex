% Basic settings for this card set
\renewcommand{\cardcolor}{basicgame}
\renewcommand{\cardextension}{Ergänzung}
\renewcommand{\cardextensiontitle}{Das Basisspiel}
\renewcommand{\seticon}{basic1.png}

\clearpage
\newpage
\section{\cardextension \ - \cardextensiontitle \ (Rio Grande Games 2017)}

\begin{tikzpicture}
	\card
	\cardstrip
	\cardbanner{banner/white.png}
	\cardicon{icons/coin.png}
	\cardprice{3}
	\cardtitle{Händlerin}
	\cardcontent{Du ziehst 1 Karte und erhältst + 1 Aktion. Wenn du in diesem Zug vor dem Ausspielen dieser \emph{HÄNDLERIN} noch kein Silber ausgespielt hast, erhältst du für das erste danach ausgespielte Silber +\coin[1]. Für jedes weitere ausgespielte Silber erhältst du keinen zusätzlichen Bonus. Hast du mehrere \emph{HÄNDLERINNEN} ausgespielt, erhältst du pro \emph{HÄNDLERIN} +\coin[1].}
\end{tikzpicture}
\hspace{-0.6cm}
\begin{tikzpicture}
	\card
	\cardstrip
	\cardbanner{banner/white.png}
	\cardicon{icons/coin.png}
	\cardprice{3}
	\cardtitle{Vasall}
	\cardcontent{Ist die aufgedeckte Karte eine Aktionskarte (auch ggf. kombinierte), \emph{darfst} du sie sofort ausspielen. Wenn du sie ausspielst, legst du sie in deinen Spielbereich und führst sofort die Anweisungen darauf aus. Dafür benötigst du keine zusätzliche Aktion. Das Ausspielen der Aktionskarte verbraucht auch keine freie oder zusätzliche Aktion, die du durch das Ausspielen anderer Karten bereits gesammelt hast.}
\end{tikzpicture}
\hspace{-0.6cm}
\begin{tikzpicture}
	\card
	\cardstrip
	\cardbanner{banner/white.png}
	\cardicon{icons/coin.png}
	\cardprice{3}
	\cardtitle{Vorbotin}
	\cardcontent{Du ziehst 1 Karte und erhältst + 1 Aktion. Schau dir deinen Ablagestapel an. Du \emph{darfst} eine Karte daraus auswählen und oben auf deinen Nachziehstapel legen. Die restlichen Karten (oder alle) legst du in beliebiger Reihenfolge zurück auf den Ablagestapel. Ist dein Ablagestapel leer, passiert nichts.}
\end{tikzpicture}
\hspace{-0.6cm}
\begin{tikzpicture}
	\card
	\cardstrip
	\cardbanner{banner/white.png}
	\cardicon{icons/coin.png}
	\cardprice{4}
	\cardtitle{Wilddiebin}
	\cardcontent{Du ziehst 1 Karte, erhältst + 1 Aktion und +\coin[1]. Dann schaust du, wie viele Vorratsstapel (Fluch-, Geld-, Punkte- und Aktionskarten) bereits leer sind. Ist kein Stapel leer, musst du keine Handkarten ablegen. Ist ein Stapel leer, legst du 1 Handkarte ab usw. Wenn du nicht so viele Karten auf der Hand hast, wie Vorratsstapel leer sind, legst du so viele Karten ab, wie du kannst.}
\end{tikzpicture}
\hspace{-0.6cm}
\begin{tikzpicture}
	\card
	\cardstrip
	\cardbanner{banner/white.png}
	\cardicon{icons/coin.png}
	\cardprice{5}
	\cardtitle{Banditin}
	\cardcontent{Zuerst nimmst du ein Gold vom Vorrat und legst es auf deinen Ablagestapel. Dann deckt jeder Mitspieler – beginnend bei deinem linken Mitspieler – die obersten zwei Karten seines Nachziehstapels auf. Deckt ein Spieler zwei Geldkarten (auch ggf. kombinierte) außer Kupfer auf, muss er \emph{eine} davon entsorgen. Dabei darf er selbst entscheiden, welche Geldkarte er entsorgt. Die andere Geldkarte wird – genauso wie alle anderen Karten – abgelegt. Deckt ein Spieler eine Geldkarte außer Kupfer sowie eine andere Karte (z. B. ein Kupfer oder eine beliebige Aktionskarte) auf, wird diese Geldkarte entsorgt. Die andere aufgedeckte Karte wird abgelegt.}
\end{tikzpicture}
\hspace{-0.6cm}
\begin{tikzpicture}
	\card
	\cardstrip
	\cardbanner{banner/white.png}
	\cardicon{icons/coin.png}
	\cardprice{5}
	\cardtitle{\footnotesize{Torwächterin}}
	\cardcontent{Du ziehst 1 Karte und erhältst + 1 Aktion. Dann siehst du dir die obersten 2 Karten deines Nachziehstapels an. Du kannst beide Karten entsorgen, beide Karten ablegen oder sie in beliebiger Reihenfolge zurück auf den Nachziehstapel legen. Du kannst aber auch eine entsorgen und eine ablegen, oder eine entsorgen und die andere zurück auf den Nachziehstapel legen, oder eine ablegen und die andere zurücklegen.}
\end{tikzpicture}
\hspace{-0.6cm}
\begin{tikzpicture}
	\card
	\cardstrip
	\cardbanner{banner/white.png}
	\cardicon{icons/coin.png}
	\cardprice{6}
	\cardtitle{Töpferei}
	\cardcontent{Nimm eine Karte vom Vorrat, die zu diesem Zeitpunkt maximal \coin[5] kostet. Du darfst kein zusätzliches \coin einsetzen, um dir eine teurere Karte zu nehmen. Außer \coin darf die Karte keine zusätzlichen Kosten enthalten. 

	\medskip

	Du darfst dir zum Beispiel keine Karte mit \potion (aus Alchemie) oder \hex (aus Empires) in den Kosten nehmen. Die genommene Karte nimmst du direkt auf die Hand. Anschließend legst du eine beliebige Handkarte (das kann die gerade genommene oder eine andere sein) oben auf deinen Nachziehstapel.}
\end{tikzpicture}
\hspace{-0.6cm}
\begin{tikzpicture}
	\card
	\cardstrip
	\cardbanner{banner/white.png}
	\cardtitle{\scriptsize{Empfohlene 10er Sätze\qquad}}
	\cardcontent{\emph{Erstes Spiel:}\\
	Burggraben, Dorf, Händlerin, Keller, Markt, Miliz, Mine, Schmiede, Umbau, Werkstatt

	\smallskip

	\emph{Verzerrte Größen:}\\
	Banditin, Bürokrat, Gärten, Hexe, Jahrmarkt, Kapelle, Thronsaal, Töpferei, Torwächterin, Werkstatt

	\smallskip

	\emph{Schleichweg:}\\
	Bürokrat, Dorf, Geldverleiher, Laboratorium, Jahrmarkt, Ratsversammlung, Töpferei, Torwächterin, Vasall, Vorbotin

	\smallskip

	\emph{Kunststück:}\\
	Bibliothek, Gärten, Jahrmarkt, Keller, Miliz, Ratsversammlung, Schmiede, Thronsaal, Vorbotin, Wilddiebin

	\smallskip

	\emph{Verbesserungen:}\\
	Burggraben, Geldverleiher, Händlerin, Hexe, Keller, Markt, Mine, Töpferei, Umbau, Wilddiebin

	\smallskip

	\emph{Silber \& Gold:}\\
	Banditin, Bürokrat, Geldverleiher, Händlerin, Kapelle, Laboratorium, Mine, Thronsaal, Vasall, Vorbotin}
\end{tikzpicture}
\hspace{-0.6cm}
\begin{tikzpicture}
	\card
	\cardstrip
	\cardbanner{banner/white.png}
	\cardtitle{\scriptsize{Empfohlene 10er Sätze\qquad}}
	\cardcontent{\emph{Basisspiel \& Die Intrige:}\\
	
	\smallskip

	\emph{Untergebene:}\\
	Bibliothek, Jahrmarkt, Keller, Torwächterin, Vasall, Adlige, Diplomatin, Handlanger, Höflinge, Lakai

	\smallskip

	\emph{Das große Ganze:}\\
	Markt, Miliz, Ratsversammlung, Töpferei, Werkstatt, Armenviertel, Bergwerk, Brücke, Mühle, Patrouille

	\smallskip

	\emph{Dekonstruktion:}\\
	Banditin, Dorf, Mine, Thronsaal, Umbau, Austausch, Diplomatin, Harem, Herumtreiberin, Trickser}
\end{tikzpicture}
\hspace{0.6cm}


\renewcommand{\cardcolor}{intrigue}
\renewcommand{\cardextension}{Ergänzung}
\renewcommand{\cardextensiontitle}{Die Intrige}
\renewcommand{\seticon}{intrigue1.png}

\clearpage
\newpage
\section{\cardextension \ - \cardextensiontitle \ (Rio Grande Games 2017)}

\begin{tikzpicture}
	\card
	\cardstrip
	\cardbanner{banner/white.png}
	\cardicon{icons/coin.png}
	\cardprice{2}
	\cardtitle{\scriptsize{Herumtreiberin}}
	\cardcontent{Die Karte, die du entsorgst oder vom Müllstapel nimmst, muss den Typ AKTION beinhalten, d. h. sie kann auch eine kombinierte Aktionskarte (z. B. MÜHLE) sein. Genommene Karten werden auf den Ablagestapel gelegt – es sei denn, auf der Karte steht etwas anderes. Wird eine Karte entsorgt, die einen speziellen Effekt beim Entsorgen hat, tritt dieser ein.}
\end{tikzpicture}
\hspace{-0.6cm}
\begin{tikzpicture}
	\card
	\cardstrip
	\cardbanner{banner/blue.png}
	\cardicon{icons/coin.png}
	\cardprice{4}
	\cardtitle{Diplomatin}
	\cardcontent{Diese Karte ist eine Aktions- und Reaktionskarte. Wird sie als Aktion in der Aktionsphase ausgespielt, nimmst du 2 Karten. Hast du dann 5 oder weniger Karten auf der Hand, erhältst du außerdem + 2 Aktionen. Spielt ein Mitspieler eine Angriffskarte aus und du hast zu diesem Zeitpunkt 5 oder mehr Karten auf der Hand, darfst du diese Karte – bevor der ausgespielte Angriff ausgeführt wird – aus der Hand aufdecken. Wenn du das tust, nimmst du diese DIPLOMATIN wieder auf die Hand, ziehst 2 Karten und legst dann 3 Karten (auch möglich inklusive dieser DIPLOMATIN) ab. Hast du dann immer noch 5 oder mehr Karten sowie eine DIPLOMATIN auf der Hand, darfst du die DIPLOMATIN noch einmal aufdecken – und dies so oft wiederholen wie du möchtest und die Bedingung der 5 oder mehr Karten auf der Hand erfüllt ist. Erst dann wird der Angriff ausgeführt. Hast du mehrere Reaktionskarten auf der Hand, mit denen du auf das Ausspielen einer Angriffskarte reagieren kannst, darfst du diese nacheinander in beliebiger Reihenfolge aufdecken.}
\end{tikzpicture}
\hspace{-0.6cm}
\begin{tikzpicture}
	\card
	\cardstrip
	\cardbanner{banner/white.png}
	\cardicon{icons/coin.png}
	\cardprice{4}
	\cardtitle{Geheimgang}
	\cardcontent{Du ziehst 2 Karten und erhältst + 1 Aktion. Dann nimmst du eine beliebige Karte aus deiner Hand (auch ggf. eine, die du gerade gezogen hast) und legst sie an eine beliebige Stelle in deinen Nachziehstapel. Du darfst sie oben drauf, unten drunter oder irgendwo in die Mitte legen. Du darfst dabei die Karten deines Nachziehstapels zählen, aber nicht ansehen. Befinden sich keine Karten in deinem Nachziehstapel, wird die zurückgelegte Karte zur einzigen Karte in deinem Nachziehstapel.}
\end{tikzpicture}
\hspace{-0.6cm}
\begin{tikzpicture}
	\card
	\cardstrip
	\cardbanner{banner/whitegreen.png}
	\cardicon{icons/coin.png}
	\cardprice{4}
	\cardtitle{Mühle}
	\cardcontent{Diese Karte ist eine kombinierte Aktions- und Punktekarte. Als Punktekarte bringt sie beim Zählen der Punkte 1. Spielst du die MÜHLE als Aktionskarte aus, ziehst du 1 Karte und erhältst + 1 Aktion. Du darfst 2 Karten aus deiner Hand ablegen. Wenn du das tust, erhältst du + 2 . Tust du das nicht (weil du zum Beispiel nicht genügend Karten auf der Hand hast), erhältst du nichts. Nur, wenn du nicht mehr als eine Karte auf der Hand hast, darfst du genau eine Karte ablegen, erhältst dafür aber kein GELD.}
\end{tikzpicture}
\hspace{-0.6cm}
\begin{tikzpicture}
	\card
	\cardstrip
	\cardbanner{banner/white.png}
	\cardicon{icons/coin.png}
	\cardprice{5}
	\cardtitle{Austausch}
	\cardcontent{Entsorge zuerst eine Karte aus deiner Hand. Dann nimmst du dir eine Karte vom Vorrat, die maximal 2 mehr kostet als die entsorgte Karte. Eine Karte kostet nur dann maximal 2 mehr, wenn die restlichen Kosten (z. B. TRANK aus Empires oder SCHULDEN aus Alchemie) gleich oder niedriger sind. Wenn die genommene Karte eine Aktions- und/oder Geldkarte ist, legst du die Karte oben auf deinen Nachziehstapel. Ansonsten legst du die Karte auf den Ablagestapel. Ist die genommene Karte eine Punktekarte, nimmt sich jeder Mitspieler – beginnend bei deinem linken Mitspieler – einen Fluch. Ist die genommene Karte eine Punktekarte sowie eine Aktions- oder Geldkarte (z. B. MÜHLE), legst du die Karte oben auf deinen Nachziehstapel und jeder Mitspieler muss sich einen Fluch nehmen.}
\end{tikzpicture}
\hspace{-0.6cm}
\begin{tikzpicture}
	\card
	\cardstrip
	\cardbanner{banner/white.png}
	\cardicon{icons/coin.png}
	\cardprice{5}
	\cardtitle{Höflinge}
	\cardcontent{Decke eine Karte aus deiner Hand auf. Zähle dann die Typen, denen diese Karte angehört – also AKTION, GELD, REAKTION, ANGRIFF, PUNKTE, FLUCH etc. Pro Typ, dem die Karte angehört, entscheidest du dich für eine der vier angegebenen Optionen. Dabei darfst du keine der Optionen doppelt auswählen. Wenn du zum Beispiel eine PATROUILLE (AKTION) aufdeckst, darfst du eine Option auswählen, deckst du einen KARAWANENWÄCHTER aus Abenteuer (AKTION – DAUER – REAKTION) auf, darfst du 3 unterschiedliche Optionen wählen. Entscheidest du dich für das Gold, legst du dieses auf den Ablagestapel. Kannst du keine Handkarte aufdecken, erhältst du nichts.}
\end{tikzpicture}
\hspace{-0.6cm}
\begin{tikzpicture}
	\card
	\cardstrip
	\cardbanner{banner/white.png}
	\cardicon{icons/coin.png}
	\cardprice{5}
	\cardtitle{Patrouille}
	\cardcontent{Ziehe zuerst 3 Karten. Decke dann die obersten 4 Karten deines Nachziehstapels auf. So aufgedeckte Punktekarten (auch ggf. kombinierte) und Flüche nimmst du alle auf die Hand. Die restlichen Karten legst du in beliebiger Reihenfolge zurück auf den Nachziehstapel.}
\end{tikzpicture}
\hspace{-0.6cm}
\begin{tikzpicture}
	\card
	\cardstrip
	\cardbanner{banner/white.png}
	\cardtitle{\scriptsize{Empfohlene 10er Sätze\qquad}}
	\cardcontent{\emph{Siegestanz:}\\
	Adlige, Austausch, Baron, Eisenhütte, Harem, Herzog, Höflinge, Maskerade, Mühle, Patrouille

	\smallskip

	\emph{Verschwörung:}\\
	Bergwerk, Eisenhütte, Geheimgang, Handelsposten, Handlanger, Herumtreiberin, Kerkermeister, Trickser, Verschwörer, Verwalter

	\smallskip

	\emph{Beste Wünsche:}\\
	Anbau, Armenviertel, Baron, Burghof, Diplomatin, Geheimgang, Herzog, Kerkermeister, Verschwörer, Wunschbrunnen}
\end{tikzpicture}
\hspace{-0.6cm}
\begin{tikzpicture}
	\card
	\cardstrip
	\cardbanner{banner/white.png}
	\cardtitle{Platzhalter\quad}
\end{tikzpicture}
\hspace{0.6cm}